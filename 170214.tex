\newpage
\section{Checking Expressions}

$\Gamma \vd e \synthesis \tau$
\begin{mathpar}
\inferr{\Gamma \vd x \synthesis \tau}{\Gamma(x) = \tau}

\inferr{\Gamma \vd e_1\ e_2 \synthesis \tau'}
       {\Gamma \vd e_1 \synthesis \tau_1 \\
        \Gamma \vd \tau_1 \whn \tau \arrow \tau' \\
        \Gamma \vd e_2 \checking \tau}
\end{mathpar}

\section{Type-Directed Translation / Syntax-Directed Translation}
A more accurate name: ``Typing-derivation-directed translation''.
We proceed by the analysis of the typing derivation of the rules.

\newcommand{\target}[1]{\textcolor{green}{#1}}
\newcommand{\tto}{\translatesto}
Let's represent the source and target languages in different colors, to
indicate that they are different.\\

Property:\\
$\Gamma \vd e \of \tau$ if and only if $\exists\bind{\target{e}}{\Gamma \vd e \of \tau \translatesto \target{e}}$.\\

We also want:\\
If $\Gamma \vd e \of \tau \tto \target{e}$, something like
$\target{\Gamma \vd e \of \tau}$.\\
But we have no concept of $\target{\Gamma}$ or $\target{\tau}$ or its derivations.\\
Instead:\\
Property:\\
If $\Gamma \vd e \of \tau \tto \target{e}$
and $\tau \tto \target{\tau}$
and $\Gamma \tto \target{\Gamma}$,
then $\target{\Gamma \vd e \of \tau}$. \\
Why not $\Gamma \vd \tau \of \type \tto \tau$.\\

Simply, we'll use `` If $\Gamma \vd e \of \tau \tto \target{e}$
then $\target{\Gamma} \vd \target{e} \of \target{\tau}$

\subsection{Coherence}
For Terms:\\
Suppose $\Gamma \vd e \of \tau \tto \target{e}$ and
$\Gamma \vd e \of \tau \tto \target{e'}$.\\
$\target{\Gamma \vd e \cong e' \of \tau}$.\\
This is too hard to even define, this is left to graduate courses.
We aspire to it but it's too much of a pain to actually do.\\

For Types:\\
Suppose $\Gamma \vd c \of k \tto \target{c}$ and
$\Gamma \vd c \of k \tto \target{c'}$.\\
Then,\\
$\Gamma \vd c \equiv c' \of k$. \\
This is not an aspiration, we cannot live without this. \\
The 2nd property above can't even be made without this, but it doesn't have
to be kind directed. And instead, we'll just make it syntax directed, which
will trivially prove that the two are equivalent.


\subsection{Definition of $\target{e}$}
\begin{align*}
\target{\alpha} &= \alpha \\
\target{\tau_1 \times \tau_2} &= \target{\tau_1} \times \target{\tau_2} \\
\target{\tau_1 \arrow \tau_2} &= \unit \arrow \target{\tau_1} \arrow \target{\tau_2} \\
&\ldots \\
\target{\epsilon} &= \epsilon \\
\target{\Gamma, x \of \tau} &= \target{\Gamma}, x \of \target{\tau} \\
\target{\Gamma, \alpha \of k} &= \target{\Gamma}, \alpha \of \target{k} \\
\end{align*}

Convoluted example:\\
\begin{mathpar}
\inferr{\Gamma \vd \lambda\bind{x \of \tau}{e} \of \tau \arrow \tau'
                    \tto \lambda\bind{z \of \unit}{\lambda\bind{x \of \target{\tau}}{\target{e}}}}
       {\Gamma \vd \tau \of \type \\
        \Gamma, x \of \tau \vd e \of \tau \tto \target{e}}
\end{mathpar}
NOTE: the right part (after $\tto$) indicates the need to shift in terms of
debruijn indices. \\

More: \\
\begin{mathpar}
\inferr{\Gamma \vd e_1\ e_2 \of \tau' \tto \target{e_1} <> \target{e_2}}
       {\Gamma \vd e_1 \of \tau \arrow \tau' \tto
          \target{e_1}^{\of \target{\tau_1 \arrow \tau_2}
                          = \unit \arrow \target{\tau} \arrow \target{\tau'}} \\
        \Gamma \vd e_2 \of \tau \tto \target{e_2}^{\of \target{\tau}}}
\end{mathpar}

More (only well typed things translate): \\
\begin{mathpar}
\inferr{\Gamma \vd e_1\ e_2 \synthesis \tau' \tto \target{e_1} <> \target{e_2}} % TODO what is <>
       {\Gamma \vd e_1 \synthesis \tau_1 \tto \target{e_1} \\
        \Gamma \vd e_1 \whn \tau \arrow \tau' \\
        \Gamma \vd e_2 \synthesis \tau_2 \tto \target{e_2} \\
        \Gamma \vd \tau_2 \ace \tau \of \type}
        % NOTE: we don't need this last one, we can optimize by skipping this rule
\end{mathpar}


\subsection{$\Gamma \vd e \of \tau \translatesto \target{e}$}
\begin{mathpar}
\inferr{\Gamma \vd x \of \tau \translatesto \target{x}}{\Gamma(x) = \tau}

\inferr{\Gamma \vd \langle e_1, e_2 \rangle \of \tau_1 \times \tau_2
                        \tto \target{\langle e_1, e_2 \rangle}}
       {\Gamma \vd e_1 \of \tau_1 \tto \target{e_1} \\
        \Gamma \vd e_2 \of \tau_2 \tto \target{e_2}}

\inferr{\Gamma \vd \pi_1 e \of \tau_1 \tto \target{\pi_1 e}}
       {\Gamma \vd e \of \tau_1 \times \tau_2 \tto \target{e}}
\end{mathpar}
