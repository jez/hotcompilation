\section{Closure Conversion}

A closure is a tuple containing code and the environment that will be passed
in as an additional argument to the code. The code is closed, with no open
terms. \\

Concrete example: \\
$\lambda\bind{x \of \intt}{x + y + z}$ \\
$\leadsto
\langle (\lambda\bind{x \of \intt}{\lambda\bind{\env \of \intt \times \intt}
          {\letbind{y}{\pi_0 \env}{\letbind{z}{\pi_1 \env}{x + y + z}}}}),
\langle y, z \rangle \rangle$ \\
$\ap{e}{5}$ \\
$\leadsto
\letbind{f}{\pi_0 e}{\letbind{\env}{\pi_1 e}{\ap{\ap{f}{5}}{\env}}}$ \\

This can get really messy and really inefficient, so what we really want
to do is convert some curried and some tupled functions and turn them into
some special internal representation for multi-argument functions. \\

Types may be different if the environment is different, so when converting,
we can try something like, where we don't care about the existential, since
we won't be manipulating it: \\
$\target{\tau_1 \arrow \tau_2} = \exists\bind{\alpha_\env \of \type}
  {(\tau_1 \arrow \alpha_\env \arrow \tau_2) \times \alpha_\env}$

\subsection{Target Language (IL-Closure)}
\begin{flalign*}
\Delta;\Gamma &\vd e \of 0 &\\
\Delta;\Gamma &\vd v \of \tau &\\
\Gamma &\bnfdef \epsilon \bnfalt \Gamma, x \of \tau &\\
\Delta &\bnfdef \epsilon \bnfalt \Delta, \alpha \of k &\\
\end{flalign*}

Only rule additional we need: \\
\begin{mathpar}
\inferr{\Delta; \Gamma \vd \lambda\bind{x \of \tau}{e} \of \neg\tau}
       {\Delta \vd \tau \of \type \\
        \Gamma; (\epsilon, x \of \tau) \vd e \of 0}
\end{mathpar}

\subsection{Type Translation}
\begin{flalign*}
\target{\alpha} &= \alpha &\\
&\vdots &\\
\target{\neg\tau} &= \exists\bind{\alpha_\env \of \type}
  {\neg(\target{\tau} \times \alpha_\env) \times \alpha_\env} &\\
\target{x[\tau_1, \dots, \tau_2]} &= x[\target{\tau_1}, \dots, \target{\tau_n}] &\\
\end{flalign*}
How would we deal with $\target{\forall\bind{\alpha \of k}{\tau}}$? Turns out
it's really hard. But because of how we got rid of them back during CPS
conversion, we don't even have to deal with it anymore!

\subsection{Type Principle}
If $\Delta; \Gamma \vd e \of 0 \tto \target{e}$ then $\target{\Delta}; \target{\Gamma} \vd \target{e} \of 0$. \\
If $\Delta; \Gamma \vd v \of \tau \tto \target{v}$ then $\target{\Delta}; \target{\Gamma} \vd \target{v} \of \target{\tau}$. \\

\subsection{$\Delta; \Gamma \vd e \of 0 \tto \target{e}$,
            $\Delta; \Gamma \vd v \of \tau \tto \target{v}$}
\begin{mathpar}
\inferr{\Delta; \Gamma \vd x \of \tau \tto x}{\Gamma(x) = \tau}

\inferr{\Delta; \Gamma \vd \langle v_1, \dots v_n \rangle \of x[\tau_1, \dots, \tau_n] \tto
          \langle \target{v_1}, \dots, \target{v_n} \rangle}
       {\Delta; \Gamma \vd v_i \of \tau_i \tto \target{v_i}}

\small
\inferr{\Delta; \Gamma \vd \lambda\bind{x \of \tau}{e} \of \neg\tau \tto
        \pack[x[\target{\tau_1}, \dots, \target{\tau_n}],
        \langle \\
          (\lambda\bind{y \of \tau \times x[\target{\tau_1}, \dots, \target{\tau_n}]}{
            \letbind{x}{\pi_0 y}{\letbind{\env}{\pi_1 y}{
              \letbind{x_1}{\pi_0\env}{\dots \letbind{x_n}{\pi_{n - 1}\env}{\target{e}}}
            }}
          }),
        \\ \langle x_1, \dots, x_n \rangle \rangle] \as
        \exists\bind{\alpha_\env \of \type}
          {\neg(\target{\tau} \times \alpha_\env) \times \alpha_\env}
       }
       {\Delta \vd \tau \of \type \\
        \Delta; \Gamma, x \of \tau \vd e \of 0 \tto \target{e} \\
        \Gamma = x_1 \of \tau_1, \dots, x_n \of \tau_n}
\normalfont

\inferr{\Delta; \Gamma \vd \ap{v_1}{v_2} \of 0 \tto
          \unpack[\alpha_\env, x] = \target{v_1} \texttt{in}
            \letbind{f}{\pi_0 x}{\letbind{\env}{\pi_1 x}{
              f \langle \target{v_2}, \env \rangle
            }}
       }
       {\Delta; \Gamma \vd v_1 \of \neg\tau \tto
          \target{v_1}^{\of \target{\neg\tau} =
                        \exists\bind{\alpha}{\neg(\target\tau \times \alpha) \times \alpha}} \\
        \Delta; \Gamma \vd v_2 \of \tau \tto \target{v_2}^{\of \target\tau}}
\end{mathpar}

\vspace{1cm}
% TODO
To solve some inefficiency, instead of passing around the environment,
instead, pass around the closure. \\
In environment passing, we had 
$\target{\tau_1 \arrow \tau_2} = \exists\bind{\alpha_\env \of \type}
  {(\tau_1 \times \alpha_\env \arrow \tau_2) \times \alpha_\env}$. \\
In closure passing, we have
$\mu\bind{\beta}{\exists\bind{\alpha_\env}{\target{\tau_1} \times \beta \arrow \target{\tau_2} \times \alpha_\env}}$. \\

We can apparently use this to understand objected oriented programming better
and some of the research might not even be all that wrong. \\

There's more stuff one can do $\dots$. \\


\section{Hoisting}
Type Translation: \\
$\target{\tau} = \tau$ \\

\subsection{Target Language: IL-Hoist}
\begin{align*}
c & \bnfdef \dots \bnfalt \forall\bind{\alpha \of k}{c} \\
e & \bnfdef \dots \\
v & \bnfdef \dots \bnfalt \lambda\bind{x \of \cancel{\tau}}{e}
    \bnfalt v[c] \bnfalt \cancel{\Lambda\bind{\alpha \of k}{v}} \\
f & \bnfdef \lambda\bind{x \of \tau}{e} \bnfalt \Lambda\bind{\alpha \of k}{f} \\
b & \bnfdef x \mathrel{=} f \\
P & \bnfdef \texttt{let}~b~\texttt{in}~P \bnfalt e \\
\end{align*}
To hoist the type: `type-erasure semantics`.
