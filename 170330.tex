\subsection{$\sigma \splitsto \sd{k}{\tau}$}
Let's revisit all of the above rules in the Debruijn world.

% TODO: motivation?
\begin{mathpar}
\inferr{\Gamma \vd \bind{m}{s} \of \Gamma', A}
       {\Gamma \vd s \of \Gamma' \\ \Gamma \vd m \of A[s]}

\inferr{\Gamma, \Gamma' \vd \lift k \of \Gamma}
       {|\Gamma'| = k}
\end{mathpar}

If $\Gamma \vd \sigma \of \sig$ and $\sigma \splitsto \sd{\alpha \of k}{\tau}$
then $\Gamma \vd k \of \kind$, $\Gamma, \alpha \of k \vd \tau \of \type$. \\

Now the Debruijn form: \\
If $\Gamma \vd \sigma \of \sig$ and $\sigma \splitsto \sd{k}{\tau}$
then $\Gamma \vd k \of \kind$ and $\Gamma, k \vd \tau \of \type$.

\begin{mathpar}
  \inferr{1 \splitsto \sd{1}{\unit}}{\strut}

  \inferr{\satom{k} \splitsto \sd{k}{\unit}}{\strut}

  \inferr{\datom{\tau} \splitsto \sd{1}{\tau[\lift]}}{\strut}

  % See NOTE 0, 1, 2 for info on how we got this
  \inferr{\Pia{\sigma_1}{\sigma_2} \splitsto
            \sd{\Pi\bind{k_1}{k_2}}
              {\forall\bind{k_1[\lift]}
                {\tau_1[\bind{0}{\lift^2}] \arrow
                 \tau_2[\bind{\ap{1}{0}}{\bind{0}{\lift^2}}]}}}
         {\sigma_1 \splitsto \sd{k_1}{\tau_1} \\
          \sigma_2 \splitsto \sd{k_2}{\tau_2}}

  % NOTE 0 These are the derivations on how we got the case for Pi app
  %\inferr{\Gamma, A \vd 0 \of A[\lift]}{\strut}
  %\inferr{\Gamma \vd i \of A[\lift^{i + 1}]}{\Gamma(i) = A}
  %\inferr{\Gamma, \Pi\bind{k_1}{k_2}, k_1[\lift] \vd \bind{0}{\lift^2} \of \Gamma, k_1}
  %       {\Gamma, \Pi\bind{k_1}{k_2}, k_1[\lift] \vd 0 \of k_1[\lift^2] \\
  %        \Gamma, \Pi\bind{k_1}{k_2}, k_1[\lift] \vd \lift^2 \of \Gamma} \\
  %\inferr{\Gamma, \Pi\bind{k_1}{k_2}, k_1[\lift] \vd
  %          \bind{\ap{1}{0}}{\bind{0}{\lift^2}} \of \Gamma, k_1, k_2}
  %       {\inferr{\Gamma, \Pi\bind{k_1}{k_2}, k_1[\lift] \vd
  %                  \ap{1}{0} \of k_2[\bind{0}{\lift^2}]}
  %               {\dots \vd 1 \of (\Pi\bind{k_1}{k_2})[\lift^2] \\
  %                \dots \vd 0 \of k_1[\lift^2]}
  %        \\ \dots } \\

  \inferr{\Pig{\of \sigma_1}{\sigma_2} \splitsto \sd{1}{
              \forall\bind{k_1[\lift]}{\tau_1[0.\lift^2] \arrow
              \exists\bind{k_2[0.\lift^2]}{\tau_2[0.1.\lift^3]}}}}
         {\sigma_1 \splitsto \sd{k_1}{\tau_1} \\
          \sigma_2 \splitsto \sd{k_2}{\tau_2}}

  % NOTE 3 These is one of the derivations for Pi gen
  %\inferr{\Gamma, 1, k_1[\lift], k_2[0.\lift^2] \vd 0.1.\lift^3 \of \Gamma, k_1, k_2}
  %       {\dots \vd 0 \of k_2[1.\lift^3] \\
  %        \inferr{\dots \vd 1.\lift^3 \of \Gamma, k_1}
  %               {\dots \vd 1 \of k_1[\lift^3] \\
  %                \dots \vd \lift^3 \of \Gamma}}

  \inferr{\Sigma\bind{\sigma_1}{\sigma_2} \splitsto
            \sd{
              \Sigma\bind{k_1}{k_2}
            }{
              \tau_1[\pi_1 0. \lift] \times
              \tau_2[\pi_2 0. \pi_1 0. \lift]
            }
         }
         {\sigma_1 \splitsto \sd{k_1}{\tau_1} \\
          \sigma_2 \splitsto \sd{k_2}{\tau_2}}

  % NOTE 4 These are the derivations for sigma
  %\inferrule{
  %  \inferr{\Gamma, \Sigma\bind{k_1}{k_2} \vd \pi_2 0 \of k_2[\pi_1 0.\lift]}
  %         {\Gamma, \Sigma\bind{k_1}{k_2} \vd 0 \of \Sigma\bind{k_1[\lift]}{k_2[0.\lift^2]}}
  %  \inferr{\Gamma, \Sigma\bind{k_1}{k_2} \vd \pi_1 0. \lift \of \Gamma, k_1}
  %         {\inferr{\Gamma, \Sigma\bind{k_1}{k_2} \vd \pi_1 0 \of k_1[\lift]}
  %                 {\Gamma, \Sigma\bind{k_1}{k_2} \vd 0 \of
  %                    \Sigma\bind{k_1[\lift]}{k_2[0.\lift^2]}} \\
  %          \Gamma, \Sigma\bind{k_1}{k_2} \vd \lift \of \Gamma}
  %}{
  %  \Gamma, \Sigma\bind{k_1}{k_2} \vd \pi_2 0. \pi_1 0. \lift \of \Gamma, k_1, k_2
  %}
\end{mathpar}

%% NOTE 1
%$\Gamma \vd \Pia{\sigma_1}{\sigma_2} \of \sig$ \\
%$\Gamma \vd \sigma_1 \of \sig$ \\
%$\Gamma, k_1 \vd \sigma_2 \of \sig$ \\
%$\Gamma \vd k_1 \of \kind$ \\
%$\Gamma, k_1 \vd \tau_1 \of \type$ \\
%$\Gamma, k_1 \vd k_2 \of \kind$ \\
%$\Gamma, k_1, k_2 \vd \tau_2 \of \type$ \\
%
%where: \\
%$\sigma_1 \splitsto \sd{k_1}{\tau_1}$ \\
%$\sigma_2 \splitsto \sd{k_2}{\tau_2}$ \\
%
%% NOTE 2
%$(\Pi\bind{k_1}{k_2})[\lift^i] = \Pi\bind{k_1[\lift^i]}{k_2[\bind{0}{\lift^{i + 1}}]}$

\subsection{$\target{\Gamma} \tto \Gamma$}
\begin{flalign*}
  \target{\epsilon} &= \epsilon &\\
  \target{\Gamma, \alpha \of k} &= \target{\Gamma}, \alpha \of k &\\
  \target{\Gamma, x \of \tau} &= \target{\Gamma}, x \of \tau &\\
  \target{\Gamma, \alpha/s \of \sigma} &= \target{\Gamma}, \alpha \of k, s \of \tau &\\
\end{flalign*}
*NOTE: in the last one, $\sigma \splitsto \sd{\alpha \of k}{\tau}$

\subsection{$\Gamma \vd_P M \of \sigma \splitsto \sd{c}{e}$}
\begin{mathpar}
  \inferrule{
    \alpha/s \of \sigma \in \Gamma
  }{ % Coding note: algorithmically, we will be inferring \sigma, and it will
     % actually be the best signature (eg: \singleton{\alpha : \sigma})
    \Gamma \vdp s \of \sigma \splitsto \sd{\alpha}{s}
  }

  \axiom{\Gamma \vdp \ast \of 1 \splitsto \sd{\ast}{\pair{}}}

  \inferrule{
    \Gamma \vd c \of k
  }{
    \Gamma \vdp \satom{c} \of \satom{k} \splitsto \sd{c}{\pair{}}
  }

  \inferrule{
    \Gamma \vd e \of \tau
  }{ % note the tau is shifted, but the e is not shifted, care for debruijn
    \Gamma \vd \datom{e} \of \datom{\tau} \splitsto \sd{\ast}{e}
  }

  \inferrule{
    \Gamma \vd \sigma_1 \of \sig \\
    \Gamma, \alpha/s \of \sigma_1 \vdp M \of \sigma_2 \splitsto \sd{c}{e} \\
    \sigma_1 \splitsto \sd{\alpha_1 \of k_1}{\tau_1}
  }{
    \Gamma \vd \lambdaa{\alpha/s \of \sigma_1}{M} \of \Pi\bind{\alpha \of \sigma_1}{\sigma_2}
      \splitsto
    \sd{
      \lambda\bind{\alpha \of k_1}{c}
    }{
      \Lambda\bind{\alpha \of k_1}{
        \lambda\bind{s \of \subst{\alpha}{\alpha_1}{\tau_1}}{
          e
        }
      }
    }
  }
\end{mathpar}
