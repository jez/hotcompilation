% TODO: now names
\subsection{Named Signatures}
\subsubsection{Target Language}
\begin{flalign*}
\name &\bnfdef \dots &\\
\sigma &\bnfdef \dots \bnfalt \name \of \sigma &\\
\end{flalign*}

$\Fst{\name \of \sigma} = \Fst{\sigma}$ \\
$\singleton{c \of \name \of \sigma} = \name \of \singleton{c \of \sigma}$ \\

\subsubsection{Typing Judgements}
\begin{mathpar}
\inferrule{
  \Gamma \vd \sigma \of \sig
}{
  \Gamma \vd \name \of \sigma \of \sig
}

\inferrule{
  \Gamma \vdk M \of \sigma
}{
  \Gamma \vdk \inn{\name}{M} \of (\name \of \sigma)
}

\inferrule{
  \Gamma \vdk M \of (\name \of \sigma)
}{
  \Gamma \vdk \outn{M} \of \sigma
}

\inferrule{
  \Gamma \vd \Fst{M} \gg c
}{
  \Gamma \vd \Fst{\inn{\name}{M}} \gg c \\
  \Gamma \vd \Fst{\outn{M}} \gg c
}

\inferrule{
  \Gamma \vdp M \of \name \of \sigma' \\
  \Gamma \vdp \outn{M} \of \sigma
}{
  \Gamma \vdp M \of \name \of \sigma
}

\inferrule{
  \Gamma \vdk M \synthesis \sigma
}{
  \Gamma \vdk \inn{M} \of \name \of \sigma
}

\inferrule{
  \Gamma \vdk M \synthesis \name \of \sigma
}{
  \Gamma \vdk \outn{M} \synthesis \sigma
}

\inferrule{
  \sigma \splitsto \sd{\alpha \of k}{\tau}
}{
  \name \of \sigma \splitsto \sd{\alpha \of k}{\tau}
}
\end{mathpar}

% TODO: coding notes for yesterday to simplify our life
\begin{mathpar}
\inferrule{
  \Gamma \vdp M \synthesis \sigma' \splitsto \sd{c}{e} \\
  \cancel{\Gamma \vd \sigma' \unlhd \sigma} \\ % cancel because already well-formed
  \cancel{\sigma \splitsto \target{\bind{\alpha \of k}{\tau}}} \\
  \cancel{\sigma' \splitsto \target{\bind{\alpha \of k'}{\tau'}}} \\
  \cancel{\Gamma, \alpha \of k' \vd \tau \equiv \tau' \of \type} \\
  \Gamma \vd \sigma' \leqq \sigma
}{
  \Gamma \vdp M \synthesis \sigma \splitsto \sd{c}{e}
}
\end{mathpar}

And now we need to add in the appropriate rules \\
\subsubsection{$\Gamma \vd \sigma \leqq \sigma'$}
\begin{mathpar}
\axiom{\Gamma \vd 1 \leqq 1}

\axiom{\Gamma \vd \satom{k} \leqq \satom{k'}}

% For this one, we cancel {\Gamma \tau \leqq \tau' \of \type}
\axiom{\Gamma \vd \datom{\tau} \leqq \datom{\tau}}

\inferrule{
  \Gamma \vd \sigma_1 \leqq \sigma_1' \\
  \Gamma, \alpha \of \Fst{\sigma_1} \vd \sigma_2 \leqq \sigma_2' \\
}{
  \Gamma \vd \Sigma\bind{\alpha \of \sigma_1}{\sigma_2} \leqq
             \Sigma\bind{\alpha \of \sigma_1'}{\sigma_2'}
}

\inferrule{
  \Gamma \vd \Pig{\alpha \of \sigma_1}{\sigma_2} \equiv
             \Pig{\alpha \of \sigma_1'}{\sigma_2'}
}{
  \Gamma \vd \Pig{\alpha \of \sigma_1}{\sigma_2} \leqq
             \Pig{\alpha \of \sigma_1'}{\sigma_2'}
}
\end{mathpar}


% New topic
\section{GC}
\begin{enumerate}
\item Mark and Sweep
\item Copying
\item Reference Counting
\end{enumerate}

For the first two, the idea is that we have pointers to objects in the heap
called the "roots". These objects have pointers to other objects. Accessible
objects are objects we can encounter by following some set of pointers from
the roots. Goal is to keep accessible objects and get rid of the rest. \\

Reference counting cannot deal with circular pointers. So we don't even
want to talk about it really. \\

Copying has the best asymptotic time (only traverse accessible objects). \\
Mark is just as good, but the sweep phase is too expensive (the whole heap). \\
Spacial locality is a huge benefit of copying (because defragmentation). \\
Copying sacrifices about half of the heap space. \\
Copying also has trouble with ``mutation'' (we will talk about this later). \\
Mark-sweep supports pinning (eg: C\#) (for external devices that cannot deal
with copying). \\

\subsection{Copying}
Stop-the-World/Incremental/Concurrent \\
Stop-the-World is easiest, and other than its undesirably behavior, it has
no potential memory-leak problems. \\

Semispace/Generational \\
EG: 2-generation generational copying collector \\
Make 3 segments, "Minor", "Major", and "2-Space" \\
Algo: we allocate in minor until we run out; collect into the major space \\
The reason this works is because objects in the major heap cannot point
to objects in the minor heap in the absence of mutation, which we have in
our functional languages. \\
Once we run out of memory in the major space, we collect it into the 2-space,
which then becomes the major space, while the old major space becomes the
2-space. \\
Note that the Major-collection is certainly more expensive. \\

This has a major limitation when we DO have mutation. We lose most of the
performance if we have to traverse everything in the end. \\
Thus, to fix this, we maintain a reference table where we maintain pointers
to everything we assign. Everytime we make the assignment, we will need to
update this table. Each generation other than the first gets its own
reference table. \\

Another problem: Pointer identification \\
``Conservative'', assume it's a pointer and trace, unless it's definitely not,
but this does not work at all with copying (because value must change). \\
``Tag bit'', tag each object with a value that decides whether or not the
value is a pointer. Downside: size of int now becomes 31 instead of 32. \\
``Tag word'', header word has bitmask that tells what values are pointers. \\
``Tag free'', use the types to check whether or not something is traceable. \\

with ``tag bit'', But then how do we get Int32? ``boxing'' (pass-by-reference)
