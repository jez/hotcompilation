\documentclass[11pt]{article}

%\usepackage{amsmath}
\usepackage{mathtools}
\usepackage{latexsym}
\usepackage{proof}
\usepackage[margin=0.5in]{geometry}
\usepackage{mathpartir}
\usepackage{graphicx} % for the nice downarrow
\usepackage{xifthen}
\usepackage{cancel}
\usepackage{tikz}

\usepackage{titlesec}
\titleformat*{\section}{\large\bfseries}
\titleformat*{\subsection}{\bfseries}
%\renewcommand{\thesetion}{\arabic{section}}
%\renewcommand{\thesubsection}{(\alpha{subsection})}


\usepackage{listings}
\usepackage{color}
\definecolor{eclipseBlue}{RGB}{42,0.0,255}
\definecolor{eclipseGreen}{RGB}{63,127,95}
\lstset {
  basicstyle=\small\ttfamily,
  captionpos=b,
  tabsize=2,
  columns=fixed,
  breaklines=true,
  mathescape=true,
% frame=l,
% numbers=left,
% numberstyle=\small\ttfamily,
  morekeywords= {
    EQUAL, GREATER, LESS, NONE, SOME, abstraction, abstype, and, andalso, array, as, before, bool, case, char, datatype, do, else, end, eqtype, exception, exn, false, fn, fun, functor, handle, if, in, include, infix, infixr, int, let, list, local, nil, nonfix, not, o, of, op, open, option, orelse, overload, print, raise, real, rec, ref, sharing, sig, signature, string, struct, structure, substring, then, true, type, unit, val, vector, where, while, with, withtype, word
  },
  morestring=[b]",
  morecomment=[s]{(*}{*)},
  stringstyle=\color{black},
  identifierstyle=\color{eclipseBlue},
  keywordstyle=\color{red},
  commentstyle=\color{eclipseGreen}
}

\setlength\parindent{0pt}

% samepage doesn't quite work in some cases, not sure why
% TODO: 16pt is an approximation
\newenvironment{grouped}[1]{\begin{minipage}{\textwidth}#1}{\end{minipage}\vspace{16pt}}

\newcommand{\trans}[1]{$\xrightarrow{\textnormal{#1}}$}
\newcommand{\tab}{\hspace*{2pt}}
\newcommand{\arrow}{\mathbin{\rightarrow}}
\newcommand{\bnfdef}{\mathrel{\Coloneqq}}
\newcommand{\bnfalt}{\mathrel{\mid}}
\newcommand{\note}[1]{{\tiny Note: #1}}
\newcommand{\vd}{\vdash}
\newcommand{\of}{\mathrel{:}}
\newcommand{\bind}[2]{#1\mathrel{.} #2}
\newcommand{\subst}[3]{[#1 \mathbin{/} #2]#3}

\newcommand{\oversetr}[2]{\overset{#2}{#1}}

\newcommand{\ace}{\Leftrightarrow} % Algorithmic Constructor Equivalence
\newcommand{\ape}{\leftrightarrow} % Algorithmic Path Equivalence

\newcommand{\Darrow}{\mathrel{\scalebox{1.2}[1]{$\Downarrow$}}}
\newcommand{\whn}{\Darrow}
\newcommand{\whr}{\mathrel{\leadsto}}
\newcommand{\Uarrow}{\mathrel{\scalebox{1.2}[1]{$\Uparrow$}}}
%\newcommand{\nk}{\mathrel{\Uarrow}}
\newcommand{\nk}{\mathrel{\uparrow}}
\newcommand{\lift}{\uparrow}
\newcommand{\sk}{\mathrel{\unlhd}} % TODO: triangle less equal

\newcommand{\synthesis}{\Rightarrow}
\newcommand{\checking}{\Leftarrow}

\newcommand{\comp}{\mathrel{\circ}}

\newcommand{\singleton}[1]{\mathop{S}(#1)}

%\newcommand{\test}[3][]{\ifthenelse{\isempty{#1}}{omitted}{given} #2 #3}
\newcommand{\inferr}[3][]{\inferrule*[Right=#1]{#3}{#2}}


\newcommand{\nats}{\mathcal{N}}
\newcommand{\reals}{\mathcal{R}}

\newcommand{\translatesto}{\mathrel{\leadsto}}

\DeclareMathOperator{\kind}{kind}
\DeclareMathOperator{\ok}{ok}
\DeclareMathOperator{\fv}{FV}
\DeclareMathOperator{\id}{id}
%\newcommand{\type}{\texttt{T}
\DeclareMathOperator{\type}{T}
%\DeclareMathOperator{\type}{Type}
\DeclareMathOperator{\refraw}{ref}
\DeclareMathOperator{\intt}{int}
\DeclareMathOperator{\stringt}{string}
\DeclareMathOperator{\unit}{unit}
\DeclareMathOperator{\as}{as}
\newcommand{\env}{\texttt{env}}
\newcommand{\exn}{\texttt{exn}}
\newcommand{\halt}{\texttt{halt}}
\newcommand{\letcc}[3]{\texttt{callcc}_{#1}\bind{#2}{#3}}
\newcommand{\cont}[1]{\texttt{cont}~#1} % TODO
\newcommand{\throw}[3]{\texttt{throw}_{#1} #2 \texttt{ to } #3}
\newcommand{\callcc}{\texttt{callcc}}
\newcommand{\pack}{\texttt{pack}}
\newcommand{\unpack}{\texttt{unpack}}
%\newcommand{\unpack}[4]{\texttt{unpack}[#1, #2] = #3~\texttt{in}~#4~\texttt{end}}
\newcommand{\sref}[1]{\refraw(#1)}
\newcommand{\letbind}[3]{\texttt{let}~#1 \mathrel{=} #2~\texttt{in}~#3~\texttt{end}}
\newcommand{\letbinds}[2]{\texttt{let}~#1~\texttt{in}~#2~\texttt{end}}
\newcommand{\ap}[2]{#1\mathop{ }#2}
\newcommand{\Ap}[2]{#1\mathop{\cdot}#2}

\newcommand{\pair}[1]{\langle #1 \rangle}
\newcommand{\prodn}[1]{\langle #1_1 \dots #1_n \rangle}
\newcommand{\Prodn}[1]{\times [ #1_1 \dots #1_n ]}

\newcommand{\alloc}[1]{\texttt{alloc}[#1]}
\newcommand{\fresh}[1]{\texttt{fresh}[#1]}
\newcommand{\ext}[1]{\texttt{ext}~#1}
\newcommand{\api}{\Pi^\texttt{ap}}
\newcommand{\alam}[2]{\lambda^\texttt{ap}\bind{#1}{#2}}
\newcommand{\sig}{\texttt{sig}}

%\newcommand{\lheadraw}[1]{
%\begin{tikzpicture}[#1]
%\coordinate (l) at (0, -1.1ex);
%\coordinate (h) at (0, 1.1ex);
%\draw (l) to[out=140, in=220] (h);
%\draw (l) to[out=90, in=270] (h);
%\end{tikzpicture}
%}
%\newcommand{\lhead}{\raisebox{-0.4ex}{\lheadraw{}}}
\newcommand{\lhead}{(\kern-0.2em{\mid}}
\newcommand{\rhead}{)\kern-0.48em{\mid} \kern+0.2em}
%\newcommand{\rhead}{\mid\kern-0.5em{)}}
\newcommand{\lsand}{\langle\kern-0.2em{\mid}}
%\newcommand{\rsand}{\mid\kern-0.5em{\rangle}}
\newcommand{\rsand}{\rangle\kern-0.48em{\mid} \kern+0.2em}
\newcommand{\satom}[1]{\lhead #1\, \rhead}
\newcommand{\datom}[1]{\lsand #1\, \rsand}

\newcommand{\Fst}[1]{\texttt{Fst}(#1)}
\newcommand{\Snd}[1]{\texttt{Snd}(#1)}
\newcommand{\Ext}[1]{\texttt{Ext}\ #1}

\newcommand{\seal}[2]{#1 :> #2}
\newcommand{\Pig}[2]{\Pi^\texttt{gen}\bind{#1}{#2}}
\newcommand{\lambdag}[2]{\lambda^\texttt{gen}\bind{#1}{#2}}
\newcommand{\Pia}[2]{\Pi^\texttt{app}\bind{#1}{#2}}
\newcommand{\lambdaa}[2]{\lambda^\texttt{app}\bind{#1}{#2}}

\newcommand{\letp}[3]{\texttt{letp}~#1 \mathrel{=} #2~\texttt{in}~#3~\texttt{end}}

\newcommand{\vdp}{\vd_P}
\newcommand{\vdi}{\vd_I}
\newcommand{\splitsto}{\leadsto}
\newcommand{\red}[1]{\textcolor{red}{#1}}
\newcommand{\sd}[2]{\red{[}#1\mathrel{\red{,}}#2\red{]}}
\newcommand{\axiom}[1]{\inferrule{#1}{\strut}}

\title{HOT Compilation Notes}
\author{Rahul Manne\\
{\tt rmanne}@andrew.cmu.edu}
\date{}

\begin{document}
\maketitle

\section*{Disclaimer/README}
These are only reference notes, and by no means fully capture what
is taught in class.\\

Notes for 170131 (on substitution) are extremely incoherent
so I did not include them by default.\\

There may be errors, feel free to report them to me.\\

\section{Compiler Structure}
SML \\
\tab\trans{elaborate} IL-Module \\
\tab\tab\trans{phase-splitting} IL-Direct \\
\tab\tab\tab\trans{cps conversion} IL-CPS \\
\tab\tab\tab\tab\trans{closure conversion} IL-Closure \\
\tab\tab\tab\tab\tab\trans{hoisting} IL-Hoist \\
\tab\tab\tab\tab\tab\tab\trans{allocation} IL-Alloc \\
\tab\tab\tab\tab\tab\tab\tab\trans{code-generation} C

\newpage
\section{Introduction to the $F\omega$ type system}

% This came first, fundementally useless for us
%\subsection{Grammar of the $F$ type system (``System F'')}
%\begin{align*}
%\tau & \bnfdef \alpha \bnfalt \tau \arrow \tau \bnfalt \forall \alpha. \tau \\
%e &\bnfdef x \bnfalt \lambda x \of \tau. e \bnfalt e\ e \bnfalt \Lambda \alpha. e \bnfalt e [ \tau ]
%\end{align*}
%Type $\tau$ and Term $e$.

\begin{grouped}{\subsection{Context for Judgements}}
\subsection{Grammar}
\begin{align*}
k & \bnfdef \type \bnfalt k \arrow k \\
c & \bnfdef \alpha \bnfalt c \arrow c \bnfalt \forall \bind{\alpha \of k}{c} \bnfalt c\ c \\
e & \bnfdef x \bnfalt \lambda \bind{x \of c}{e} \bnfalt e\ e
    \bnfalt \Lambda \bind{\alpha \of k}{e} \bnfalt e [ c ] \\
\end{align*}
Kind $k$, Type Constructor $c$, and Term $e$.\\
\note{$\type$ is often referred to with just ``T'' (by Crary), for simplicity.\\}
\end{grouped}

\begin{grouped}{\subsection{Context for Judgements}}
\begin{equation} \label{Context}
\Gamma \bnfdef \epsilon \bnfalt \Gamma, x \of \tau \bnfalt \Gamma, \alpha \of k
\end{equation}
\note{For simplicity, whenever a new $\alpha$ appears in the context, we
implicitly ensure that $\alpha$ is not already in $\Gamma$.\\}
\end{grouped}

\begin{grouped}{\subsection{$\Gamma \vd c \of k$}}
\begin{mathpar}
\inferr{\Gamma \vd \alpha \of k}
       {\Gamma(\alpha) = k}

\inferr{\Gamma \vd \tau_1 \arrow \tau_2 \of T}
       {\Gamma \vd \tau \of T \\ \Gamma \vd \tau_2 \of T}

\inferr{\Gamma \vd \forall\bind{\alpha \of k}{\tau}}
       {\Gamma, \alpha \of k \vd \tau \of T}

\inferr{\Gamma \vd \lambda\bind{\alpha \of k}{c} \of k \arrow k'}
       {\Gamma, \alpha \of k \vd c \of k'}

\inferr{\Gamma \vd c_1\ c_2 \of k'}
       {\Gamma \vd c_1 \of k \arrow k' \\ \Gamma \vd c_2 : k}
\end{mathpar}
\end{grouped}

\begin{grouped}{\subsection{$\Gamma \vd e \of \tau$}}
\begin{mathpar}
\inferr{\Gamma \vd x \of \tau}{\Gamma(x) = \tau}

\inferr{\Gamma \vd \lambda\bind{x \of \tau}{e} \of \tau \arrow \tau'}
       {\Gamma, x \of \tau \vd e \of \tau'}

\inferr{\Gamma \vd e_1\ e_2 \of \tau'}
       {\Gamma \vd e_1 \of \tau \arrow \tau' \\ \Gamma \vd e_2 : \tau}

\inferr{\Gamma \vd \Lambda\bind{\alpha \of k}{e} \of \forall\bind{\alpha \of k}{\tau}}
       {\Gamma, \alpha \of k \vd e \of \tau}

\inferr{\Gamma \vd e [c] \of \subst{c}{\alpha}{e}}
       {\Gamma \vd e \of \forall\bind{\alpha \of k}{e} \\ \Gamma \vd c \of k}

\inferr{\Gamma \vd e \of \tau'}
       {\Gamma \vd e \of \tau \\ \Gamma \vd \tau \equiv \tau' : T}
\end{mathpar}
\end{grouped}

\begin{grouped}{\subsection{$\Gamma \vd c \equiv c \of k$}}
Definitional Equivalence.\\

\begin{mathpar}
\inferr{\Gamma \vd c \equiv c \of k}{\Gamma \vd c \of k}

\inferr{\Gamma \vd c' \equiv c \of k}{\Gamma \vd c \equiv c' \of k}

\inferr{\Gamma \vd c_1 \equiv c_3 \of k}
       {\Gamma \vd c_1 \equiv c_2 \of k \\ \Gamma \vd c_2 \equiv c_2 \of k}
\end{mathpar}
The above are identity, reflexivity, and transitivity respectively.\\

The following are ``compatibility'' rules.\\
\begin{mathpar}
\inferr{\Gamma \vd c_1c_2 \equiv c_1'c_2' \of k}
       {\Gamma \vd c_1 \equiv c_1' \of k \\ \Gamma \vd c_2 \equiv c_2' \of k}

\inferr{\Gamma \vd \lambda\bind{\alpha \of k_1}{c} \equiv
                   \lambda\bind{\alpha \of k_1}{c'} \of k_1 \arrow k_2}
       {\Gamma, \alpha \of k_1 \vd c \equiv c' \of k_2}

\inferr{\Gamma \vd \tau_1 \arrow \tau_2 \equiv \tau_1' \arrow \tau_2' \of T}
       {\Gamma \vd \tau_1 \equiv \tau_1' \of k \\
        \Gamma \vd \tau_2 \equiv \tau_2' \of k}

\inferr{\Gamma \vd \forall\bind{\alpha \of k}{\tau} \equiv
                   \forall\bind{\alpha \of k}{\tau'} \of T}
       {\Gamma, \alpha \of k \vd \tau \equiv \tau' : T}
\end{mathpar}

congruence = compatible equivalence relation \\ %% TODO: WTF IS THIS
The following are the rules for beta equivalence and extensionality:\\
\begin{mathpar}
\inferr{\Gamma \vd (\lambda\bind{\alpha \of k}{c_1})\ c_2 \equiv
                  \subst{c_2}{\alpha}{c_1} \of k'}
       {\Gamma \vd c_2 \of k \\ \Gamma, \alpha \of k \vd c_1 \of k'}

\inferr{\Gamma \vd c \equiv c' \of k_1 \arrow k_2}
       {\Gamma, \alpha \of k_1 \vd c\ \alpha \equiv c'\ \alpha \of k_2 \\
        \Gamma \vd c \of k_1 \arrow k_2 \\ \Gamma \vd c' : k_1 \arrow k_2}
\end{mathpar}
\end{grouped}

\begin{grouped}{\subsection{Extending $F\omega$}}
\note{This helps in the understanding of sml's module system\\}

Grammar:
\begin{align*}
k & \bnfdef \ldots \bnfalt k \times k \\
c & \bnfdef \ldots \bnfalt \langle c, c \rangle \bnfalt \pi_1 c \bnfalt \pi_2 c
\end{align*}

New Judgements:
\begin{mathpar}
\inferr{\Gamma \vd \langle c_1, c_2 \rangle \of k_1 \times k_2}
       {\Gamma \vd c_1 \of k_2 \\ \Gamma \vd c_2 \of k_2}

\inferr{\Gamma \vd \pi_i\ c \of k_1}{\Gamma \vd c \of k_1 \times k_2}

\inferr{\Gamma \vd \langle c_1, c_2 \rangle \equiv
                   \langle c_1', c_2' \rangle \of k_1 \times k_2}
       {\Gamma \vd c_1 \equiv c_1' \of k_1 \\ \Gamma \vd c_2 \equiv c_2' \of k_2}

\inferr{\Gamma \vd \pi_i\ c \equiv \pi_i\ c' \of k_i}
       {\Gamma \vd c \equiv c' \of k_1 \times k_2}

\inferr{\Gamma \vd \pi_i\langle c_1, c_2 \rangle \equiv c_i \of k_i}
       {\Gamma \vd c_1 \of k_1 \\ \Gamma \vd c_2 \of k_2}

\inferr{\Gamma \vd c \equiv c' \of k_1 \times k_2}
       {\Gamma \vd \pi_1\ c \equiv \pi_1\ c' \of k_1 \\
        \Gamma \vd \pi_2\ c \equiv \pi_2\ c' \of k_2}
\end{mathpar}
\end{grouped}


\newpage
\section{Algorithmic Equivalence in the $F\omega$ Type System}
\begin{grouped}{\subsection{Normalize-and-Compare}}
\note{We don't use this.\\}

$\lambda\bind{\alpha \of k}{c_1}\ c_2 \xrightarrow{\beta} \subst{c_2}{\alpha}{c_1}$ \\
$\pi_i \langle c_1, c_2 \rangle \xrightarrow{\beta} c_i$ \\
+ some $\eta$ reduction rules \\

According to some equivalence theorem, they will have normal forms
and those normal forms will be equal if they are equivalent.
\end{grouped}

\begin{grouped}{\subsection{Grammar and Properties}}
Paths:\\
$p \bnfdef \alpha \bnfalt p\ c \bnfalt \pi_1\ p \bnfalt \pi_2\ p$\\

Weak-Head Normal Form:\\
$n \bnfdef p \bnfalt c_1 \arrow c_2 \bnfalt \forall\bind{\alpha \of k}{c}$.\\

Regularity:\\
If $\vd \Gamma \ok$ and $\Gamma \vd c_1 \equiv c_2 \of k$,
then $\Gamma \vd c_1 \of k$ and $\Gamma \vd c_2 \of k$.\\
If $\vd \Gamma \ok$ and $\Gamma \vd c \of k$, then $\Gamma \vd k \of \kind$.\\

Soundness:\\
If $\vd \Gamma \ok$ and $\Gamma \vd c_1, c_2 \of k$ and $\Gamma \vd c_1 \ace c_2 \of k$,
then $\vd c_1 \equiv c_2 \of k$.\\

Completeness:\\
If $\vd \Gamma \ok$ and $\Gamma \vd c_1 \equiv c_2 \of k$,
then $\Gamma \vd c_1 \ace c_2 \of k$.\\

\begin{mathpar}
\inferr{\vd \epsilon \ok}{\strut}

\inferr{\vd \Gamma, \alpha \of k \ok}{\vd \Gamma \ok \\ \Gamma \vd k \of \kind}

\inferr{\vd \Gamma, x \of \tau \ok}{\vd \Gamma \ok \\ \Gamma \vd \tau \of \type}
\end{mathpar}
\end{grouped}

\begin{grouped}{\subsection{Algorithmic Constructor Equivalence}}
Form: $\oversetr{\Gamma}{+} \vd \oversetr{c_1}{+} \ace \oversetr{c_2}{+} \of \oversetr{k}{+}$\\
\note{$\oversetr{x}{+}$ indicates that $x$ is an input.}

\begin{mathpar}
\inferr{\Gamma \vd c \ace c' \of k_1 \arrow k_2}
       {\Gamma, \alpha \of k_1 \vd c\ \alpha \ace c'\ \alpha \of k_2}

\inferr{\Gamma \vd c \ace c' \of k_1 \times k_2}
       {\Gamma \vd \pi_1\ c \ape \pi_1\ c' \of k_1 \\
        \Gamma \vd \pi_2\ c \ape \pi_2\ c' \of k_2}

\inferr{\Gamma \vd c_1 \ace c_2 \of \type}
       {c_1 \whn c_1' \\ c_2 \whn c_2' \\ \Gamma \vd c_1' \ape c_2' \of \type}
\end{mathpar}

\subsection{Algorithmic Path Equivalence}
Form: $\oversetr{\Gamma}{+} \vd \oversetr{c_1}{+} \ace \oversetr{c_2}{+} \of \oversetr{k}{-}$\\
\note{$\oversetr{x}{-}$ indicates that $x$ is an output.}

\begin{mathpar}
\inferr{\Gamma \vd \alpha \ape \alpha \of k}{\Gamma(\alpha) = k}

\inferr{\Gamma \vd p\ c \ape p'\ c' \of k_1}
       {\Gamma \vd p \ape p' \of k_1 \arrow k_2 \\ \Gamma \vd c \ace c' \of k_1}

\inferr{\Gamma \vd \pi_i\ p \ape \pi_i\ p' \of k_i}
       {\Gamma \vd p \ape p' \of k_1 \times k_2}

\inferr{\Gamma \vd c_1 \arrow c_2 \ape c_1' \arrow c_2' \of T}
       {\Gamma \vd c_1 \ace c_1' \of T \\ \Gamma \vd c_1 \ace c_2' \of T}

\inferr{\Gamma \vd \forall\bind{\alpha \of k}{c} \ape
                   \forall\bind{\alpha \of k}{c'} \of T}
       {\Gamma, \alpha \of k \vd c \ace c' \of T}
\end{mathpar}

\subsection{Weak-Head Normalization}
Form: $\oversetr{c}{+} \whn \oversetr{n}{-}$

\begin{mathpar}
\inferr{c \whn c''}{c \whr c' \\ c' \whn c''}

\inferr{c \whn c}{c \not\whr}
\end{mathpar}

\subsection{Weak-Head Reduction}
Form: $\oversetr{c}{+} \whr \oversetr{c'}{-}$

\begin{mathpar}
\inferr{(\lambda\bind{\alpha \of k}{c_1})\ c_2 \whr \subst{c_2}{\alpha}{c_1}}{\strut}

\inferr{\pi_i \langle c_1, c_2 \rangle \whr c_i}{\strut}

\inferr{c_1\ c_2 \whr c_1'\ c_2}{c_1 \whr c_1'}

\inferr{\pi_i c \whr \pi_i c'}{c \whr c'}
\end{mathpar}
\end{grouped}

\begin{grouped}{\subsection{Kind Synthesis and Checking}}
Form: $\oversetr{\Gamma}{+} \vd \oversetr{c}{+} \synthesis \oversetr{k}{-}$ and
$\oversetr{\Gamma}{+} \vd \oversetr{c}{+} \checking \oversetr{k}{+}$

\begin{mathpar}
\inferr{\Gamma \vd \alpha \synthesis k}{\Gamma(\alpha) = k}

\inferr{\Gamma \vd \lambda\bind{\alpha \of k}{c} \synthesis k \arrow k'}
       {\Gamma, \alpha \of k \vd c \synthesis k'}

\inferr{\Gamma \vd c_1\ c_2 \synthesis k'}
       {\Gamma \vd c_1 \synthesis k \arrow k' \\ \Gamma \vd c_2 \checking k}

\inferr{\Gamma \vd \langle c_1, c_2 \rangle \synthesis k_1 \times k_2}
       {\Gamma \vd c_1 \synthesis k_1 \\ \Gamma \vd c_2 \synthesis k_2}

\inferr{\Gamma \vd \pi_i\ c \synthesis k_1}{\Gamma \vd c \synthesis k_1 \times k_2}

\inferr{\Gamma \vd c_1 \arrow c_2 \synthesis T}
       {\Gamma \vd c_1 \checking T \\ \Gamma \vd c_2 \checking T}

\inferr{\Gamma \vd \forall\bind{\alpha \of k}{c} \synthesis T}
       {\Gamma, \alpha \of k \vd c \checking T}

\inferr{\Gamma \vd c \checking k}{\Gamma \vd c \synthesis k}
\end{mathpar}

\subsection{Type Checking and Synthesis}
Form: $\oversetr{\Gamma}{+} \vd \oversetr{e}{+} \synthesis \oversetr{c}{-}$ and
$\oversetr{\Gamma}{+} \vd \oversetr{e}{+} \checking \oversetr{c}{+}$

\begin{mathpar}
\inferr{\Gamma \vd x \synthesis \tau}{\Gamma(x) = \tau}

\inferr{\Gamma \vd \lambda\bind{x \of \tau}{e} \synthesis \tau \arrow \tau'}
       {\Gamma \vd \tau \checking T \\ \Gamma, x \of \tau \vd e \synthesis \tau'}

\inferr{\Gamma \vd e_1\ e_2 \synthesis \tau'}
       {\Gamma \vd e_1 \synthesis \tau_1 \\
        \tau_1 \whn \tau \arrow \tau' \\
        \Gamma \vd e_2 \checking \tau}

\inferr{\Gamma \vd \Lambda\bind{\alpha \of}{e} \synthesis
                   \forall\bind{\alpha \of k}{\tau}}
       {\Gamma, \alpha \of k \vd e \synthesis \tau}

\inferr{\Gamma \vd e [c] \synthesis \subst{c}{\alpha}{\tau'}}
       {\Gamma \vd e \synthesis \tau \\
        \tau \whn \forall\bind{\alpha \of k}{\tau'} \\
        \Gamma \vd c \checking k}

\inferr{\Gamma \vd e \checking \tau}
       {\Gamma \vd e \synthesis \tau' \\ \Gamma \vd \tau \ace \tau' : T}
\end{mathpar}
\end{grouped}


%\newpage
% TODO: this section is super incoherent
\section{Substitution}

\begin{grouped}{\subsection{Explicit Variables}}
\begin{lstlisting}
datatype con = Var of variable | Lam of variable * kind * con
datatype term = Var of variable | Lam of variable * con * term
\end{lstlisting}

$c \bnfdef \alpha \bnfalt \lambda\bind{\alpha \of \tau}{c}$\\
We always alpha variate because we can (saves the cost of a search):\\
$\subst{M}{x}{(\lambda\bind{y}{N})} = \lambda\bind{y'}{\subst{M}{x}{\subst{y'}{y}{N}}}$ (find $y' \not\in \fv(M)$)\\
\end{grouped}

\begin{grouped}{\subsection{de Bruijn}}
\begin{lstlisting}
datatype con = Var of int | Lam of kind * con
\end{lstlisting}

Rules for substitution:
\begin{align*}
\subst{M}{i}{j} &=
  \begin{cases} 
    M & i = j \\
    j - 1 & i < j \\
    j & i > j
  \end{cases}\\
\subst{M}{i}{N\ P} &= \subst{M}{i}{N}\ \subst{M}{i}{P} \\
\subst{M}{i}{\lambda\bind{}{N}} &= \lambda\bind{}{\subst{\lift_{\ge0}M}{i+1}{N}}
\end{align*}

Rules for $\lift$:
\begin{align*}
\lift_{\ge i} j = \begin{cases} j + 1 & j \ge i \\ j & j < i \end{cases} \\
\lift_{\ge i} M\ N = \uparrow_{\ge i} M\ \uparrow_{\ge i} N \\
\lift_{\ge i} \lambda\bind{}{N} = \lambda\bind{}{\lift_{\ge i + 1} N}
\end{align*}
\end{grouped}

\begin{grouped}{\subsection{Explicit Substitution}}
Grammar:\\
$\sigma \bnfdef \bind{M}{\sigma} \bnfalt \lift^i$\\

\note{$\id \overset{\mathop{def}}{=} \lift^0$\\}

Syntax Definition:\\
\begin{flalign*}
0 [\bind{M}{\sigma}] &= M &\\
i + 1 [\bind{M}{\sigma}] &= i [\sigma] &\\
n [\lift^i] &= n + i &\\
(MN) [\sigma] &= M[\sigma] N[\sigma] &\\
(\lambda\bind{A}{M}) [\sigma] &= \lambda A[\sigma]. M[0.(\sigma \comp \lift^1)] &\\
M[\sigma \comp \sigma'] &= M[\sigma][\sigma'] &\\
(M.\sigma) \comp \sigma' &= M[\sigma'].(\sigma \comp \sigma') &\\
\lift^0 \comp \sigma &= \sigma &\\
\lift^{i + 1} \comp \sigma &= \lift^i \comp \sigma &\\ % TODO
\lift^i \comp \lift^j &= \lift^{i + j}
\end{flalign*}

Syntax Exampes:\\
$[M/0, N/1]$ % TODO: wtf
\begin{flalign*}
0 [\bind{M}{\bind{N}{\id}}] &= M &\\
1 [\bind{M}{\bind{N}{\id}}] &= N &\\
2 [\bind{M}{\bind{N}{\id}}] &= 0 &\\
M &= M[0.\uparrow^1] &\\
  &= M[0.1.2....n.\uparrow^{n + 1}] &\\
\lift_{\ge 0}M &= M[\uparrow^1] &\\
\lift_{\ge 0}\lift_{\ge 0}M &= M[\uparrow^2] &\\
\lift_{\ge 1}M &= M[0.\uparrow^2]
\end{flalign*}
The last one here is a bit tricky. If you see $0$, we want to leave it as $0$,
and if we see $1$ or more, it is lowered to $0$, so to make up for that
change, shift up by $2$.\\
\end{grouped}

\begin{grouped}{Substitution Code (provided)\\}
\begin{lstlisting}
subst X Gen : int -> con list -> int -> X -> X
subst X Gen i [$c_1 \ldots c_k$] l M = M[$0. ... i - 1. c_1[\lift^i]. ... c_k[\lift^i]. \lift^{i+l}$]
  (* leave first i vars alone ($0 \ldots i-1$)
   * subst $c_i$ but need to shift up by i when passing into $c_i$ *)

substX c M = substXGen 0 [c] 0

liftX l M = substXGen 0 [] l (* whenever you move under a binder *)
\end{lstlisting}
\end{grouped}

% TODO: huge disconnect, FORMAT-------------------------------------------------

$\Gamma \bnfdef \epsilon \bnfalt \Gamma, A$ \\

% TODO: huge disconnect

Typing Judgement for a substitution:\\
if $\Gamma \vd \sigma \of \Gamma'$ and $\Gamma' \vd M \of B$
then $\Gamma \vd M[\sigma] \of B[\sigma]$\\
``Any term in $\Gamma'$, after we apply $\sigma$, we should get $\Gamma$''\\

\begin{mathpar}
\inferr{\Gamma, A_1 \ldots A_i \vd \lift^i \of \Gamma}{\strut}

\inferr{\Gamma \vd \bind{M}{\sigma} \of \Gamma', A}
       {\Gamma' \vd \sigma \of \Gamma' \\ \Gamma \vd M \of A[\sigma]}
\end{mathpar}

Example:\\
\newcommand{\str}{\texttt{string}}
\[
\infer{\Gamma \vd [\bind{\texttt{"hello world"}}
                        {\bind{\str}{\id}}] \of \Gamma, \type, 0}
      {\infer{\Gamma \vd \bind{\str}{\id} \of \Gamma, \type} % TODO
             {\Gamma \vd \id \of \Gamma & \Gamma \vd \str \of \type[\id]}
      &\infer{\Gamma \vd \texttt{"hello world"} \of 0[\bind{\str}{\id}]}{\strut}}
\]
\note{$0[\bind{\str}{\id}]$ is just $\str$}
% TODO: FORMAT ^^^^^^^^^^^^^^^^^^^^^^^^^^^^^^^^^^^^^^^^^^^^^^^^^^^^^^^^^^^^^^^^^


\newpage
\section{Singleton Kinds}

\begin{lstlisting}
sig
  type t
  type 'a u
  type ('a,'b) v
  type w = int
  type w' = w
  .
  .
  .
end
\end{lstlisting}

To represent this in type our type system,
$t \of \type$\\
$u \of \type \arrow \type$\\
$v \of \type \arrow \type \arrow \type$\\
(or $v \of \type \times \type \arrow \type$)\\
$w \of \singleton{\int}$\\
$w' \of \singleton{w}$\\

\subsection{Grammar and Judgements (Attempt 1)}
Grammar:
\begin{flalign*}
k &\bnfdef \type \bnfalt k \arrow k \bnfalt k \times k \bnfalt \singleton{c} &\\
c &\bnfdef \ldots
\end{flalign*}

Judgements:
\begin{mathpar}
\inferr{\Gamma \vd c \of \singleton{c}}{\strut}

\inferr{\Gamma \vd c \equiv c' \of \type}{\Gamma \vd c \of \singleton{c}}

\inferr{\Gamma \vd \singleton{c} \of \kind}{\Gamma \vd c \of \type}
\end{mathpar}

Signature for {\tt list}.
\begin{lstlisting}
sig
  .
  .
  .
  type 'a s = 'a list
  type 'a t
end
\end{lstlisting}
So we have $t \of \type \arrow \type$.\\
How do we represent 'a s? Is $s \of \type \arrow \singleton{\alpha}$? But
then what's $\alpha$.

\begin{grouped}{\subsection{Dependent Kinds (Grammar)}}
\begin{flalign*}
k &\bnfdef \type \bnfalt \Pi\bind{\alpha \of k}{k}
   \bnfalt \Sigma\bind{\alpha \of k}{k} \bnfalt \singleton{c} &\\
c &\bnfdef \ldots
\end{flalign*}

% TODO
\note{$\Pi$ is also known as ``dependent product''\\
$\Sigma$ is also known as ``dependent sum'' (but also sometimes as
``dependent product'').\\
To avoid confusion, we name $\Pi$ ``dependent function (spaces)''.\\}

Now, we have $s \of \Pi\bind{\alpha \of \type}{\singleton{\mathop{list} \alpha}}$.\\

New judgements we need to be able to make: \\
$\Gamma \vd k \of \kind$ \\
$\Gamma \vd k \equiv k' \of kind$ \\
$\Gamma \vd k \le k'$ \\
$\Gamma \vd c \of k$ \\
$\Gamma \vd c \equiv c' \of k$ \\
$\Gamma \vd e \of \tau$ \\

\note{$\singleton{\int} \le \type$}
\end{grouped}

\begin{grouped}{\subsection{$\Gamma \vd k \of \kind$}}
\begin{mathpar}
\inferr{\Gamma \vd \type \of \kind}{\strut}

\inferr{\Gamma \vd \singleton{c} \of \kind}{\Gamma \vd c \of \type}

\inferr{\Gamma \vd \Pi\bind{\alpha \of k_1}{k_2} \of \kind}
       {\Gamma \vd k_1 \of \kind \\ \Gamma, \alpha \of k_1 \vd k_2 \of \kind}

\inferr{\Gamma \vd \Sigma\bind{\alpha \of k_1}{k_2} \of \kind}
       {\Gamma \vd k_1 \of \kind \\ \Gamma, \alpha \of k_1 \vd k_2 \of \kind}
\end{mathpar}
\end{grouped}

\begin{grouped}{\subsection{$\Gamma \vd k \equiv k' \of \kind$}}
\begin{mathpar}
\inferr{\Gamma \vd k \equiv k \of \kind}{\Gamma \vd k \of \kind}

\inferr{\Gamma \vd k_2 \equiv k_1 \of \kind}{\Gamma \vd k_1 \equiv k_2 \of \kind}

\inferr{\Gamma \vd k_1 \equiv k_2 \of \kind}
       {\Gamma \vd k_1 \equiv k_2 \of \kind \\ \Gamma \vd k_2 \equiv k_2 \of \kind}

% compatibility rules
\inferr{\Gamma \vd \singleton{c} \equiv \singleton{c'} \of \kind}
       {\Gamma \vd c \equiv c' \of \type}

\inferr{\Gamma \vd \Pi\bind{\alpha \of k_1}{k_2} \equiv
                   \Pi\bind{\alpha \of k_1'}{k_2'} \of \kind}
       {\Gamma \vd k_1 \equiv k_1' \of \kind \\
        \Gamma, \alpha \of k_1 \vd k_2 \equiv k_2' \of \kind}

\inferr{\Gamma \vd \Sigma\bind{\alpha \of k_1}{k_2} \equiv
                   \Sigma\bind{\alpha \of k_1'}{k_2'} \of \kind}
       {\Gamma \vd k_1 \equiv k_1' \of \kind \\
        \Gamma, \alpha \of k_1 \vd k_2 \equiv k_2' \of \kind}
\end{mathpar}
\note{for the latter two, keep
$\Pi\bind{\alpha \of k_1}{k_2} \oversetr{\equiv}{?} \Pi\bind{\alpha' \of k_1'}{k_2'}$
in mind}
\end{grouped}

\begin{grouped}{\subsection{$\Gamma \vd \alpha \of k$}}
\begin{mathpar}
\inferr{\Gamma \vd \alpha \of k}{\Gamma(\alpha) = k}

\inferr{\Gamma \vd c_1 \arrow c_2 \of \type}
       {\Gamma \vd c_1 \of \type \\ \Gamma \vd c_2 \of \type}

\inferr{\Gamma \vd \forall\bind{\alpha : k}{c} \of \type}
       {\Gamma \vd k \of kid \\ \Gamma, \alpha \of k \vd c \of \type}

\inferr{\Gamma \vd \lambda\bind{\alpha \of k_1}{c} \of \Pi\bind{\alpha \of k_1}{k_2}}
       {\Gamma \vd k_1 \of \kind \\ \Gamma, \alpha \of k_1 \vd c \of k_2}

\inferr{\Gamma \vd c_1\ c_2 \of \subst{c_1}{\alpha}{k'}}
       {\Gamma \vd c_1 \of \Pi\bind{\alpha \of k}{k'} \\ \Gamma \vd c_2 \of k}

\inferr{\Gamma \vd \langle c_1, c_2 \rangle \of \Sigma\bind{\alpha \of k_2}{k_2}}
       {\Gamma \vd c_1 \of k_2 \\
        \Gamma \vd c_2 \of \subst{c_1}{\alpha}{k_2} \\
        \Gamma, \alpha \of k_1 \vd k_2 \of \kind}

\inferr{\Gamma \vd \pi_1 c \of k_1}
       {\Gamma \vd c \of \Sigma\bind{\alpha \of k_1}{k_2}}

\inferr{\Gamma \vd \pi_2 c \of \subst{\pi_1 c}{\alpha}{k_2}}
       {\Gamma \vd c \of \Sigma\bind{\alpha \of k_1}{k_2}}

% TODO
\inferr{\Gamma \vd c \of k'}{\Gamma \vd c \of k \\ \Gamma \vd k \le k'}
\end{mathpar}
\end{grouped}

\begin{grouped}{Additional Judgements}
If $\vd \Gamma \ok$ and $\Gamma \vd c \of k$, then $\Gamma \vd k \of \kind$.\\
If $\vd \Gamma \ok$ and $\Gamma \vd k_1 \equiv k_2$, then $\Gamma \vd k_1, k_2 \kind$.\\
If $\vd \Gamma \ok$ and $\Gamma \vd k_1 \le k_2$, then $\Gamma \vd k_1, k_2 \kind$.\\

\begin{mathpar}
\inferr{\Gamma \vd c \of \singleton{c}}{\Gamma \vd c \of \type}
\end{mathpar}

Sub-kinding:
\begin{mathpar}
\inferr{\Gamma \vd k \le k'}{\Gamma \vd k \equiv k' \of \kind}

\inferr{\Gamma \vd k_1 \le k_3}{\Gamma \vd k_1 \le k_2 \\ \Gamma \vd k_2 \le k_3}

% substantiative
\inferr{\Gamma \vd \singleton{c} \le \type}{\Gamma \vd c \of \type}

% compatibility rules
\inferr{\Gamma \vd \singleton{c} \le \singleton{c'}}
       {\Gamma \vd c \equiv c' \of \type}

\inferr{\Gamma \vd \Pi\bind{\alpha \of k_1}{k_2} \le
                   \Pi\bind{\alpha \of k_1'}{k_2'}}
       {\Gamma \vd k_1' \le k_1 \\
        \Gamma, \alpha \of k_1' \vd k_2 \le k_2' \\
        \Gamma, \alpha \of k_1 \vd k_2 \of \kind}

\inferr{\Gamma \vd \Sigma\bind{\alpha \of k_1}{k_2} \le
                   \Sigma\bind{\alpha \of k_1'}{k_2'}}
       {\Gamma \vd k_1 \le k_1' \\
        \Gamma, \alpha \of k_1 \vd k_2 \le k_2' \\
        \Gamma, \alpha \of k_1' \vd k_2' \of \kind}
\end{mathpar}
\note{Something about contravariance for 1st condition.
$\Pi$ contravariant the same way arrow is contravariant.
Covariance for 2nd condition. This is for $\Pi$.}

\end{grouped}

\begin{grouped}{\subsection{$\Gamma \vd c \equiv c \of k$}}
\begin{mathpar}
\inferr{\Gamma \vd c \equiv c \of k}{\Gamma \vd c \of k}

\inferr{\Gamma \vd c_2 \equiv c_1 \of k}{\Gamma \vd c_1 \equiv c_2 \of k}

\inferr{\Gamma \vd c_1 \equiv c_3 \of k}
       {\Gamma \vd c_1 \equiv c_2 \of k \\ \Gamma \vd c_2 \equiv c_3 \of k}

% SUBSTANTIATIVE
\inferr{\Gamma \vd (\lambda\bind{\alpha \of k}{c_1})\ c_2 \equiv \subst{c_2}{\alpha}{c_1} \of \subst{c_2}{\alpha}{k'}}{\Gamma \vd c_2 \of k \\ \Gamma, \alpha \of k \vd c_1 \of k'}

\inferr{\Gamma \vd \pi_i \langle c_1, c_2 \rangle \equiv c_i \of k_i}{\Gamma \vd c_1 \of k_1 \\ \Gamma \vd c_2 \of k_2}

\inferr{\Gamma \vd c \equiv c' \of \singleton{c'}}
       {\Gamma \vd c \of \singleton{c'}}

\inferr[$\star$]
       {\Gamma \vd c_1 \equiv c_2 \of k'}
       {\Gamma \vd c_1 \equiv c_2 \of k \\ \Gamma \vd k \le k'}

\inferr[$\star$]
       {\Gamma \vd c \equiv c' \of \singleton{c}}
       {\Gamma \vd c \equiv c' \of \type}

\inferr{\Gamma \vd \lambda\bind{\alpha \of k_1}{c} \equiv
                   \lambda\bind{\alpha \of k_1'}{c'} \of \Pi\bind{\alpha \of k_1}{k_2}}
       {\Gamma \vd k_1 \equiv k_1' \of \kind \\
        \Gamma, \alpha \of k_1 \vd c \equiv c' \of k_2}

\inferr{\Gamma \vd c_1\ c_2 \equiv c_1'\ c_2' \of \subst{c_2}{\alpha}{k'}}
       {\Gamma \vd c_1 \equiv c_1' \of \Pi\bind{\alpha \of k}{k'} \\
        \Gamma \vd c_2 \equiv c_2' \of k}
\end{mathpar}
\end{grouped}

%$\Gamma \vd k \le k'$ \\
%$\Gamma \vd c \equiv c' \of k$ \\
%$\Gamma \vd e \of \tau$ \\



\newpage
\section{Sub-Typing}

$\tau \le \tau'$ means you can use a $\tau$ whrever a $\tau'$ is expected.\\

\begin{mathpar}
\inferr%[subsumption]
       {\Gamma \vd e \of \tau'}{\Gamma \vd e \of \tau \\ \tau \le \tau'}

\inferr%[$*^1$]
       {\Gamma \vd \tau \times \tau_2 \le \tau_1' \times \tau_2'}
       {\tau_1 \le \tau_1' \\ \tau_2 \le \tau_2'}

\inferr%[$*^2$]
       {\tau_1 \arrow \tau_2 \le \tau_1' \arrow \tau_2'}
       {\tau_1' \le \tau_1 \\ \tau_2 \le \tau_2'}
\end{mathpar}
Note 1: This is a case of covariance on both sides.\\
Note 2: This is contravariant on the left and covariant on the right.\\

$\nats \le \reals$.\\
Assume we have $f \of \nats \arrow \nats$.\\
$f \not\of \reals \arrow \reals$.\\

Assume we have $f \of \reals \arrow \reals$.\\
Contravariance: $f \of \nats \arrow \reals$.\\

\begin{mathpar}
\inferr{\sref{\tau} \le \sref{\tau'}}{\tau \equiv \tau' \of \type}
\end{mathpar}
$\sref{\tau}$ is neither covariant nor contravariant. Called ``invariant''.
(Poorly named, but it's what's used in literature.)\\

\begin{mathpar}
\inferr{\Gamma \vd \Pi\bind{\alpha \of k_1}{k_2} \le
                   \Pi\bind{\alpha \of k_1'}{k_2'}}
       {\Gamma \vd k_1' \le k_1 \\ \Gamma, \alpha \of k_1' \vd k_2 \le k_2' \\
        \Gamma, \alpha \of k_1 \vd k_2 \of kind}

\inferr{\Gamma \vd \Sigma\bind{\alpha \of k_1}{k_2} \le
                   \Sigma\bind{\alpha \of k_1'}{k_2'}}
       {\Gamma \vd k_1' \le k_1 \\ \Gamma, \alpha \of k_1' \vd k_2 \le k_2' \\
        \Gamma, \alpha \of k_1 \vd k_2 \of \kind}

\inferr%[$\times^1$]
       {\Gamma \vd c \equiv c' \of \singleton{c'}}
       {\Gamma \vd c \of \singleton{c'}}

\inferr{\Gamma \vd c \equiv c' \of \type}{\Gamma \vd c \of \singleton{c'}}

\inferr{\Gamma \vd c \of c' \of \singleton{c}}{\Gamma \vd c \equiv c' \of \type}
\end{mathpar}
Note 1: Sound, but not what we want.\\

More compatiblity rules.\\
\begin{mathpar}
\inferr{\Gamma \vd \langle c_1, c_2 \rangle \equiv \langle c_1', c_2' \rangle \of \Sigma\bind{\alpha \of k_1}{k_2}}
       {\Gamma \vd c_1 \equiv c_1' \of k_1 \\
        \Gamma \vd c_2 \equiv c_2' \of \subst{c_1}{\alpha}{k_2} \\
        \Gamma, \alpha \of k_1 \vd k_2 \of \kind}

\inferr{\Gamma \vd \pi_1 c \equiv \pi_1 c' \of k_1}
       {\Gamma \vd c \equiv c' \of \Sigma\bind{\alpha \of k_1}{k_2}}

\inferr{\Gamma \vd \pi_2 c \equiv \pi_2 c' \of \subst{\pi_1 c}{\alpha}{k_2}}
       {\Gamma \vd c \equiv c' \of \Sigma\bind{\alpha \of k_1}{k_2}}

\inferr{\Gamma \vd c_1 \arrow c_2 \equiv c_1' \arrow c_2' \of \type}
       {\Gamma \vd c_1 \equiv c_1' \of \type \\
        \Gamma \vd c_2 \equiv c_2' \of \type}

\inferr{\Gamma \vd \forall\bind{\alpha \of k}{c} \equiv
                   \forall\bind{\alpha \of k'}{c'} \of \type}
       {\Gamma \vd k \equiv k' \of \kind \\ \Gamma, \alpha \of k \vd c \equiv c' \of \type}
\end{mathpar}

Rules for extentionality.
\begin{mathpar}
\inferr{\Gamma \vd c \equiv c' \of \Pi\bind{alpha \of k_1}{k_2}}
       {\Gamma, \alpha \of k_1 \vd c\ \alpha \equiv c'\ \alpha \of k_2 \\
        \Gamma \vd c \of \Pi\bind{alpha \of k_1}{k_2'} \\
        \Gamma \vd c' \of \Pi\bind{alpha \of k_1}{k_2''}}

\inferr%[$*^1$]
       {\Gamma \vd c \equiv c' \of \Pi\bind{\alpha \of k_1}{k_2}}
       {\Gamma, \alpha \of k_1 \vd c\ \alpha \equiv c'\ \alpha \of k_2 \\
        \Gamma \vd c \equiv c' \of \Pi\bind{alpha \of k_1}{k_2'}}

\inferr{\Gamma \vd c \equiv c' \of \Sigma\bind{alpha \of k_1}{k_2}}
       {\Gamma \vd \pi_1 c \equiv \pi_1 c' \of k_1 \\
        \Gamma \vd \pi_2 c \equiv \pi_2 c' \of \subst{\pi_1 c}{\alpha}{k_2} \\
        \Gamma, \alpha \of k_1 \vd k_2 \of \kind}
\end{mathpar}
Note 1: We only need this for proofs (regularity). We can safely ignore this.\\

% TODO: motivation?
We have no way of dealing with $\singleton{c \of k}$. So instead of redefining
everything, treat it as a macro following the following rules:\\
$\singleton{c \of \type} = \singleton{c}$ \\
$\singleton{c \of \Pi\bind{\alpha \of k_1}{k_2}}
  = \Pi\bind{\alpha \of k_1}{\singleton{c\ \alpha \of k_2}}$ \\
$\singleton{c \of \singleton{c'}} = \singleton{c}$
(note here, $c \equiv c'$, so we can use either, but it's easier for us to use $c$) \\
$\singleton{c \of \Pi\bind{\alpha \of k_1}{k_2}}
  = \Sigma\bind{\alpha \of \singleton{\pi_1 c \of k_1}}{\singleton{\pi_2 c \of k_2}}$ \\
OR $\singleton{\pi_1 c \of k_1} \times \singleton{\pi_2 c \of \subst{\pi_1 c}{\alpha}{k_2}}$ \\
We use the 2nd because it's nicer when not working without theory. The first is
more theoretic, the second is syntactic.\\


% TODO
\begin{enumerate}
\item If $\Gamma \vd c \of k$, then $\Gamma \vd c \of \singleton{c \of k}$
\item If $\Gamma \vd c \of \singleton{c' \of k}$, then $\Gamma \vd c \equiv c' \of k$
\end{enumerate}

But the first doesn't hold. So let's make it hold. Add ``declarative'' rules:
\begin{mathpar}
\inferr{\Gamma \vd c \of \Pi\bind{\alpha \of k_1}{k_2}}
       {\Gamma \vd k_1 \of \kind \\
        \Gamma, \alpha \of k_1 \vd c\ \alpha \of k_2}

\inferr{\Gamma \vd c \of \Sigma\bind{\alpha \of k_1}{k_2}}
       {\Gamma \vd \pi_1 c \of k_2 \\
        \Gamma \vd \pi_1 c \of \subst{\pi_1 c}{\alpha}{k_2} \\
        \Gamma, \alpha \of k_1 \vd k_2 \of \kind}
\end{mathpar}

Notes on definitional equivalence:\\
$\alpha \of \type \vd \alpha \not\equiv \intt \of \type$ \\
$\alpha \of \singleton\intt \vd \alpha \equiv \intt \of \type$ \\
$\vd \lambda\bind{\alpha \of \type}{\alpha} \not\equiv \lambda\bind{\alpha \of \type}{\intt} \of \type \arrow \type$ \\
$\vd \lambda\bind{\alpha \of \type}{\alpha} \not\equiv \lambda\bind{\alpha \of \type}{\intt} \of \singleton\intt \arrow \type$ \\
$\beta \of (\type \arrow \type) \arrow \type \vd \beta(\lambda\bind{\alpha \of \type}{\alpha} \not\equiv \beta(\lambda\bind{\alpha \of \type}{\intt} \of \type$ \\
$\beta \of (\singleton\intt \arrow \type) \arrow \type \vd \beta(\lambda\bind{\alpha \of \type}{\alpha} \equiv \beta(\lambda\bind{\alpha \of \type}{\intt} \of \type$ \\
$\type \arrow \type \le \singleton\intt \arrow \type$ \\


\subsection{Algorithm for Equivalence Checking}
\begin{mathpar}
\inferr{\Gamma \vd c \ace c' \of \Pi\bind{\alpha \of k_1}{k_2}}
       {\Gamma, \alpha \of k_1 \vd c\ \alpha \ace c'\ \alpha \of k_2}

\inferr{\Gamma \vd c \ace c' \of \Sigma\bind{\alpha \of k_1}{k_2}}
       {\Gamma \vd \pi_1 c \ace \pi_2 c' \of k_1
        \Gamma \vd \pi_2 c \ace \pi_2 c' \of \subst{\pi_1 c}{\alpha}{k_2}}

\inferr{\Gamma \vd c_1 \ace c_2 \of \type}
       {\Gamma \vd c_1 \whn c_1' \\ \Gamma \vd c_2 \whn c_2' \\
        \Gamma \vd c_1' \ape c_2' \of \type}

\inferr{\Gamma \vd c \whn c''}{\Gamma \vd c \whr c' \\ \Gamma \vd c' \whn c''}

\inferr{\Gamma \vd c \whn c}{\Gamma \vd c \not\whr}

\inferr{\Gamma \vd (\lambda\bind{\alpha \of k}{c_1})\ c_2 \whr \subst{c_2}{\alpha}{c_1}}{\strut}

\inferr{\Gamma \vd c_1\ c_2 \whr c_1'\ c_2}{\Gamma \vd c_1 \whr c_1'}

\inferr{\Gamma \vd \pi_i \langle c_1, c_2 \rangle \whr c_i}{\strut}

\inferr{\Gamma \vd pi_i c \whr pi_i c'}{\Gamma \vd c \whr c'}

\inferr{\Gamma \vd p}{\Gamma \vd p \nk \singleton{c}}

\inferr{\Gamma \vd \alpha \nk k}{\Gamma(\alpha) = k}

\inferr{\Gamma \vd p\ c \nk \subst{c}{\alpha}{k_2}}
       {\Gamma \vd p \nk \Pi\bind{\alpha \of k_1}{k_2}}

\inferr{\Gamma \vd \pi_1 p \nk k_1}{\Gamma \vd p \nk \Sigma\bind{\alpha \of k_1}{k_2}}

\inferr{\Gamma \vd \pi_2 p \nk \subst{\pi_1 p}{\alpha}{k_2}}
       {\Gamma \vd p \nk \Sigma\bind{\alpha \of k_1}{k_2}}
\end{mathpar}

% TODO nk "natural kind" is uparrow
% \Gamma+ \vd p+ \nk k-




\newpage
% TODO: continuation from previous time

\begin{mathpar}
\inferr{\Gamma \vd p \whr c}{\Gamma \vd p \nk \singleton{c}}
\end{mathpar}

Example:\\
\begin{mathpar}
\inferr
  {\vd \lambda\bind{\alpha \of \type}{\alpha} \ace \lambda\bind{\alpha \of \type}{\intt} \of \singleton{\intt} \arrow \type}
  {\inferr
    {\alpha \of \singleton{\intt} \vd
      (\lambda\bind{\alpha \of \type}{\alpha}) \alpha \ace
      (\lambda\bind{\alpha \of \type}{\intt}) \alpha \ace}
    {\inferr
      {\alpha \of \singleton{\intt} \vd
        (\lambda\bind{\alpha \of \type}{\alpha}) \alpha \whn
        }
      {\ldots \vd (\lambda{\alpha \of \type}{\alpha}{\alpha} \whr \alpha \\
       \inferr
        {\ldots \vd \alpha \whn \intt}
        {\inferr
          {\alpha \of \singleton{\intt} \vd \alpha \whr \intt}
          {\alpha \of \singleton{\intt} \vd \alpha \nk \singleton{\intt}} \\
         \inferr{\ldots \vd \intt \whn \intt}{\ldots \vd \intt \not\whr}
        }
      }
     \inferr{\alpha \of \singleton{\intt} \vd (\lambda{\alpha \of \type}{\intt}) \alpha \whn \intt}{\strut} \\
     \inferr{\alpha \of \singleton{\intt} \vd \intt \ape \intt \of \type}{\strut}
    }
  }
\end{mathpar}

One final rule: \\
\begin{mathpar}
\inferr{\Gamma \vd c_1 \ace c_2 \of \singleton{c}}{\strut}
\end{mathpar}
The precondition is that both $c_1$ and $c_2$ belong to $\singleton{c}$, meaning
they are equivalent to $c$ and by transitivity, equivalent to each other.\\

Some rules that we will never use:\\
\begin{mathpar}
\inferr{\Gamma \vd c_1 \arrow c_2 \nk \type}{\strut}

\inferr{\Gamma \vd \forall\bind{\alpha \of k}{c} \nk \type}{\strut}
\end{mathpar}

Structural rules:
\begin{mathpar}
\inferr{\Gamma \vd \alpha \ape \alpha \of k}{\Gamma(\alpha) = k}

\inferr{\Gamma \vd p\ c \ape p'\ c' \of \subst{c}{\alpha}{k_2}}
       {\Gamma \vd p \ape p' \of \Pi\bind{\alpha \of k_1}{k_2} \\
        \Gamma \vd c \ace c' \of k_1}

\inferr{\Gamma \vd \pi_1 p \ape \pi_1 p' \of k_1}
       {\Gamma \vd p \ape p' \of \Sigma\bind{\alpha \of k_1}{k_2}}

\inferr{\Gamma \vd \pi_1 p \ape \pi_1 p' \of \subst{\pi_1 p}{\alpha}{k_2}}
       {\Gamma \vd p \ape p' \of \Sigma\bind{\alpha \of k_1}{k_2}}

\inferr{\Gamma \vd c_1 \arrow c_2 \ape c_1' \arrow c_2' \of \type}
       {\Gamma \vd c_1 \ace c_1' \of \type \\ \Gamma \vd c_2 \ace c_2' \of \type}

\inferr{\Gamma \vd \forall\bind{\alpha \of k}{c} \ape \forall\bind{\alpha \of k'}{c'} \of \type}
       {\Gamma \vd k \ace k' \of \kind \\ \Gamma, \alpha \of k \vd c \ace c' \of \type}
\end{mathpar}

If $\Gamma \vd c \ape c' \of k$ then $\Gamma \vd c \nk k$ also
$\exists\bind{k'}{\Gamma \vd c' \nk k'}$ and $\Gamma \vd k \equiv k' \of \kind$\\

Structural comparison:
\begin{mathpar}
\inferr{\Gamma \vd \type \ace \type \of \kind}{\strut}
\inferr{\Gamma \vd \singleton{c} \ace \singleton{c'} \of \kind}
       {\Gamma \vd c \ace c' \of \type}

\inferr{\Gamma \vd \Pi\bind{\alpha \of k_1}{k_2} \ace
                   \Pi\bind{\alpha \of k_1'}{k_2'}}
       {\Gamma \vd k_1 \ace k_1' \of \kind \\
        \Gamma, \alpha \of k_1 \vd k_2 \ace k_2' \of \kind}

\inferr{\Gamma \vd \Sigma\bind{\alpha \of k_1}{k_2} \ace
                   \Sigma\bind{\alpha \of k_1'}{k_2'}}
       {\Gamma \vd k_1 \ace k_1' \of \kind \\
        \Gamma, \alpha \of k_1 \vd k_2 \ace k_2' \of \kind}
\end{mathpar}

% TODO different from \le triangle less/equal subkind
$\Gamma \vd k \sk k'$
\begin{mathpar}
\inferr{\Gamma \vd \type \sk \type}{\strut}

\inferr{\Gamma \vd \singleton{c} \sk \type}{\strut}

\inferr{\Gamma \vd \singleton{c} \sk \singleton{c'}}{\Gamma \vd c \ace c' \of \type}

\inferr{\Gamma \vd \Pi\bind{\alpha \of k_1}{k_2} \sk
                   \Pi\bind{\alpha \of k_1'}{k_2'}}
       {\Gamma \vd k_1' \sk k_1 \\ \Gamma, \alpha \of k_1' \vd k_2 \sk k_2'}

\inferr{\Gamma \vd \Sigma\bind{\alpha \of k_1}{k_2} \sk
                   \Sigma\bind{\alpha \of k_1'}{k_2'}}
       {\Gamma \vd k_1 \sk k_1' \\ \Gamma, \alpha \of k_1 \vd k_2 \sk k_2'}
\end{mathpar}

% TODO checks against
$\Gamma \vd k \checking \kind$
\begin{mathpar}
\inferr{\Gamma \vd \type \checking \kind}{\strut}

\inferr{\Gamma \vd \singleton{c} \checking \kind}{\Gamma \vd c \checking \type}

\inferr{\Gamma \vd \Pi\bind{\alpha \of k_1}{k_2} \checking \kind}
       {\Gamma \vd k_1 \checking \kind \\ \Gamma, \alpha \of k_1 \vd k_2 \checking \kind}
\end{mathpar}

% TODO
Suppose $\vd \Gamma \ok$. Then:\\

{\underline Soundness}\\
\begin{itemize}
\item If $\Gamma \vd c_1, c_2 \of k$ and $\Gamma \vd c_1 \ace c_2 \of k$ then
$\Gamma \vd c_1 \equiv c_2 \of k$
\item If $\Gamma \vd k_1, k_2 \of \kind$ and $\Gamma \vd k_1 \ace k_2 \of \kind$
then $\Gamma \vd k_1 \equiv k_2 \of \kind$
\item If $\Gamma \vd k_1, k_2 \of \kind$ and $\Gamma \vd k_1 \sk k_2$
then $\Gamma \vd k_1 \le k_2$
\item If $\Gamma \vd k \checking \kind$ then $\Gamma \vd k \of \kind$
\item If $\Gamma \vd c \synthesis k$ then $\Gamma \vd c \of k$
\end{itemize}

{\underline Completeness}\\
\begin{itemize}
\item If $\Gamma \vd c_1 \equiv c_2 \of k$ then $\Gamma \vd c_1 \ace c_2 \of k$
\item If $\Gamma \vd k_1 \equiv k_2 \of \kind$
then $\Gamma \vd k_1 \ace k_2 \of \kind$
\item If $\Gamma \vd k_1 \le k_2$ then $\Gamma \vd k_1 \sk k_2$
\item If $\Gamma \vd k \of \kind$ then $\Gamma \vd k \checking \kind$
\item If $\Gamma \vd c \of k$
then $\Gamma \vd c \synthesis k'$ and $\Gamma \vd k' \le \singleton{c \of k}$
\end{itemize}

\vspace{1cm}
TODO: principle type \\
TODO: principle kind is a subkind of every other kind \\

Checking principle...\\
$\Gamma \vd c \synthesis k$
\begin{mathpar}
\inferr{\Gamma \vd c \checking k}{\Gamma \vd c \synthesis k' \\ \Gamma \vd k' \sk k}
\end{mathpar}


% TODO
\begin{mathpar}
\inferr%[selfification]
       {\Gamma \vd \alpha \synthesis \singleton{\alpha \of k}}
       {\Gamma(\alpha) = k}

\inferr{\Gamma \vd \lambda\bind{\alpha \of k}{c} \synthesis \Pi\bind{\alpha \of k}{k'}}
       {\Gamma \vd k \checking \kind \\
        \Gamma, \alpha \of k \vd c \synthesis k'}

\inferr{\Gamma \vd c_1\ c_2 \synthesis \subst{c_2}{\alpha}{k'}}
       {\Gamma \vd c_1 \synthesis \Pi\bind{\alpha \of k}{k'} \\
        \Gamma \vd c_2 \checking k}

\inferr{\Gamma \vd \langle c_1, c_2 \rangle \synthesis k_1 \times k_2}
       {\Gamma \vd c_1 \synthesis k_1 \\ \Gamma \vd c_2 \synthesis k_2}

\inferr{\Gamma \vd \pi_1 c \synthesis k_1}
       {\Gamma \vd c \synthesis \Sigma\bind{\alpha \of k_1}{k_2}}

\inferr{\Gamma \vd \pi_2 c \synthesis \subst{\pi_1 c}{\alpha}{k_2}}
       {\Gamma \vd c \synthesis \Sigma\bind{\alpha \of k_1}{k_2}}

\inferr{\Gamma \vd c_1 \arrow c_2 \synthesis \singleton{c_1 \arrow c_2}}
       {\Gamma \vd c_1 \checking \type \\ \Gamma \vd c_2 \checking \type}

\inferr{\Gamma \vd \forall\bind{\alpha \of k}{c} \synthesis
                        \singleton{\forall\bind{\alpha \of k}{c}}}
       {\Gamma \vd k \checking \kind \\ \Gamma, \alpha \of k \vd c \checking \type}
\end{mathpar}

% TODO: algorithmic kind formation call what again?




\newpage
\section{Checking Expressions}

$\Gamma \vd e \synthesis \tau$
\begin{mathpar}
\inferr{\Gamma \vd x \synthesis \tau}{\Gamma(x) = \tau}

\inferr{\Gamma \vd e_1\ e_2 \synthesis \tau'}
       {\Gamma \vd e_1 \synthesis \tau_1 \\
        \Gamma \vd \tau_1 \whn \tau \arrow \tau' \\
        \Gamma \vd e_2 \checking \tau}
\end{mathpar}

\section{Type-Directed Translation / Syntax-Directed Translation}
A more accurate name: ``Typing-derivation-directed translation''.
We proceed by the analysis of the typing derivation of the rules.

\newcommand{\target}[1]{\textcolor{green}{#1}}
\newcommand{\tto}{\translatesto}
Let's represent the source and target languages in different colors, to
indicate that they are different.\\

Property:\\
$\Gamma \vd e \of \tau$ if and only if $\exists\bind{\target{e}}{\Gamma \vd e \of \tau \translatesto \target{e}}$.\\

We also want:\\
If $\Gamma \vd e \of \tau \tto \target{e}$, something like
$\target{\Gamma \vd e \of \tau}$.\\
But we have no concept of $\target{\Gamma}$ or $\target{\tau}$ or its derivations.\\
Instead:\\
Property:\\
If $\Gamma \vd e \of \tau \tto \target{e}$
and $\tau \tto \target{\tau}$
and $\Gamma \tto \target{\Gamma}$,
then $\target{\Gamma \vd e \of \tau}$. \\
Why not $\Gamma \vd \tau \of \type \tto \tau$.\\

Simply, we'll use `` If $\Gamma \vd e \of \tau \tto \target{e}$
then $\target{\Gamma} \vd \target{e} \of \target{\tau}$

\subsection{Coherence}
For Terms:\\
Suppose $\Gamma \vd e \of \tau \tto \target{e}$ and
$\Gamma \vd e \of \tau \tto \target{e'}$.\\
$\target{\Gamma \vd e \cong e' \of \tau}$.\\
This is too hard to even define, this is left to graduate courses.
We aspire to it but it's too much of a pain to actually do.\\

For Types:\\
Suppose $\Gamma \vd c \of k \tto \target{c}$ and
$\Gamma \vd c \of k \tto \target{c'}$.\\
Then,\\
$\Gamma \vd c \equiv c' \of k$. \\
This is not an aspiration, we cannot live without this. \\
The 2nd property above can't even be made without this, but it doesn't have
to be kind directed. And instead, we'll just make it syntax directed, which
will trivially prove that the two are equivalent.


\subsection{Definition of $\target{e}$}
\begin{align*}
\target{\alpha} &= \alpha \\
\target{\tau_1 \times \tau_2} &= \target{\tau_1} \times \target{\tau_2} \\
\target{\tau_1 \arrow \tau_2} &= \unit \arrow \target{\tau_1} \arrow \target{\tau_2} \\
&\ldots \\
\target{\epsilon} &= \epsilon \\
\target{\Gamma, x \of \tau} &= \target{\Gamma}, x \of \target{\tau} \\
\target{\Gamma, \alpha \of k} &= \target{\Gamma}, \alpha \of \target{k} \\
\end{align*}

Convoluted example:\\
\begin{mathpar}
\inferr{\Gamma \vd \lambda\bind{x \of \tau}{e} \of \tau \arrow \tau'
                    \tto \lambda\bind{z \of \unit}{\lambda\bind{x \of \target{\tau}}{\target{e}}}}
       {\Gamma \vd \tau \of \type \\
        \Gamma, x \of \tau \vd e \of \tau \tto \target{e}}
\end{mathpar}
NOTE: the right part (after $\tto$) indicates the need to shift in terms of
debruijn indices. \\

More: \\
\begin{mathpar}
\inferr{\Gamma \vd e_1\ e_2 \of \tau' \tto \target{e_1} <> \target{e_2}}
       {\Gamma \vd e_1 \of \tau \arrow \tau' \tto
          \target{e_1}^{\of \target{\tau_1 \arrow \tau_2}
                          = \unit \arrow \target{\tau} \arrow \target{\tau'}} \\
        \Gamma \vd e_2 \of \tau \tto \target{e_2}^{\of \target{\tau}}}
\end{mathpar}

More (only well typed things translate): \\
\begin{mathpar}
\inferr{\Gamma \vd e_1\ e_2 \synthesis \tau' \tto \target{e_1} <> \target{e_2}} % TODO what is <>
       {\Gamma \vd e_1 \synthesis \tau_1 \tto \target{e_1} \\
        \Gamma \vd e_1 \whn \tau \arrow \tau' \\
        \Gamma \vd e_2 \synthesis \tau_2 \tto \target{e_2} \\
        \Gamma \vd \tau_2 \ace \tau \of \type}
        % NOTE: we don't need this last one, we can optimize by skipping this rule
\end{mathpar}


\subsection{$\Gamma \vd e \of \tau \translatesto \target{e}$}
\begin{mathpar}
\inferr{\Gamma \vd x \of \tau \translatesto \target{x}}{\Gamma(x) = \tau}

\inferr{\Gamma \vd \langle e_1, e_2 \rangle \of \tau_1 \times \tau_2
                        \tto \target{\langle e_1, e_2 \rangle}}
       {\Gamma \vd e_1 \of \tau_1 \tto \target{e_1} \\
        \Gamma \vd e_2 \of \tau_2 \tto \target{e_2}}

\inferr{\Gamma \vd \pi_1 e \of \tau_1 \tto \target{\pi_1 e}}
       {\Gamma \vd e \of \tau_1 \times \tau_2 \tto \target{e}}
\end{mathpar}


\newpage
\section{More Things}
\begin{mathpar}
\inferr{\Gamma \vd c \equiv c' \of 1}{\Gamma \vd c \of 1 \\ \Gamma \vd c' \of 1}

\inferr{\Gamma \vd \ast \of 1}{\strut}
\\
\inferr{\Gamma \vd (\texttt{pack} [c, e] \mathop{as}
          \exists\bind{\alpha \of k}{\tau}) \of \exists\bind{\alpha \of k}{\tau}}
       {\Gamma \vd c \of k \\ \Gamma \vd e \of \subst{c}{\alpha}{\tau} \\
        \Gamma, \alpha \of k \vd \tau \of \type}

\inferr{\Gamma \vd \texttt{unpack} [\alpha, x] = e_1 in e_2 \of \tau'}
       {\Gamma \vd e_1 \of \exists\bind{\alpha \of k}{\tau} \\
        \Gamma, \alpha \of k, x \of \tau \vd e_2 \of \tau' \\
        \Gamma \vd \tau' \of \type}
\\
% TODO: recursive type rules
\inferr{\Gamma \vd \texttt{newtag} [\tau] \of \texttt{tag} t}{\Gamma \vd \tau \of \type}

\inferr{\Gamma \vd \texttt{tag} (e_1, e_2) \of \exn}
       {\Gamma \vd e_1 \of \texttt{tag} \\ \Gamma \vd e_2 \of \tau}

\inferr{\Gamma \vd \texttt{iftag}(e_1, e_2, \bind{x}{e_3}, e_4) \of \tau'}
       {\Gamma \vd e_1 \of \texttt{tag} \\
        \Gamma \vd e_2 \of \exn \\
        \Gamma, x \of e \vd e_3 \of \tau' \\
        \Gamma \vd e_4 \of \tau'}
\\
\inferr{\Gamma \vd \texttt{unpack} [\alpha, x] = e_1 in e_2 \of \tau'}
       {\Gamma \vd e_1 \of \exists\bind{\alpha \of k}{\tau} \\
        \Gamma, \alpha \of k, x \of \tau \vd e_2 \of \tau' \\
        \Gamma \vd \tau \of \type}

\inferr{\Gamma \vd \texttt{unpack} [\alpha, x] = e_1 in e_2 \synthesis \subst{c}{\alpha}{\tau'}}
       {\Gamma \vd e_1 \synthesis \tau_1 \\
        \Gamma \vd c \of k \\
        \Gamma \vd e_1 \whn \exists\bind{\alpha \of k}{\tau} \\
        \Gamma, \alpha \of k, x \of \tau \vd e_2 \synthesis \tau' \\
        \Gamma, \alpha \of k \vd \tau \ace \subst{c}{\alpha}{\tau'} \of \type}
\end{mathpar}
% TODO
%For proof, suppose $\Gamma, \alpha \of k \vd \tau \of \type$ and
%$\Gamma \vd \tau' \of \type$ and
%$\Gamma, \alpha \of k \vd \tau \equiv \tau' \of \type$ and $\Gamma \vd c \of k$.\\
%$\therefore \Gamma \vd \subst{c}{\alpha}{\tau} \equiv \subst{c}{\alpha}{\tau'} \of \type$
%Note that $\subst{c}{\alpha}{\tau'} \equiv \tau'$.\\
%$\therefore \Gamma, \alpha \of k \vd \subst{c}{\alpha}{\tau} \equiv \tau' \of \type$. \\
%Since $\tau' \equiv \tau$, by transitivity, $\subst{c}{\alpha}{\tau'} \equiv \tau$.


\newpage
\section{Continuation-Passing Style (CPS)}
\begin{itemize}
\item control-flow is explicit
\item name all intermediate results
\item reify control-flow (continuations) as data
\end{itemize}
The first two are often called ``monadic form'' or in literature,
``A-normal form'' (or by Harper, 2/3 CPS).


\subsection{Target Language}
Still have $k$, $c$, and now we have expressions $e$ (which do not return)
and values $v$. \\
\begin{align*}
k & \bnfdef \dots \\
c & \bnfdef \dots \bnfalt \cancel{\tau \arrow \tau}
    \bnfalt \cancel{\forall\bind{\alpha \of k}{\tau}}
    \bnfalt \neg\tau \\
e & \bnfdef \ap{v}{v} \bnfalt \texttt{\unpack} [\alpha,x] = v in e
    \bnfalt \letbind{x}{\pi_i v}{e}
    \bnfalt \letbind{x}{v}{e}
    \bnfalt \texttt{halt}
    \bnfalt \dots
v & \bnfdef x \bnfalt \lambda\bind{x \of \tau}{e}
    \bnfalt \texttt{pack} [c,v] as \exists\bind{\alpha \of k}{\tau}
    \bnfalt \langle v_1, \dots, v_n \rangle
    \bnfalt \dots \\
\end{align*}

Judgements: \\
$\Gamma \vd v \of \tau$ \\
$\Gamma \vd e \of 0$ \\
($e$ does not return, so we use `0' for `OK')

\begin{mathpar}
\inferr{\Gamma \vd \lambda\bind{x \of \tau}{e} \of \neg\tau}
       {\Gamma \vd \tau \of \type \\ \Gamma, x \of \tau \vd e \of 0}

\inferr{\Gamma \vd \ap{v_1}{v_2} \of 0}
       {\Gamma \vd v_1 \of \neg\tau \\ \Gamma \vd v_2 \of \tau}

\inferr{\Gamma \vd \texttt{pack} [c,v] as \exists\bind{\alpha \of k}{\tau} \of
                                          \exists\bind{\alpha \of k}{\tau}}
       {\Gamma \vd c \of k \\
        \Gamma \vd v \of \subst{c}{\alpha}{\tau} \\
        \Gamma, \alpha \of k \vd \tau \of \type}

\inferr{\Gamma \vd \texttt{unpack} [\alpha,x] = v \in e \of 0}
       {\Gamma \vd v \of \exists\bind{\alpha \of k}{\tau}
        \Gamma, \alpha \of k, x \of \tau \vd e \of 0}

\inferr{\Gamma \vd \langle v_1, \dots v_2 \rangle \of x[\tau, \dots \tau_n]}
       {\Gamma \vd v_i \of \tau_i \\ (\forall i \in [n])}

\inferr{\Gamma v \letbind{x}{\pi_i v}{e} \of 0}
       {\Gamma \vd v \of x[\tau_0, \dots \tau_{n - 1} \\
        \Gamma, x \of \tau_i \vd e \of 0}

\inferr{\Gamma \vd \letbind{x}{v}{e} \of 0}
       {\Gamma \vd v \of \tau \\ \Gamma, x \of \tau \vd e \of 0}

\inferr{\Gamma \vd \halt \of 0}{\strut}
\end{mathpar}


\subsection{Translation}
(NOTE: this is syntax-directed)

\begin{flalign*}
\target{T} &= T &\\
\target{\Pi\bind{\alpha \of k_1}{k_2}} &= \Pi\bind{\alpha \of \target{k_1}}{\target{k_2}} &\\
\target{\singleton{c}} &= \singleton{\target{c}} &\\
\target{1} &= 1 &\\
\target{\alpha} &= \alpha &\\
\target{\lambda\bind{\alpha \of k}{c}} &= \lambda\bind{\alpha \of \target{k}}{\target{c}} &\\
\target{c_1\ c_2} &= \target{c_1}\ \target{c_2} &\\
\target{\langle c_1, c_2 \rangle} &= \langle \target{c_1}, \target{c_2} \rangle &\\
\target{\pi_1 c} &= \pi_1 \target{c} &\\
\target{\pi_2 c} &= \pi_2 \target{c} &\\
& &\\
\target{\tau_1 \arrow \tau_2} &= \neg(\target{\tau_1} \times \neg\target{\tau_2}) &\\
\target{x[\tau_1, \dots, \tau_n]} &= x[\target{\tau_1}, \dots, \target{\tau_n}] &\\
\target{\forall\bind{\alpha \of k}{\tau}} &=
  \neg(\exists\bind{\alpha \of \target{k}}{\neg\target{\tau}}) &\\
\target{\exists\bind{\alpha \of k}{\tau}} &=
  \exists\bind{\alpha \of \target{k}}{\target{\tau}} &\\
& &\\
\target{\epsilon} &= \epsilon &\\
\target{\Gamma, \alpha \of k} &= \target{\Gamma}, \alpha \of \target{k} &\\
\target{\Gamma, x \of \tau} &= \target{\Gamma}, x \of \target{\tau} &\\
\target{[c_1/\alpha]c_2} &= [\target{c_1}/\alpha]\target{c_2} &\\
\target{\alpha} &= \alpha &\\
\end{flalign*}

Type directed translation: \\
Judgement: \\
$\Gamma \vd e \of \tau \tto \bind{x}{\target{e}}$ \\
Note here that this an expression where we compute the value of $\target{e}$
and send it to the continuation $x$. \\

Type Principle: \\
If $\Gamma \vd e \of \tau \tto \bind{x}{\target{e}}$ (and $\vd \Gamma \ok$) then
$\target{\Gamma}, k \of \neg\target{\tau} \vd \target{e} \of 0$. \\
\note{$k$ is not the metavariable for kind in this case.}

\subsection{$\Gamma \vd e \of \tau \tto \bind{k}{\target{e}}$}
\begin{mathpar}
\inferr{\Gamma \vd x \of \tau \tto \bind{k \of \neg\target{\tau}}{k x}}
       {\Gamma(x) = \tau}

\inferr{\Gamma \vd \pi_i e \of \tau_i \tto \bind{k \of \not \target{\tau_i}}{
        \letbind{k'}{\bind{\lambda x \of \ast[\target{\tau_0}, \dots, \target{\tau_{n - 1}}]}
            {\letbind{y}{\pi_i x}{k y}}}{\target{e}}
       }}
       {\Gamma \vd e \of x[\tau_0, \dots, \tau_{n - 1} \tto
          \bind{k' \of \neg\target{x[\tau_0, \dots, \tau_{n - 1}]}}{\target{e}}}

\inferr{\Gamma \vd \langle e_1, \dots e_n \rangle \of \ast[\tau_i, \dots, \tau_n] \tto \\
        \bind{k \of \neg x[\target{\tau_1}, \dots, \target{\tau_n}}{
          \letbind{k_1}{\lambda\bind{x_1 \of \target{\tau_1}}{
            \letbind{k_2}{\lambda\bind{x_2 \of \target{\tau_2}}{
              \dots
              \letbind{k_n}{\lambda\bind{x_n}{\target{\tau_n}}{
                k \langle x_1, \dots, x_n
              }}{\target{e_n}}
            }}{\target{e_2}}
          }}{\target{e_1}}
       }}
       {\Gamma \vd e_i \of \tau_i \tto \bind{k_i \of \neg\target{\tau_i}}{\target{e_i}} \\
        (i = 1 \dots n)}

\inferr{\Gamma \vd \lambda\bind{x \of \tau_1}{e} \of \tau_1 \arrow \tau_2 \tto
        \bind{k^{\of \not\target{\tau_1 \arrow \tau_2} =
                     \not\not(\target{\tau_1} \times \not \target{\tau_2})}}
        {k(\lambda\bind{y \of \target{\tau_1} \times \target{\tau_2}}
          \letbind{x}{\pi_0 y}{\letbind{k}{\pi_1 y}{\target{e}}})}}
       {\Gamma \vd \tau_1 \of \type \\
        \Gamma, x \of \tau_1 \vd e \of \tau_2 \tto \bind{k' \of \not\target{\tau_2}}{\target{e}}}

\inferr{\Gamma \vd e_1\ e_2 \of \tau' \tto \bind{k^{\neg\target{\tau'}}}{
          \letbind{k_1}{
            \lambda\bind{f \of \neg(\target{\tau} \times \neg\target{\tau'}}
                        {\letbind{k_2}{\lambda\bind{x \of \target{\tau}}
                                      {f \langle x, k \rangle}}
                        }{\target{e_2}
          }}{\target{e_1}}}}
       {\Gamma \vd e_1 \of \tau \arrow \tau' \tto
          \bind{k^{\of \neg\target{\tau \arrow \tau'} =
                   \neg\neg(\target{\tau_1} \times \neg \target{\tau_2})}}
               {\target{e}} \\
        \Gamma \vd e_2 \of \tau \tto \bind{k_2^{\of \neg\target{\tau}}}{\target{e_2}}}

\inferr{\Gamma \vd \texttt{pack}[c, e] \as \exists\bind{\alpha \of k}{\tau} \of \exists\bind{\alpha \of k}{e} \tto \\
        \bind{k^{\of \neg\target{\exists\bind{\alpha \of k}{\tau}} = 
                 \neg\exists\bind{\alpha \of \target{k}}{\target{\tau}}}}
             {\letbind{k' \of \lambda\bind{x \of \target{\subst{c}{\alpha}{\tau}}}
                                          {k(\texttt{pack}[\target{c}, x] \as \exists\bind{\alpha \of \target{k}}{\target{\tau}}}}{}{\target{e}}}
       }
       {\Gamma \vd c \of k \\
        \Gamma \vd e \of \subst{c}{\alpha}{\tau} \tto \bind{k' \of \neg\target{[c/\alpha]\tau}}{\target{e}} \\
        \Gamma, \alpha \of k \vd \tau \of \type}

\inferr{\Gamma \vd \unpack[\alpha, x] = e_1 \texttt{in} e_2 \tto \\
          \bind{k^{\of \neg \target{\tau'}}}{
            \letbind{k_1}{
            \lambda\bind{x_1 \of \exists{\alpha \of \target{k}}{\target{\tau}}}
                        {\unpack[\alpha, x] = x_1 \texttt{in} [k/k_2]\target{e_2}}
            }{\target{e_1}}
          }
       }
       {\Gamma \vd e_1 \of \exists\bind{\alpha \of k}{\tau} \tto
          \neg\exists\bind{\alpha \of \target{k}}{\target\tau} \\
        \Gamma, \alpha \of k, x \of \tau \vd e_2 \of \tau' \tto
          \bind{k_2^{\of \neg\target{\tau'}}}{\target{e_2}} \\
        \Gamma \vd \tau \of \type
       }

\inferr{\Gamma \vd \Lambda\bind{\alpha \of k}{e} \of \forall\bind{\alpha \of k}{\tau} \tto
          \bind{k^{\of \neg\target{\forall\bind{\alpha \of k}{\tau}} = \neg\neg(\exists\bind{\alpha \of \target{k}}{\target{\tau}})}}{k(\lambda\bind{x \of \exists\bind{\alpha \of \target{k}}{\neg\target{\tau}}}{\unpack[\alpha, k'] = x \texttt{in} \target{e}})}
       }
       {\Gamma \vd k \of \kind \\
        \Gamma, \alpha \of k \vd e \of \tau \tto \bind{k'^{\of \neg\target{\tau}}}{\target{e}}}
\end{mathpar}


% Continued...
\begin{mathpar}
\inferr{\Gamma \vd e[c] \of \subst{c}{\alpha}{\tau} \tto
          \bind{k^{\of \neg\target{\subst{c}{\alpha}{\tau}}}}
            {\letbind{k'}{\lambda\bind{f \of \neg(\exists\bind{\alpha \of \target{k}}{\neg\target{\tau}})}{f(\pack [\target{c}, k] \as \exists\bind{\alpha \of \target{k}}{\neg\target{\tau}})}}{\target{e}}}
       }
       {\Gamma \vd e \of \forall\bind{\alpha \of k}{\tau} \tto
          \bind{k'^{\of \neg\target{\forall\bind{\alpha \of k}{\tau}} = \neg\neg(\exists\bind{\alpha}{\target{k}}{\neg\target{\tau}})}}{\target{e}} \\
        \Gamma \vd c \of k}
\end{mathpar}
Note $\neg\subst{\target{c}}{\alpha}{\target{\tau}} = \neg\target{\subst{c}{\alpha}{\tau}}$. \\

% TODO:
% We've talked about:
%   Sums
%   References
%   Exns
%   Primitives
%   Recursive Types
% Now, we need to talk about:
%   Exceptions

\subsection{Exceptions}
\begin{flalign*}
\target{\tau_1 \arrow \tau_2} &= \neg(\times[\target{\tau_1}, \neg\target{\tau_2}, \neg\exn] &\\
\target{\forall\bind{\alpha \of k}{\tau}} &=
  \neg(\exists\bind{\alpha \of \target{k}}{\neg\target\tau \times \neg\exn}) &\\
\end{flalign*}

Judgement: $\Gamma \vd e \of \tau \tto
              \bind{k^{\of \neg\target\tau} k_{ex}^{\of \neg\exn}}{\target{e}}$
\begin{mathpar}
\inferr{\Gamma \vd x \of \tau \tto \bind{k k_{ex}}{k x}}{\Gamma(x) = \tau}

\inferr{\Gamma \vd \langle e_1, \dots, e_n \rangle \of x[\tau_1, \dots, \tau_n] \tto
          \bind{k k_{ex}}{\letbind{k_1}{
            \lambda\bind{x_1 \of \target{\tau_1}}{\dots
              \letbind{k_n}{\lambda\bind{x_n \of \target{\tau_n}}
                                        {k \langle x_i, \dots x_n \rangle}
              }{\target{e_n}} \dots
            }
          }{\target{e_1}}}
       }
       {\Gamma \vd e_i \of \tau_i \tto
          \bind{k_i^{\of \neg\target{\tau_1}} k_{ex_i}^{\of \neg\exn}}{\target{e_i}} \\
        (i = 1 \dots n)}

\inferr{\Gamma \vd \texttt{raise}_{\tau}{e} \of \tau \tto
          \bind{k^{\of \neg\target\tau} k_{ex}^{\of \neg\exn}}{
            \letbind{k'}{k_{ex}}{\target{e}}
          }
       }
       {\Gamma \vd e \of \type \\
        \Gamma \vd e \of \exn \tto
          \bind{k'^{\of \neg\target{\exn}} k_{ex}'^{\of \neg\exn}}{\target{e}}}

\inferr{\Gamma \vd \texttt{handle}(e_1, \bind{x}{e_2} \of \tau \tto
          \bind{k^{\of \neg\target\tau} k_{ex}^{\of \neg\exn}}{
            \letbind{k_{em}}{\lambda\bind{x \of \exn}{\target{e_2}}}{\target{e_1}}
          }
       }
       {\Gamma \vd e \of \tau \tto \bind{k^{\of \neg\target\tau} k_{ex1}}{\target{e_1}} \\
        \Gamma, x \of \exn \vd e_2 \of \tau \tto
          \bind{k^{\of \neg\target\tau} k_{ex}}{\target{e_2}}}

\inferr{\Gamma \vd \lambda\bind{x \of \tau_1}{e} \of \tau_1 \arrow \tau_2 \tto
          \bind{k^{\of \neg\target{\tau_1 \arrow \tau_2} = \neg\neg(\times[\target{\tau_1}, \dots \target{\tau_n}} k_{ex}}{k(\lambda\bind{y \of x[\target{\tau_1}, \neg\target{\tau_2}, \neg\exn]}{
            \letbind{x}{\pi_0 y}{\letbind{k'}{\pi_1 y}{\letbind{k_{ex}'}{\pi_2 y}{\target{e}}}}
          }}
       }
       {\Gamma \vd \tau_1 \of \type \\
        \Gamma, x \of \tau_1 \vd e \of \tau_2 \tto
          \bind{k'^{\neg\target{\tau_2}} k_{ex}'}{\target{e_2}}}

\inferr{\Gamma \vd \ap{e_1}{e_2} \of \tau' \tto
          \bind{k^{\neg\target{\tau'}} k_{ex}}{
            \letbind{k_1}{
              (\lambda\bind{f \of \neg(x[\target\tau, \neg\target{\tau'}, \neg\exn])}{
                \letbind{k_2}{
                  \lambda\bind{x \of \target\tau}{f \langle x, k, k_{ex} \rangle}
                }{\target{e_2}}
              })
            }{\target{e_1}}
          }
       }
       {\Gamma \vd e_1 \of \tau \arrow \tau' \tto
          \bind{k_1^{\neg\target{\tau \arrow \tau'} = \neg\neg(\times[\target{\tau}, \neg\target{\tau'}, \neg\exn])} k_{ex}}{\target{e_1}} \\
        \Gamma \vd e_2 \of \tau \tto \bind{k_2^{\neg\target\tau} k_{ex2}}{\target{e_2}}}
\end{mathpar}

\subsection{Continuations}
\begin{lstlisting}
callcc : ('a cont $\arrow$ 'a) $\arrow$ 'a
throw : ('a cont $\times$ 'a) $\arrow$ 'b
\end{lstlisting}

Typing Rules
\begin{mathpar}
\inferr{\Gamma \vd \letcc{\tau}{x}{e} \of \tau}
       {\Gamma \vd \tau \of \type \\
        \Gamma, x \of \cont{\tau} \vd e \of \tau}

\inferr{\Gamma \vd \throw{\tau'}{e_1}{e_2} \of \tau'}
       {\Gamma \vd \tau' \of \type \\
        \Gamma \vd e_1 \of \tau \\
        \Gamma \vd e_2 \of \cont{\tau}}
\end{mathpar}

Translation Rules
\begin{mathpar}
\inferr{\Gamma \vd \letcc{\tau}{x}{e} \of \tau \tto
          \bind{k^{\of \neg\target\tau}}{\letbind{k'}{k}{\letbind{x}{k}{\target{e}}}}}
       {\Gamma \vd \tau \of \type \\
        \Gamma, x \of \cont{\tau} \vd e \of \tau \tto \bind{k'^{\neg\target\tau}}{\target e}}

\inferr{\Gamma \vd \throw{\tau'}{e_1}{e_2} \of \tau' \tto
          \bind{k^{\of\neg\target{\tau'}}}{
            \letbind{k_1}{
              \lambda\bind{x \of \tau}{
                \letbind{x_2}{
                  \lambda\bind{y \of \neg\target\tau}{\ap{y}{x}}
                }{\target{e_2}}
              }
            }{\target{e_1}}
          }
       }
       {\Gamma \vd \tau' \of \type \\
        \Gamma \vd e_1 \of \tau \tto \bind{k_1^{\of\neg\target\tau}}{\target{e_1}} \\
        \Gamma \vd e_2 \of \cont{\tau} \tto
          \bind{k_2^{\of \neg\target{\cont{\tau}} = \neg\neg\target\tau}}{\target{e_2}}}
\end{mathpar}


\section{Closure Conversion}

A closure is a tuple containing code and the environment that will be passed
in as an additional argument to the code. The code is closed, with no open
terms. \\

Concrete example: \\
$\lambda\bind{x \of \intt}{x + y + z}$ \\
$\leadsto
\langle (\lambda\bind{x \of \intt}{\lambda\bind{\env \of \intt \times \intt}
          {\letbind{y}{\pi_0 \env}{\letbind{z}{\pi_1 \env}{x + y + z}}}}),
\langle y, z \rangle \rangle$ \\
$\ap{e}{5}$ \\
$\leadsto
\letbind{f}{\pi_0 e}{\letbind{\env}{\pi_1 e}{\ap{\ap{f}{5}}{\env}}}$ \\

This can get really messy and really inefficient, so what we really want
to do is convert some curried and some tupled functions and turn them into
some special internal representation for multi-argument functions. \\

Types may be different if the environment is different, so when converting,
we can try something like, where we don't care about the existential, since
we won't be manipulating it: \\
$\target{\tau_1 \arrow \tau_2} = \exists\bind{\alpha_\env \of \type}
  {(\tau_1 \arrow \alpha_\env \arrow \tau_2) \times \alpha_\env}$

\subsection{Target Language (IL-Closure)}
\begin{flalign*}
\Delta;\Gamma &\vd e \of 0 &\\
\Delta;\Gamma &\vd v \of \tau &\\
\Gamma &\bnfdef \epsilon \bnfalt \Gamma, x \of \tau &\\
\Delta &\bnfdef \epsilon \bnfalt \Delta, \alpha \of k &\\
\end{flalign*}

Only rule additional we need: \\
\begin{mathpar}
\inferr{\Delta; \Gamma \vd \lambda\bind{x \of \tau}{e} \of \neg\tau}
       {\Delta \vd \tau \of \type \\
        \Gamma; (\epsilon, x \of \tau) \vd e \of 0}
\end{mathpar}

\subsection{Type Translation}
\begin{flalign*}
\target{\alpha} &= \alpha &\\
&\vdots &\\
\target{\neg\tau} &= \exists\bind{\alpha_\env \of \type}
  {\neg(\target{\tau} \times \alpha_\env) \times \alpha_\env} &\\
\target{x[\tau_1, \dots, \tau_2]} &= x[\target{\tau_1}, \dots, \target{\tau_n}] &\\
\end{flalign*}
How would we deal with $\target{\forall\bind{\alpha \of k}{\tau}}$? Turns out
it's really hard. But because of how we got rid of them back during CPS
conversion, we don't even have to deal with it anymore!

\subsection{Type Principle}
If $\Delta; \Gamma \vd e \of 0 \tto \target{e}$ then $\target{\Delta}; \target{\Gamma} \vd \target{e} \of 0$. \\
If $\Delta; \Gamma \vd v \of \tau \tto \target{v}$ then $\target{\Delta}; \target{\Gamma} \vd \target{v} \of \target{\tau}$. \\

\subsection{$\Delta; \Gamma \vd e \of 0 \tto \target{e}$,
            $\Delta; \Gamma \vd v \of \tau \tto \target{v}$}
\begin{mathpar}
\inferr{\Delta; \Gamma \vd x \of \tau \tto x}{\Gamma(x) = \tau}

\inferr{\Delta; \Gamma \vd \langle v_1, \dots v_n \rangle \of x[\tau_1, \dots, \tau_n] \tto
          \langle \target{v_1}, \dots, \target{v_n} \rangle}
       {\Delta; \Gamma \vd v_i \of \tau_i \tto \target{v_i}}

\small
\inferr{\Delta; \Gamma \vd \lambda\bind{x \of \tau}{e} \of \neg\tau \tto
        \pack[x[\target{\tau_1}, \dots, \target{\tau_n}],
        \langle \\
          (\lambda\bind{y \of \tau \times x[\target{\tau_1}, \dots, \target{\tau_n}]}{
            \letbind{x}{\pi_0 y}{\letbind{\env}{\pi_1 y}{
              \letbind{x_1}{\pi_0\env}{\dots \letbind{x_n}{\pi_{n - 1}\env}{\target{e}}}
            }}
          }),
        \\ \langle x_1, \dots, x_n \rangle \rangle] \as
        \exists\bind{\alpha_\env \of \type}
          {\neg(\target{\tau} \times \alpha_\env) \times \alpha_\env}
       }
       {\Delta \vd \tau \of \type \\
        \Delta; \Gamma, x \of \tau \vd e \of 0 \tto \target{e} \\
        \Gamma = x_1 \of \tau_1, \dots, x_n \of \tau_n}
\normalfont

\inferr{\Delta; \Gamma \vd \ap{v_1}{v_2} \of 0 \tto
          \unpack[\alpha_\env, x] = \target{v_1} \texttt{in}
            \letbind{f}{\pi_0 x}{\letbind{\env}{\pi_1 x}{
              f \langle \target{v_2}, \env \rangle
            }}
       }
       {\Delta; \Gamma \vd v_1 \of \neg\tau \tto
          \target{v_1}^{\of \target{\neg\tau} =
                        \exists\bind{\alpha}{\neg(\target\tau \times \alpha) \times \alpha}} \\
        \Delta; \Gamma \vd v_2 \of \tau \tto \target{v_2}^{\of \target\tau}}
\end{mathpar}

\vspace{1cm}
% TODO
To solve some inefficiency, instead of passing around the environment,
instead, pass around the closure. \\
In environment passing, we had 
$\target{\tau_1 \arrow \tau_2} = \exists\bind{\alpha_\env \of \type}
  {(\tau_1 \times \alpha_\env \arrow \tau_2) \times \alpha_\env}$. \\
In closure passing, we have
$\mu\bind{\beta}{\exists\bind{\alpha_\env}{\target{\tau_1} \times \beta \arrow \target{\tau_2} \times \alpha_\env}}$. \\

We can apparently use this to understand objected oriented programming better
and some of the research might not even be all that wrong. \\

There's more stuff one can do $\dots$. \\


\section{Hoisting}
Type Translation: \\
$\target{\tau} = \tau$ \\

\subsection{Target Language: IL-Hoist}
\begin{align*}
c & \bnfdef \dots \bnfalt \forall\bind{\alpha \of k}{c} \\
e & \bnfdef \dots \\
v & \bnfdef \dots \bnfalt \lambda\bind{x \of \cancel{\tau}}{e}
    \bnfalt v[c] \bnfalt \cancel{\Lambda\bind{\alpha \of k}{v}} \\
f & \bnfdef \lambda\bind{x \of \tau}{e} \bnfalt \Lambda\bind{\alpha \of k}{f} \\
b & \bnfdef x \mathrel{=} f \\
P & \bnfdef \texttt{let}~b~\texttt{in}~P \bnfalt e \\
\end{align*}
To hoist the type: `type-erasure semantics`.


\subsection{Judgements}
$\Delta; \Gamma \vd e \of 0 \tto let \vec{b} in \target{e}$ \\
$\Delta; \Gamma \vd v \of \tau \tto let \vec{b} in \target{v}$ \\

\subsection{$\Delta; \Gamma \vd f \of \tau$}
\begin{mathpar}
\inferr{\Delta; \Gamma \vd \Lambda\bind{\alpha \of k}{f} \of \forall\bind{\alpha \of k}{\tau}}
       {\Delta, \alpha \of k; \Gamma \vd f \of \tau}

\inferr{\Delta; \Gamma \vd \lambda\bind{x \of \tau}{e} \of \neg\tau}
       {\Delta \vd \tau \of \type \\ \Delta; \Gamma, x \of \tau \vd e \of 0}
\end{mathpar}

\subsection{$\Gamma \vd P \of 0$}
\begin{mathpar}
\inferr{\Gamma \vd \letbind{x}{f}{P} \of 0}
       {\cdot; \Gamma \vd f \of \tau \\ \Gamma, x \of \tau \vd P \of 0}

\inferr{\Gamma \vd e \of 0}{\cdot; \Gamma \vd e \of 0}
\end{mathpar}

\subsection{$\Delta; \Gamma \vd v \of \tau \tto \letbinds{\vec{b}}{\target{v}}$}
\begin{mathpar}
\inferr{\Delta; \Gamma \vd x \of \tau \tto \letbinds{}{x}}{\Gamma(x) = \tau}

\inferr{\Delta; \Gamma \vd \prodn{v} \of \Prodn{\tau}
          \letbinds{\vec{b_1} \dots \vec{b_n}}{\langle{\prodn{\target{v}}}} % TODO: prodn target
       }
       {\Delta; \Gamma \vd v_i \of \tau_i \tto
          \letbinds{\vec{b_i}}{\target{v_i}} \\ (\text{for} i = 1 \dots n)}

\inferr{\Delta; \Gamma \vd v_1 v_2 \of 0 \tto
          \letbinds{\vec{b_1}, \vec{b_2}}{\target{v_1}\target{v_2}}}
       {\Delta; \Gamma \vd v_1 \of \neg\tau \tto \letbinds{\vec{b_1}}{\target{v_1}} \\
        \Delta; \Gamma \vd v_2 \of \tau \tto \letbinds{\vec{b_2}}{\target{v_2}}}

\inferr{\Delta; \Gamma \vd \lambda\bind{x \of \tau}{e} \of \neg\tau \tto
          \letbinds{\vec{b}, y \mathrel{=} \Lambda\bind{\alpha_1 \of k_1}{
            \dots \Lambda\bind{\alpha_n \of k_n}{\lambda\bind{x \of \target{\tau}}{\target{e}}}
          }}{y[\alpha_1]\dots[\alpha_n]}}
       {\Delta \vd \tau \of \type \\
        \Delta; x \of \tau \vd e \of 0 \tto \letbinds{\vec{b}}{\target{e}} \\
                % TODO: keep Lambda?
        y \not\in FV(\Gamma), y \ne x, y \not\in BV(\vec{b}) \\ % TODO same as \fresh{y}
        \Delta = \alpha_1 \of k_1, \dots, \alpha_n \of k_n}
\end{mathpar}

\subsection{$\Delta; \Gamma \vd e \of 0 \tto \letbinds{\vec{b}}{\target{v}}$}
\begin{mathpar}
\inferr{\vd_{\texttt{top}} e \of 0 \tto
          \letbinds{b_1}{\dots \letbinds{b_n}{\target{e}}}}
       {\cdot; \cdot \vd e \of 0 \tto \letbinds{\vec{b}}{\target{e}} \\
          (\vec{b} = b_1, \dots, b_n)}
\end{mathpar}


TODO: ASIDE PREVIOUS: CLOSURE CONVERSION
\begin{mathpar}
\inferr{\vd_{\texttt{top}} e \of 0 \tto \target{e}}
       {\cdot; \cdot \vd e \of 0 \tto \target{e}}
\end{mathpar}

TODO: ASIDE PREVIOUS: CPS CONVERSION
\begin{mathpar}
\inferr{\vd_{\texttt{top}} e \of \tau \tto \\
            \letbind{k}{
              \lambda\bind{x \of \target{\tau}}{\halt}
            }{\letbind{k_{ex}}{
              \lambda\bind{x \of \exn}{
                \letbind{\_}{\texttt{print "uncaught exception"}}{\halt}
              }
            }{\target{e}}}}
       {\cdot; \cdot \vd e \of \tau \tto \bind{k k_{ex}}{\target{e}}}
\end{mathpar}


\section{Alloc}

\subsection{Target Language: IL-Alloc}
\begin{align*}
a & \bnfdef x = \alloc{n} \bnfalt x = \pi_i y \bnfalt \pi_i y \Coloneq v \bnfalt x = v \\
e & \bnfdef \letbinds{a}{e} \bnfalt x\mathop{ }x \bnfalt \halt \\
v & \bnfdef x \bnfalt \dots \bnfalt \cancel{\prodn{v}} \\
f & \bnfdef \lambda\bind{x \of \tau}{e} \\
b & \bnfdef x \mathrel{=} f \\
P & \bnfdef \texttt{let}~b~\texttt{in}~P \bnfalt e \\
\end{align*}

\subsection{Judgements and Translations}
$\Delta; \Gamma \vd e \of 0 \tto \target{e}$ \\
$\Delta; \Gamma \vd v \of \tau \tto \letbinds{\vec{a}}{\target{v}}$

\subsection{$\Delta; \Gamma \vd v \of \tau \tto \letbinds{\vec{a}}{\target{v}}$}
\begin{mathpar}
\inferr{\Delta; \Gamma \vd \prodn{v} \of \Prodn{\tau} \tto
         \letbinds{\vec{a_1} \dots \vec{a_n}
           , x = \alloc{n}
           , \pi_0 x = \target{v_1}
           \dots
           , \pi_{n - 1} x = \target{v_n}
         }{x}
       }
       {\Delta; \Gamma \vd v_i \of \tau_i \tto \letbinds{\vec{a_i}}{\target{v_i}} \\
        (\texttt{for}~i = 1 \dots n) \\ \fresh{x}}
\end{mathpar}

\subsection{$\Delta; \Gamma \vd e \of 0 \tto \target{e}$}
\begin{mathpar}
\inferr{\Delta; \Gamma \vd \letbind{x}{v}{e} \of 0 \tto
          \letbinds{a_1}{\dots \letbinds{a_n}{\letbind{x}{\target{v}}{\target{e}}}}}
       {\Delta; \Gamma \vd v \of \tau \tto \letbinds{\vec{a}} in \target{v} \\
        \Delta; \Gamma, x \of \tau \vd e \of 0 \tto \target{e} \\
        \vec{a} = a_1, \dots, a_n}

\inferr{\Delta; \Gamma \vd v_1 v_2 \of 0 \tto
          \letbinds{\vec{a_1}}{\letbinds{\vec{a_2}}{\target{v_1} \target{v_2}}}}
       {\Delta; \Gamma \vd v_1 \of \neg\tau \tto \letbinds{\vec{a_1}}{\target{v_1}} \\
        \Delta; \Gamma \vd v_2 \of \tau \tto \letbinds{\vec{a_2}}{\target{v_2}}}
\end{mathpar}


\section{Module}
\subsection{Target Language: IL-Module}
\begin{align*}
k & \bnfdef \type \bnfalt \singleton{c} \bnfalt \Pi\bind{\alpha \of k_1}{k_2}
    \bnfalt \Sigma\bind{\alpha \of k_1}{k_2} \bnfalt 1 \\
c & \bnfdef \dots \bnfalt c_1 \arrow c_2 \bnfalt \forall\bind{\alpha \of k}{c} \bnfalt \dots \\
e & \bnfdef \dots \bnfalt \ext{M} \\
\sigma & \bnfdef 1 \bnfalt \satom{k} \bnfalt \datom{\tau}
         \bnfalt \Pi^\texttt{gen}\bind{\alpha \of \sigma_1}{\sigma_2} % generative functor
         \bnfalt \Pi^\texttt{app}\bind{\alpha \of \sigma_1}{\sigma_2} % applicative functor
         \bnfalt \Sigma\bind{\alpha \of \sigma_1}{\sigma_2} \\
M & \bnfdef \ast \bnfalt \satom{k} \bnfalt \datom{e} \\
\end{align*}



\begin{align*}
k & \bnfdef \type \bnfalt \singleton{c} \bnfalt \Pi\bind{\alpha \of k_1}{k_2}
    \bnfalt \Sigma\bind{\alpha \of k_1}{k_2} \bnfalt 1 \\
\tau, c
  & \bnfdef \alpha \bnfalt \lambda\bind{\alpha \of k}{c} \bnfalt \ap{c}{c}
    \bnfalt \pair{c, c} \bnfalt \pi_1 c \bnfalt \pi_2 c \bnfalt \ast
    \bnfalt \tau \arrow \tau \bnfalt \prodn{\tau} \bnfalt \cancel{\ext{M}} \\
e & \bnfdef \dots \bnfalt \ext{M} \\
\sigma & \bnfdef 1 \bnfalt \satom{k} \bnfalt \datom{\tau}
         \bnfalt \Pi^\texttt{gen}\bind{\alpha\cancel{/s} \of \sigma_1}{\sigma_2}
         % generative functor
         \bnfalt \Pi^\texttt{app}\bind{\alpha\cancel{/s} \of \sigma_1}{\sigma_2}
         % applicative functor
         \bnfalt \Sigma\bind{\alpha\cancel{/s} \of \sigma_1}{\sigma_2} \\
M & \bnfdef s \bnfalt \ast \bnfalt \satom{c} \bnfalt \datom{e}
    \bnfalt \lambda^\texttt{gen}\bind{\alpha/s \of \sigma}{M}
    \bnfalt \lambda^\texttt{app}\bind{\alpha/s \of \sigma}{M}
    \bnfalt \Ap{M}{M}
    \bnfalt \ap{M}{M}
    \bnfalt \pair{M, M}
    \bnfalt \Pi_1 M
    \bnfalt \Pi_2 M
    \bnfalt M :> \sigma
    \\
\Gamma & \bnfdef \epsilon \bnfalt \Gamma, \alpha \of k \bnfalt \Gamma, x \of \tau
         \bnfalt \Gamma, \alpha/s \of \sigma \\
\end{align*}

$\sigma$ for signatures \\
$M$ for modules \\
$\alpha/s$ is called twinning.
$\alpha$ stands for the static portion of $s$. \\
in $\sigma$, we don't need twinning because no kind ever contains $s$, so
we can't actually even make use of it even on accident. \\

So how does this map to SML modules? \\
\begin{lstlisting}
sig
  type t
  val f : t
end

struct
  type t = int
  val f = 12
end
\end{lstlisting}
The signature translates to $\Sigma\bind{\alpha\cancel{/s} \of \satom{\type}}{\datom{\ext{s}}}$ \\
The structure translates to $\pair{\satom{\intt}, \datom{12}}$ \\

\begin{enumerate}
\item Phase distinction: static vs. dynamic \\
Type Checking relies only on static components \\
so: static can never depend on dynamic \\
Having a 2nd class module system solves this problem. \\
If we had a 1st class module system, we need to try very hard to preserve
phase distinction. \\

\item Static Projection
\begin{mathpar}
\inferr{\Gamma \vd \Fst{M} \gg c}{}

\inferr{\Gamma \vd \Fst{\ast} \gg \ast}{}

\inferr{\Gamma \vd \Fst{\satom{c}} \gg c}{}

\inferr{\Gamma \vd \Fst{\datom{c}} \gg \ast}{}

\inferr{\Gamma \vd \Fst{\Ap{M_1}{M_2}} \gg \ap{c_1}{c_2}}
       {\Gamma \vd \Fst{M_1} \gg c_1 \\ \Gamma \vd \Fst{M_2} \gg c_2 \\ }

\inferr{\Gamma \vd \Fst{\pair{M_1, M_2}} \gg \pair{c_1, c_2}}
       {\Gamma \vd \Fst{M_1} \gg c_1 \\ \Gamma \vd \Fst{M_2} \gg c_2}

\inferr{\Gamma \vd \Fst{\pi_i M} \gg \pi_i c}{\Gamma \vd \Fst{M} \gg c}

\inferr{\Gamma \vd \Fst{s} \gg \alpha}{\alpha/s \of \sigma \in \Gamma}

\inferr{\Gamma \vd \Fst{\lambda^\texttt{ap}\bind{s \of \sigma}{M}} \gg
          \lambda\bind{\alpha \of \Fst{\sigma}}{c}}
       {\Gamma, \alpha/s \of \sigma \vd \Fst{M} \gg c}
\end{mathpar}

\item Twinning
\item Sealing is an effect
\end{enumerate}

\begin{align*}
\Fst{1} &= 1 \\
\Fst{\satom{k}} &= k \\
\Fst{\datom{k}} &= 1 \\
\Fst{\api\bind{\alpha/s \of \sigma_1}{\sigma_2}} &=
  \Pi\bind{\alpha \of \Fst{\sigma_1}}{\Fst{\sigma_2}} \\
\Fst{\Sigma\bind{\alpha/s \of \sigma_1}{\sigma_2}} &=
  \Sigma\bind{\alpha \of \Fst{\sigma_1}}{\Fst{\sigma_2}} \\
\end{align*}


\subsection{$\Gamma \vd s \of \sigma$}
\begin{mathpar}
\cancel{\inferr{\Gamma \vd \alpha \of k}{\Gamma(\alpha) = k}}

\inferr{\Gamma \vd \alpha \of k}{\alpha \of k \in \Gamma}

\inferr{\Gamma \vd \alpha \of \Fst{\sigma}}{\alpha/s \of \sigma \in \Gamma}
\end{mathpar}

\begin{mathpar}
\inferr{\Gamma \vd s \of \sigma}{\alpha/s \of \sigma \in \Gamma}

\inferr{\Gamma \vd \ast \of 1}{\strut}

\inferr{\Gamma \vd \satom{c} \of \satom{k}}{\Gamma \vd c \of k}

\inferr{\Gamma \vd \datom{e} \of \datom{\tau}}{\Gamma \vd e \of \tau}

\inferr{\Gamma \vd \alam{\alpha/s \of \sigma_1}{M} \of \api\bind{\alpha \of \sigma_1}{\sigma_2}}
       {\Gamma \vd \sigma_1 \of \sig \\
        \Gamma, \alpha/s \of \sigma_1 \vd M \of \sigma_2}

\inferr{\Gamma \vd \Ap{M_1}{M_2} \of \subst{c_2}{\alpha}{\sigma_2}}
       {\Gamma \vd M_1 \of \api\bind{\alpha \of \sigma_1}{\sigma_2} \\
        \Gamma \vd M_2 \of \sigma_1 \\
        \Gamma \vd \Fst{M_2} \gg c_2}

\inferr{\Gamma \vd \pair{M_1, M_2} \of \sigma_1 \times \sigma_2}
       {\Gamma \vd M_1 \of \sigma_2 \\ \Gamma \vd M_2 \of \sigma_2}

\inferr{\Gamma \vd \pi_1 M \of \sigma_1}
       {\Gamma \vd M \of \Sigma\bind{\alpha \of \sigma_1}{\sigma_2}}

\inferr{\Gamma \vd \pi_2 M \of \subst{\pi_1 c}{\alpha}{\sigma_2}}
       {\Gamma \vd M \of \Sigma\bind{\alpha \of \sigma_1}{\sigma_2} \\
        \Gamma \vd \Fst{M} \gg c}

\inferr{\Gamma \vd M \of \sigma'}
       {\Gamma \vd M \of \sigma \\ \Gamma \vd \sigma \le \sigma'}
\end{mathpar}

\subsection{$\Gamma \vd s \of \sig$, $\Gamma \vd \sigma \le \sigma $}
\begin{mathpar}
\inferr{\Gamma \vd \sigma \le \sigma}{\Gamma \vd \sigma \of \sig}

\inferr{\Gamma \vd \sigma_1 \le \sigma_3}
       {\Gamma \vd \sigma_1 \le \sigma_2 \\ \Gamma \vd \sigma_2 \le \sigma_3}

\inferr{\Gamma \vd \satom{k} \le \satom{k'}}{\Gamma \vd k \le k'}

\cancel{\inferr{\Gamma \vd \datom{\tau} \le \datom{\tau'}}
       {\Gamma \vd \tau \equiv \tau' \of \type}}

\inferr{\Gamma \vd \api\bind{\alpha \of \sigma_1}{\sigma_2} \le
          \api\bind{\alpha \of \sigma_1'}{\sigma_2'}}
       {\Gamma \vd \sigma_1' \le \sigma_1 \\
        \Gamma, \alpha \of \Fst{\sigma_1'} \vd \sigma_2 \le \sigma_2' \\
        \Gamma, \alpha \of \Fst{\sigma_1} \vd \sigma_2 \of \sig}

\inferr{\Gamma \vd \Sigma\bind{\alpha \of \sigma_1}{\sigma_2} \le
          \Sigma\bind{\alpha \of \sigma_1'}{\sigma_2'}}
       {\Gamma \vd \sigma_1' \le \sigma_1 \\
        \Gamma, \alpha \of \Fst{\sigma_1} \vd \sigma_2 \le \sigma_2' \\
        \Gamma, \alpha \of \Fst{\sigma_1'} \vd \sigma_2' \of \sig}
\end{mathpar}

\begin{mathpar}
\inferr{\Gamma \vd 1 \of \sig}{\strut}

\inferr{\Gamma \vd \satom{k} \of \sig}{\Gamma \vd k \of \kind}

\inferr{\Gamma \vd \datom{\tau} \of \sig}{\Gamma \vd \tau \of \type}

\inferr{\Gamma \vd \api\bind{\alpha \of \sigma_1}{\sigma_2} \of \sig}
       {\Gamma \vd \sigma \of \sig \\
        \Gamma, \alpha \of \Fst{\sigma_1} \vd \sigma_2 \of \sig}
\end{mathpar}
(Last rule also works for $\Pi^\texttt{gen}$ and $\Sigma$ \\

\begin{mathpar}
\inferr{\Gamma \vd \sigma \equiv \sigma \of \sig}
       {\Gamma \vd \sigma \of \sig}

\inferr{\Gamma \vd \sigma_2 \equiv \sigma_1 \of \sig}
       {\Gamma \vd \sigma_1 \equiv \sigma_2 \of \sig}

\inferr{\Gamma \vd \sigma_1 \equiv \sigma_3 \of \sig}
       {\Gamma \vd \sigma_1 \equiv \sigma_2 \of \sig \\
        \Gamma \vd \sigma_2 \equiv \sigma_3 \of \sig}

\inferr{\Gamma \vd \satom{k} \equiv \satom{k'} \of \sig}
       {\Gamma \vd k \equiv k' \of \kind}

\inferr{\Gamma \vd \datom{\tau} \equiv \datom{\tau'} \of \sig}
       {\Gamma \vd \tau \equiv \tau' \of \type}

\inferr{\Gamma \vd \api\bind{\alpha \of \sigma_1}{\sigma_2} \equiv
          \api\bind{\alpha \of \sigma_1'}{\sigma_2'} \of \sig}
       {\Gamma \vd \sigma_1 \equiv \sigma_1' \of \sig \\
        \Gamma, \alpha \of \Fst{\sigma_1} \vd \sigma_2 \equiv \sigma_2' \of \sig}
\end{mathpar}
(Last rule also works for $\Pi^\texttt{gen}$ and $\Sigma$ \\


% TODO:
If we have $\Gamma \vd M \of \sigma \tto [c,e]$, we can split it up into its
static portion and the rest of it, ``phase separation''. More on this
next time.


% examples?
Consider
\begin{lstlisting}
$\sigma$ =
  sig
    type t
    val x : t
    val f : t $\to$ t
    val g : t $\to$ bool
  end

$M_1$ =
  struct
    type t = bool
    val x = true
    val f = not
    val g = $\lambda$x. x
  end

$M_2$ =
  struct
    type t = int
    val x = 0
    val f = $\lambda$x. x + 1
    val g = even?
  end
\end{lstlisting}

$M_1 \of \sigma$, $M_2 \of \sigma$ \\
note that the two $t$s are different unless they are sealed, eg: \\
$\seal{M_1}{\sigma}$, $\seal{M_2}{\sigma}$ \\

We don't want to even be able to ask questions about the equivalence of
the internal types ($t$) after being sealed. \\
We will call $\seal{M_1}{\sigma}$ ``indeterminate''. \\

Going back to one of the old judgements,
$\Gamma \vd \Fst{M} \gg c$ only applies when $M$ is ``determinate''. \\


Now, consider \\
$F \of \sigma_1 \arrow \sigma_2$ \\ % TODO: overset the arrow with \texttt{gen}
$M \of \sigma_1$ \\
$\ap{F}{M} \of \sigma_2$ \\

Typesystem has to track whether or not a module is pure. (Pure in this sense
meaning determinate meaning unsealed.) We treat sealing as an effect. \\ % TODO: see pfpl

Thus, we use judgement assigning with the purity class $\kappa$: 
$\Gamma \vd_\kappa M \of \sigma$, with $\kappa \bnfdef P \bnfalt I$ \\

\subsection{$\Gamma \vd e \of \tau$}
Only new rule we need:
\begin{mathpar}
\inferr{\Gamma \vd \Ext{M} \of \tau}{\Gamma \vd_I M \of \datom{\tau}}
\end{mathpar}

\subsection{$\Gamma \vd_\kappa M \of \sigma$}
\begin{mathpar}
\inferr{\Gamma \vd_P \ast \of 1}{\strut}

\inferr{\Gamma \vd_P s \of \sigma}{\alpha/s \of \sigma \in \Gamma}

\inferr{\Gamma \vd_P \satom{c} \of \satom{k}}{\Gamma \vd c \of k}

\inferr{\Gamma \vd_P \datom{e} \of \datom{\tau}}{\Gamma \vd e \of \tau}

\inferr{\Gamma \vd_I M \of \sigma}{\Gamma \vd_P M \of \sigma} % the forgetting rule

\inferr{\Gamma \vd_\kappa M \of \sigma'}
       {\Gamma \vd_\kappa M \of \sigma \\
        \Gamma \vd \sigma \le \sigma'}

\inferr{\Gamma \vd_I \seal{M}{\sigma} \of \sigma}{\Gamma \vd_I M \of \sigma}
  % here, we don't actually need \vd_I on the top, we can use \kappa, but since
  % we allow the forgetting rule, we can just use that to convert back as desired

                   % no effects in a lambda
\inferr{\Gamma \vd_P \lambdag{\alpha/s \of \sigma_1}{M} \of \Pig{\alpha \of \sigma_1}{\sigma_2}}
       {\Gamma \vd \sigma_1 \of \sig \\
        \Gamma, \alpha/s \of \sigma_1 \vd_I M \of \sigma_2}

\inferr{\Gamma \vd_I \ap{M_1}{M_2} \of \subst{c_2}{\alpha}{\sigma'}}
       {\Gamma \vd_I M_1 \of \Pig{\alpha \of \sigma}{\sigma'} \\
        \Gamma \vd_P M_2 \of \sigma \\ % need to be able to compute static part of M_2, so pure
        \Gamma \vd \Fst{M_2} \gg c_2}

\inferr{\Gamma \vd_P \lambdaa{\alpha/s \of \sigma_1}{M} \of \Pia{\alpha \of \sigma_1}{\sigma_2}}
       {\Gamma \vd \sigma_1 \of \sig \\
        \Gamma, \alpha/s \of \sigma_1 \vd_P M \of \sigma_2}

\inferr{\Gamma \vd_\kappa \Ap{M_1}{M_2} \of \subst{c_2}{\alpha}{\sigma'}}
       {\Gamma \vd_\kappa M_1 \of \Pia{\alpha \of \sigma}{\sigma'} \\
        \Gamma \vd_P M_2 \of \sigma \\
        \Gamma \vd \Fst{M_2} \gg c_2}

% for this, we ensure that both have the same purity class because if either is
% impure, we need the result to be impure (see: forgetting rule)
\inferr{\Gamma \vd_\kappa \pair{M_1, M_2} \of \sigma_1 \times \sigma_2}
       {\Gamma \vd_\kappa M_1 \of \sigma_1 \\ \Gamma \vd_\kappa M_2 \of \sigma_2}

\inferr{\Gamma \vd_P \pi_1 M \of \sigma_1} % we don't NEED purity here, but just for uniformity
       {\Gamma \vd_P M \of \Sigma\bind{\alpha \of \sigma_1}{\sigma_2}}

\inferr{\Gamma \vd_P \pi_2 M \of \subst{c}{\alpha}{\sigma_1}}
       {\Gamma \vd_P M \of \Sigma\bind{\alpha \of \sigma_1}{\sigma_2} \\
        \Gamma \vd \Fst{M} \gg c}
\end{mathpar}

\subsection{$\Gamma \vd \Fst{M} \gg k$}
\begin{flalign*}
\Fst{1} &= 1 &\\
\Fst{\satom{k}} &= k &\\
\Fst{\datom{\tau}} &= 1 &\\
\Fst{\Pia{\alpha \of \sigma_1}{\sigma_2}} &=
  \Pi\bind{\alpha \of \Fst{\sigma_1}}{\Fst{\sigma_2}} &\\
\Fst{\Pig{\alpha \of \sigma_1}{\sigma_2}} &= 1 &\\
\Fst{\Sigma\bind{\alpha \of \sigma_1}{\sigma_2}} &=
  \Sigma\bind{\alpha \of \Fst{\sigma_1}}{\Fst{\sigma_2}} &\\
\end{flalign*}

\subsection{$\Gamma \vd \Fst{M} \gg c$}
\begin{mathpar}
\inferr{\Gamma \vd \Fst{s} \gg \alpha}{\alpha/s \of \sigma \in \Gamma}

\inferr{\Gamma \vd \Fst{\ast} \gg \ast}{\strut}

\inferr{\Gamma \vd \Fst{\satom{c}} \gg c}{\strut}

\inferr{\Gamma \vd \Fst{\datom{e}} \gg \ast}{\strut}

\inferr{\Gamma \vd \Fst{\lambdaa{\alpha/s \of \sigma_1}}
                       {\lambda\bind{\alpha \of \Fst{\sigma_1}}{c}}}
       {\Gamma \vd \alpha/s \of \sigma_1 \vd \Fst{M} \gg c}

\inferr{\Gamma \vd \Fst{\Ap{M_1}{M_2}} \gg \ap{c_1}{c_2}}
       {\Gamma \vd \Fst{M_1} \gg c_1 \\
        \Gamma \vd \Fst{M_2} \gg c_2}

\inferr{\Gamma \vd \Fst{\lambdag{\alpha/s \of \sigma_1}{M}} \gg \ast}{\strut}

\cancel{\inferr{\Gamma \vd \Fst{\ap{M_1}{M_2}} \gg }{\strut}}

\inferr{\Gamma \vd \Fst{\pair{M_1, M_2}} \gg \pair{c_1, c_2}}
       {\Gamma \vd \Fst{M_1} \gg c_1 \\
        \Gamma \vd \Fst{M_2} \gg c_2}

\inferr{\Gamma \vd \Fst{\pi_1 M} \gg \pi_i c}{\Gamma \vd \Fst{M} \gg c}
\end{mathpar}


\normalsize
Need to support some sort of {\tt let} binding in the language in order for
us to do anything with sealed modules. So let's change the language as such: \\
$M \bnfdef \dots \bnfalt \letbind{\alpha/s}{M}{M}$ \\

with typing rule
\begin{mathpar}
\inferr{\Gamma \vd_I \letbind{\alpha/s}{M_1}{M_2}}
       {\Gamma \vd_I M_1 \of \sigma_1 \\
        \Gamma, \alpha/s \of \sigma_1 \vd_I M_2 \of \sigma_2 \\
        \Gamma \vd \sigma_2 \of \sig}

\inferr{\Gamma \vd \letbind{\alpha/s}{M_1}{M_2} \Rightarrow }
       {\Gamma \vd M_1 \Rightarrow \sigma_1 \\
        \Gamma, \alpha/s \of \sigma_1 \vd M_2 \Rightarrow \sigma_2}
\end{mathpar}
Big idea \#5: ``Avoidance Problem''. \\
There is no ``best'' signature for this last rule above. \\
Problem: \\
Given $\Gamma, \alpha \of k \vd \sigma \of \sig$ \\
obtain $\sigma'$ s.t. \\
1. $\Gamma \vd \sigma' \of \sig$ \\
2. $\Gamma, \alpha \of k \vd \sigma \le \sigma'$ \\
3. forall $\sigma''$ if $\Gamma \vd \sigma'' \of \sig$ and
    $\Gamma, \alpha \of k \vd \sigma \le \sigma''$ then $\Gamma \vd \sigma' \le \sigma''$ \\
But this problem has no solution. \\

Example: $\alpha \of \type \vd (\type \arrow \singleton{\alpha}) \times \singleton{\alpha} \of \kind$ \\
This is a sub-kind of $\sigma\bind{\beta \of \type \arrow \type}{\singleton{\beta \intt}}$ \\
And also a sub-kind of $\sigma\bind{\beta \of \type \arrow \type}{\singleton{\beta \stringt}}$ \\
But these two that we just generated are not equivalent. \\

so instead, we ask the programmer for the $\sigma_2$ \\
$M \bnfdef \dots \bnfalt \letbind{\alpha/s}{M}{M \of \sigma}$ \\
\begin{mathpar}
\inferr{\Gamma \vd \letbind{\alpha/s}{M_1}{M_2 \of \sigma_2} \synthesis \sigma_2 }
       {\Gamma \vd M_1 \synthesis \sigma_1 \\
        \Gamma \vd \sigma \checking \sig \\
        \Gamma, \alpha/s \of \sigma_1 \vd M_2 \checking \sigma_2}
\end{mathpar}

But there is an issue we cannot deal with in this language. When we parse a
structure into a tuple, we would have a bunch of $\of \sigma$ in the nested
lets (multiple structures one after another in a structure). While this is not
incorrect, it's grossly inefficient. \\
So instead, we add one more feature: \\
$M \bnfdef \dots \bnfalt \letbind{\alpha/s}{M}{M \of \sigma} \bnfalt \pair{\alpha/s = M, M}$ \\
\begin{mathpar}
\inferr{\Gamma \vd_\kappa \pair{\alpha/s = M_1, M_2} \of
          \Sigma\bind{\alpha \of \sigma_1}{\sigma_2}}
       {\Gamma \vd_\kappa M_1 \of \sigma_1 \\
        \Gamma, \alpha/s \of \sigma_1 \vd_\kappa M_2 \of \sigma_2}
\end{mathpar}
This is somewhat motivated by the Avoidance Problem.

% TODO: what is this motivated by
Add one more feature: \\
$M \bnfdef \dots \bnfalt \letp{\alpha/s}{M}{M}$ \\
\begin{mathpar}
\inferr{\Gamma \vd_\kappa \letp{\alpha/s}{M_1}{M_2} \of \subst{c_1}{\alpha}{\sigma_2}}
       {\Gamma \vd_P M_1 \of \sigma_1 \\
        \Gamma, \alpha/s \of \sigma_1 \vd_\kappa M_2 \of \sigma_2 \\
        \Gamma \vd \Fst{M_1} \gg c_1}
\end{mathpar}

Note, in the first rule in the next section below, we find that we can't just
use $\sigma$ because it's not the ``best'' signature. So we need to introduce
a new $\singleton{\sigma}$ \\
Kind level: \\
\begin{flalign*}
\singleton{c \of \type} &= \singleton{c} &\\
\singleton{c \of \Pi\bind{\alpha \of k_1}{k_2}} &= \Pi\bind{\alpha \of k_1}{\singleton{\ap{c}{\alpha} \of k_2}} &\\
\singleton{c \of \Sigma\bind{\alpha \of k_1}{k_2}} &= \singleton{\pi_1 c \of k_1} \times \singleton{\pi_2 c \of \subst{\pi_1 c}{\alpha}{k_2}} &\\
\singleton{c \of \singleton{c'}} &= \singleton{c} &\\
\end{flalign*}
What we think the property should look like: \\
``$c' \of \singleton{c \of k}$ iff $c' \of k$ and $c \equiv c' \of k$''
``$ \vd_P M \of \singleton{c \of \sigma}$ iff $M \of \sigma$
  and $\Fst{M} \equiv c \of \Fst{\sigma}$'' % TODO
Module level: \\
\begin{flalign*}
\singleton{c \of 1} &= 1 &\\
\singleton{c \of \satom{k}} &= \satom{\singleton{c \of k}} &\\
\singleton{c \of \datom{\tau}} &= \datom{\tau} &\\
\singleton{c \of \Pig{\alpha \of \sigma_1}{\sigma_2}} &= \Pig{\alpha \of \sigma_1}{\sigma_2} &\\
\singleton{c \of \Pia{\alpha \of \sigma_1}{\sigma_2}} &=
  \Pia{\alpha \of \sigma_1}{\singleton{\ap{c}{\alpha} \of \sigma_2}} &\\
\singleton{c \of \Sigma\bind{\alpha \of \sigma_1}{\sigma_2}} &=
  \singleton{\pi_1 c \of \sigma_1} \times
  \singleton{\pi_2 c \of \subst{\pi_1 c}{\alpha}{\sigma_2}} &\\
\end{flalign*}

% TODO
NOTE selfification: when you take the signature of a module and write it back
into the module. \\

So now, we want the property to be: \\
If $\Gamma \vd_P M \of \sigma$ and $\Gamma \vd \Fst{M} \gg c$
  then $\Gamma \vd_P M \of \singleton{c \of \sigma}$ \\

But as it is right now, it's not actually true. So like back in the singleton
calculus, we need to add in the extentionality rules: \\

\subsection{$\overset{+}{\Gamma} \vd_{\overset{-}{\kappa}} \overset{+}{M}
               \synthesis \overset{-}{\sigma}$}
\begin{mathpar}
\inferr{\Gamma \vd_P s \synthesis \singleton{\alpha \of \sigma}}
       {\alpha/s \of \sigma \in \Gamma}
\end{mathpar}


\newpage
\subsection{Extentionality Rules} % TODO title
Suppose we had some \\
$F \of \Pia{\alpha \of \sigma_1}{\sigma_2}$ where \\
$\Fst{F} \gg \varphi$ \\

Then, $F \of \singleton{\varphi \of \Pia{\alpha \of \sigma_1}{\sigma_2}}$
should be true, but we have no way of proving it right now.\\

\begin{mathpar}
\inferr{\Gamma \vd_P M \of \Pia{\alpha \of \sigma_1}{\sigma_2}}
       {\Gamma \vd_P M \of \Pia{\alpha \of \sigma_1}{\sigma_2'} \\
        \Gamma, \alpha \of \sigma_1 \vd_P \Ap{M}{\alpha} \of \sigma_2}

\inferr{\Gamma \vd M \of \sigma\bind{\alpha \of \sigma_1}{\sigma_2}}
       {\Gamma \vd \pi_1 M \of \sigma_1 \\
        \Gamma \vd \Fst{M} \gg c \\
        \Gamma \vd \pi_2 M \of \subst{\pi_1 c}{\alpha}{\sigma_2} \\
        \Gamma, \alpha \of \Fst{\sigma_1} \vd \sigma_2 \of \sig}

\inferr{\Gamma \vd f \of \singleton{\varphi \of \Pia{\alpha \of \sigma_1}{\sigma_2}}}
       {f \of \Pia{\alpha \of \sigma_1}{\sigma_2} \\
        \alpha/s \of \sigma_1 \vd \ap{f}{s} \of \singleton{\ap{\varphi}{\alpha} \of \sigma_2}}
\end{mathpar}
(Note in the last one, $\Pia{\alpha \of \sigma_1}{\sigma_2}$ is equivalent to
$\Pia{\alpha \of \sigma_1}{\singleton{\ap{\varphi}{\alpha} \of \sigma_2}}$.) \\

Also: % TODO
\begin{mathpar}
\inferr{\Gamma \vd_P M \of \satom{k}}
       {\Gamma \vd_P M \of \satom{k'} \\
        \Gamma \vd \Fst{M} \gg c \\
        \Gamma \vd c \of k}
\end{mathpar}

% TODO
Selfification property: \\
If $\vd \Gamma \ok$, $\Gamma \vd M \of \sigma$, $\Gamma \vd \Fst{M} \gg c$ \\
then $\Gamma \vd M \of \singleton{c \of \sigma}$.


\subsection{$\Gamma \vd M \synthesis \sigma$}
\begin{mathpar}
\inferr{\Gamma \vd_P s \synthesis \singleton{\alpha \of \sigma}}
       {\alpha/s \of \sigma \in \Gamma}

\inferr{\Gamma \vd_P \satom{c} \synthesis \satom{k}}{\Gamma \vd c \synthesis k}

\inferr{\Gamma \vd_P \datom{e} \synthesis \datom{\tau}}{\Gamma \vd e \synthesis \tau}

\inferr{\Gamma \vd_P \lambdag{\alpha/s \of \sigma_1}{M} \synthesis
          \Pig{\alpha \of \sigma_1}{\sigma_2}}
       {\Gamma \vd \sigma_1 \checking \sig \\
        \Gamma, \alpha/s \of \sigma_1 \vd_\kappa M \synthesis \sigma_2}

\inferr{\Gamma \vd_I \ap{M_1}{M_2} \synthesis \subst{c_2}{\alpha}{\sigma'}}
       {\Gamma \vd_\kappa M_1 \synthesis \Pig{\alpha \of \sigma}{\sigma'} \\
        \Gamma \vd_P M_2 \checking \sigma \\
        \Gamma \vd \Fst{M_2} \gg c_2}

\inferr{\Gamma \vd_P \lambdaa{\alpha/s \of \sigma_1}{M} \synthesis
          \Pia{\alpha \of \sigma_1}{\sigma_2}}
       {\Gamma \vd \sigma_1 \checking \sig \\
        \Gamma, \alpha/s \of \sigma_1 \vd_P M \synthesis \sigma_2}

\inferr{\Gamma \vd \Ap{M_1}{M_2} \synthesis \subst{c_2}{\alpha}{\sigma'}}
       {\Gamma \vd_\kappa M_1 \synthesis \Pia{\alpha \of \sigma}{\sigma'} \\
        \Gamma \vd_P M_2 \checking \sigma \\
        \Gamma \vd \Fst{M_2} \gg c_2}

% TODO: wait, so why did we not want to just call them all kappa? cup is the higher of the two
% we always can forget right...?
\inferr{\Gamma \vd_{\kappa_1 \cup \kappa_2} \pair{\alpha/s = M_1, M_2} \synthesis
          \Sigma\bind{\alpha \of \sigma_1}{\sigma_2}}
       {\Gamma \vd_{\kappa_1} M_1 \synthesis \sigma_1 \\
        \Gamma, \alpha/s \of \sigma_1 \vd_{\kappa_2} M_2 \synthesis \sigma_2}

\inferr{\Gamma \vd_P \pi_1 M \synthesis \sigma_1}
       {\Gamma \vd_P M \synthesis \Sigma\bind{\alpha \of \sigma_1}{\sigma_2}}

\inferr{\Gamma \vd_P \pi_2 M \synthesis \subst{\pi_1 c}{\alpha}{\sigma_2}}
       {\Gamma \vd_P M \synthesis \Sigma\bind{\alpha \of \sigma_1}{\sigma_2} \\
        \Gamma \vd \Fst{M} \gg c}

\inferr{\Gamma \vd_I \seal{M}{\sigma} \synthesis \sigma}
       {\Gamma \vd \sigma \checking \sig \\
        \Gamma \vd_\kappa M \checking \sigma}

\inferr{\Gamma \vd_I \letbind{\alpha/s}{M_1}{M_2} \of \sigma_2 \synthesis \sigma_2}
       {\Gamma \vd_{\kappa_1} M_1 \synthesis \sigma_1 \\
        \Gamma \vd \sigma_2 \checking \sig \\
        \Gamma, \alpha/s \of \sigma_1 \vd_{\kappa_2} M_2 \checking \sigma_2}

\inferr{\Gamma \vd \letp{\alpha/s}{M_1}{M_2} \synthesis \subst{c_1}{\alpha}{\sigma_2}}
       {\Gamma \vd_P M_1 \synthesis \sigma_1 \\
        \Gamma \vd \Fst{M_1} \gg c_1 \\
        \Gamma, \alpha/s \of \sigma_1 \vd_\kappa M_2 \synthesis \sigma_2}
\end{mathpar}

\subsection{$\Gamma \vd M \checking \sigma$}
\begin{mathpar}
\inferr{\Gamma \vd_\kappa M \checking \sigma}
       {\Gamma \vd_\kappa M \synthesis \sigma' \\
        \Gamma \vd \sigma' \unlhd \sigma} % TODO
\end{mathpar}

\subsection{$\Gamma \vd e \of \tau$}
\begin{mathpar}
\inferr{\Gamma \vd \Ext{M} \synthesis \tau}
       {\Gamma_\kappa M \synthesis \datom{\tau}}
\end{mathpar}

\subsection{Closing}
We skipped $\Gamma \vd \sigma_1 \unlhd \sigma_2$ and
$\Gamma \vd \sigma \checking \sig$ because they are fairly straightforward.

% NOTE: purity class is an output, can make input if you really want
% Coding note:
%   Everything you return should be some result sum type with:
%     Pure of sig * con
%     Impure of sig
%   where in the pure case, we also pair the signature with its Fst

\newpage
\section{Phase Splitting}
With phase distinction, we separate out the static (Fst) and dynamic components. \\
Target language is IL-Direct, which we already discussed in great detail. % TODO: find it

Judgements: \\
$\sigma \leadsto [\bind{\alpha \of k}{\tau}]$ \\
$\Gamma \vd_P M \of \sigma \leadsto [c, e]$ \\
$\Gamma \vd_\mp M \of \sigma \leadsto e$ \\

% TODO Sanity case
If $\Gamma \vd \sigma \of \sig$, $\sigma \leadsto [\bind{\alpha \of k}{\tau}]$ \\
then $\bar\Gamma \vd k \of \kind$, $\bar\Gamma, \alpha \of k \vd \tau \of \type$. \\
If $\sigma \leadsto [\bind{\alpha \of k}{\tau}]$ then $\Fst{\sigma} = k$. \\

If $\sigma \leadsto [\bind{\alpha \of k}{\tau}]$, $\Gamma \vd_P M \of \sigma \leadsto [c, e]$ \\
then $\bar\Gamma \vd c \of k$, $\bar\Gamma \vd e \of \subst{c}{\alpha}{\tau}$. \\

If $\sigma \leadsto [\bind{\alpha \of k}{\tau}]$, $\Gamma \vd_I M \of \sigma \leadsto e$ \\
then $\bar\Gamma \vd e \of \exists\bind{\alpha \of k}{\tau}$. \\

\subsection{$\sigma \leadsto [\bind{\alpha \of k}{\tau}]$}
\begin{mathpar}
\inferr{1 \leadsto \sd{\alpha \of 1}{\unit}}{\strut}

\inferr{\satom{k} \leadsto \sd{\alpha \of k}{\unit}}{\strut}

% NOTE: don't forget to lift the \tau because it previous was not in a binding
\inferr{\datom{\tau} \leadsto \sd{\alpha \of 1}{\tau}}{\strut}

\inferr{\Pia{\alpha \of \sigma_1}{\sigma_2} \leadsto
          \sd{\beta \of \Pi\bind{\alpha \of k_1}{k_2}}
            {\forall\bind{\alpha \of k_1}
              {\subst{\alpha}{\alpha_1}{\tau_1} \arrow
               \subst{\ap{\beta}{\alpha}}{\alpha_2}{\tau_2}}}}
       {\sigma_1 \leadsto \sd{\alpha_1 \of k_1}{\tau_1} \\
        \sigma_2 \leadsto \sd{\alpha_2 \of k_2}{\tau_2}}

\inferr{\Pig{\alpha \of \sigma_1}{\sigma_2} \leadsto \sd{\beta \of 1}{
          \forall\bind{\alpha \of k_1}{\subst{\alpha}{\alpha_1}{\tau_1} \arrow
            \exists\bind{\alpha \of k_2}{\tau_2}}}}
       {\sigma_1 \leadsto \sd{\alpha_1 \of k_1}{\tau_1} \\
        \sigma_2 \leadsto \sd{\alpha_2 \of k_2}{\tau_2}}

\inferr{\Sigma\bind{\alpha \of \sigma_1}{\sigma_2} \leadsto
          \sd{\beta \of
            \Sigma\bind{\alpha \of k_1}{k_2}
          }{
            \subst{\pi_1 \beta}{\alpha_1}{\tau_1} \times
            \subst{
              \pi_1 \beta, \pi_2 \beta
            }{\alpha,\alpha_2}{\tau_2}
          }}
       {\sigma_1 \leadsto \sd{\alpha_1 \of k_1}{\tau_1} \\
        \sigma_2 \leadsto \sd{\alpha_2 \of k_2}{\tau_2}}
\end{mathpar}


\subsection{$\sigma \splitsto \sd{k}{\tau}$}
Let's revisit all of the above rules in the Debruijn world.

% TODO: motivation?
\begin{mathpar}
\inferr{\Gamma \vd \bind{m}{s} \of \Gamma', A}
       {\Gamma \vd s \of \Gamma' \\ \Gamma \vd m \of A[s]}

\inferr{\Gamma, \Gamma' \vd \lift k \of \Gamma}
       {|\Gamma'| = k}
\end{mathpar}

If $\Gamma \vd \sigma \of \sig$ and $\sigma \splitsto \sd{\alpha \of k}{\tau}$
then $\Gamma \vd k \of \kind$, $\Gamma, \alpha \of k \vd \tau \of \type$. \\

Now the Debruijn form: \\
If $\Gamma \vd \sigma \of \sig$ and $\sigma \splitsto \sd{k}{\tau}$
then $\Gamma \vd k \of \kind$ and $\Gamma, k \vd \tau \of \type$.

\begin{mathpar}
  \inferr{1 \splitsto \sd{1}{\unit}}{\strut}

  \inferr{\satom{k} \splitsto \sd{k}{\unit}}{\strut}

  \inferr{\datom{\tau} \splitsto \sd{1}{\tau[\lift]}}{\strut}

  % See NOTE 0, 1, 2 for info on how we got this
  \inferr{\Pia{\sigma_1}{\sigma_2} \splitsto
            \sd{\Pi\bind{k_1}{k_2}}
              {\forall\bind{k_1[\lift]}
                {\tau_1[\bind{0}{\lift^2}] \arrow
                 \tau_2[\bind{\ap{1}{0}}{\bind{0}{\lift^2}}]}}}
         {\sigma_1 \splitsto \sd{k_1}{\tau_1} \\
          \sigma_2 \splitsto \sd{k_2}{\tau_2}}

  % NOTE 0 These are the derivations on how we got the case for Pi app
  %\inferr{\Gamma, A \vd 0 \of A[\lift]}{\strut}
  %\inferr{\Gamma \vd i \of A[\lift^{i + 1}]}{\Gamma(i) = A}
  %\inferr{\Gamma, \Pi\bind{k_1}{k_2}, k_1[\lift] \vd \bind{0}{\lift^2} \of \Gamma, k_1}
  %       {\Gamma, \Pi\bind{k_1}{k_2}, k_1[\lift] \vd 0 \of k_1[\lift^2] \\
  %        \Gamma, \Pi\bind{k_1}{k_2}, k_1[\lift] \vd \lift^2 \of \Gamma} \\
  %\inferr{\Gamma, \Pi\bind{k_1}{k_2}, k_1[\lift] \vd
  %          \bind{\ap{1}{0}}{\bind{0}{\lift^2}} \of \Gamma, k_1, k_2}
  %       {\inferr{\Gamma, \Pi\bind{k_1}{k_2}, k_1[\lift] \vd
  %                  \ap{1}{0} \of k_2[\bind{0}{\lift^2}]}
  %               {\dots \vd 1 \of (\Pi\bind{k_1}{k_2})[\lift^2] \\
  %                \dots \vd 0 \of k_1[\lift^2]}
  %        \\ \dots } \\

  \inferr{\Pig{\of \sigma_1}{\sigma_2} \splitsto \sd{1}{
              \forall\bind{k_1[\lift]}{\tau_1[0.\lift^2] \arrow
              \exists\bind{k_2[0.\lift^2]}{\tau_2[0.1.\lift^3]}}}}
         {\sigma_1 \splitsto \sd{k_1}{\tau_1} \\
          \sigma_2 \splitsto \sd{k_2}{\tau_2}}

  % NOTE 3 These is one of the derivations for Pi gen
  %\inferr{\Gamma, 1, k_1[\lift], k_2[0.\lift^2] \vd 0.1.\lift^3 \of \Gamma, k_1, k_2}
  %       {\dots \vd 0 \of k_2[1.\lift^3] \\
  %        \inferr{\dots \vd 1.\lift^3 \of \Gamma, k_1}
  %               {\dots \vd 1 \of k_1[\lift^3] \\
  %                \dots \vd \lift^3 \of \Gamma}}

  \inferr{\Sigma\bind{\sigma_1}{\sigma_2} \splitsto
            \sd{
              \Sigma\bind{k_1}{k_2}
            }{
              \tau_1[\pi_1 0. \lift] \times
              \tau_2[\pi_2 0. \pi_1 0. \lift]
            }
         }
         {\sigma_1 \splitsto \sd{k_1}{\tau_1} \\
          \sigma_2 \splitsto \sd{k_2}{\tau_2}}

  % NOTE 4 These are the derivations for sigma
  %\inferrule{
  %  \inferr{\Gamma, \Sigma\bind{k_1}{k_2} \vd \pi_2 0 \of k_2[\pi_1 0.\lift]}
  %         {\Gamma, \Sigma\bind{k_1}{k_2} \vd 0 \of \Sigma\bind{k_1[\lift]}{k_2[0.\lift^2]}}
  %  \inferr{\Gamma, \Sigma\bind{k_1}{k_2} \vd \pi_1 0. \lift \of \Gamma, k_1}
  %         {\inferr{\Gamma, \Sigma\bind{k_1}{k_2} \vd \pi_1 0 \of k_1[\lift]}
  %                 {\Gamma, \Sigma\bind{k_1}{k_2} \vd 0 \of
  %                    \Sigma\bind{k_1[\lift]}{k_2[0.\lift^2]}} \\
  %          \Gamma, \Sigma\bind{k_1}{k_2} \vd \lift \of \Gamma}
  %}{
  %  \Gamma, \Sigma\bind{k_1}{k_2} \vd \pi_2 0. \pi_1 0. \lift \of \Gamma, k_1, k_2
  %}
\end{mathpar}

%% NOTE 1
%$\Gamma \vd \Pia{\sigma_1}{\sigma_2} \of \sig$ \\
%$\Gamma \vd \sigma_1 \of \sig$ \\
%$\Gamma, k_1 \vd \sigma_2 \of \sig$ \\
%$\Gamma \vd k_1 \of \kind$ \\
%$\Gamma, k_1 \vd \tau_1 \of \type$ \\
%$\Gamma, k_1 \vd k_2 \of \kind$ \\
%$\Gamma, k_1, k_2 \vd \tau_2 \of \type$ \\
%
%where: \\
%$\sigma_1 \splitsto \sd{k_1}{\tau_1}$ \\
%$\sigma_2 \splitsto \sd{k_2}{\tau_2}$ \\
%
%% NOTE 2
%$(\Pi\bind{k_1}{k_2})[\lift^i] = \Pi\bind{k_1[\lift^i]}{k_2[\bind{0}{\lift^{i + 1}}]}$

\subsection{$\target{\Gamma} \tto \Gamma$}
\begin{flalign*}
  \target{\epsilon} &= \epsilon &\\
  \target{\Gamma, \alpha \of k} &= \target{\Gamma}, \alpha \of k &\\
  \target{\Gamma, x \of \tau} &= \target{\Gamma}, x \of \tau &\\
  \target{\Gamma, \alpha/s \of \sigma} &= \target{\Gamma}, \alpha \of k, s \of \tau &\\
\end{flalign*}
*NOTE: in the last one, $\sigma \splitsto \sd{\alpha \of k}{\tau}$

\subsection{$\Gamma \vd_P M \of \sigma \splitsto \sd{c}{e}$}
\begin{mathpar}
  \inferrule{
    \alpha/s \of \sigma \in \Gamma
  }{ % Coding note: algorithmically, we will be inferring \sigma, and it will
     % actually be the best signature (eg: \singleton{\alpha : \sigma})
    \Gamma \vdp s \of \sigma \splitsto \sd{\alpha}{s}
  }

  \axiom{\Gamma \vdp \ast \of 1 \splitsto \sd{\ast}{\pair{}}}

  \inferrule{
    \Gamma \vd c \of k
  }{
    \Gamma \vdp \satom{c} \of \satom{k} \splitsto \sd{c}{\pair{}}
  }

  \inferrule{
    \Gamma \vd e \of \tau
  }{ % note the tau is shifted, but the e is not shifted, care for debruijn
    \Gamma \vd \datom{e} \of \datom{\tau} \splitsto \sd{\ast}{e}
  }

  \inferrule{
    \Gamma \vd \sigma_1 \of \sig \\
    \Gamma, \alpha/s \of \sigma_1 \vdp M \of \sigma_2 \splitsto \sd{c}{e} \\
    \sigma_1 \splitsto \sd{\alpha_1 \of k_1}{\tau_1}
  }{
    \Gamma \vd \lambdaa{\alpha/s \of \sigma_1}{M} \of \Pi\bind{\alpha \of \sigma_1}{\sigma_2}
      \splitsto
    \sd{
      \lambda\bind{\alpha \of k_1}{c}
    }{
      \Lambda\bind{\alpha \of k_1}{
        \lambda\bind{s \of \subst{\alpha}{\alpha_1}{\tau_1}}{
          e
        }
      }
    }
  }
\end{mathpar}


\end{document}
