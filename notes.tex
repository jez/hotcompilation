\documentclass[11pt]{article}

%\usepackage{amsmath}
\usepackage{mathtools}
\usepackage{latexsym}
\usepackage{proof}
\usepackage[margin=1in]{geometry}
\usepackage{mathpartir}
\usepackage{graphicx} % for the nice downarrow
\usepackage{xifthen}

\usepackage{titlesec}
\titleformat*{\section}{\large\bfseries}
\titleformat*{\subsection}{\bfseries}
%\renewcommand{\thesetion}{\arabic{section}}
%\renewcommand{\thesubsection}{(\alpha{subsection})}


\usepackage{listings}
\usepackage{color}
\definecolor{eclipseBlue}{RGB}{42,0.0,255}
\definecolor{eclipseGreen}{RGB}{63,127,95}
\lstset {
  basicstyle=\small\ttfamily,
  captionpos=b,
  tabsize=2,
  columns=fixed,
  breaklines=true,
  mathescape=true,
% frame=l,
% numbers=left,
% numberstyle=\small\ttfamily,
  morekeywords= {
    EQUAL, GREATER, LESS, NONE, SOME, abstraction, abstype, and, andalso, array, as, before, bool, case, char, datatype, do, else, end, eqtype, exception, exn, false, fn, fun, functor, handle, if, in, include, infix, infixr, int, let, list, local, nil, nonfix, not, o, of, op, open, option, orelse, overload, print, raise, real, rec, ref, sharing, sig, signature, string, struct, structure, substring, then, true, type, unit, val, vector, where, while, with, withtype, word
  },
  morestring=[b]",
  morecomment=[s]{(*}{*)},
  stringstyle=\color{black},
  identifierstyle=\color{eclipseBlue},
  keywordstyle=\color{red},
  commentstyle=\color{eclipseGreen}
}

\setlength\parindent{0pt}

% samepage doesn't quite work in some cases, not sure why
% TODO: 16pt is an approximation
\newenvironment{grouped}[1]{\begin{minipage}{\textwidth}#1}{\end{minipage}\vspace{16pt}}

\newcommand{\trans}[1]{$\xrightarrow{\textnormal{#1}}$}
\newcommand{\tab}{\hspace*{2pt}}
\newcommand{\arrow}{\mathbin{\rightarrow}}
\newcommand{\bnfdef}{\mathrel{\Coloneqq}}
\newcommand{\bnfalt}{\mathrel{\mid}}
\newcommand{\note}[1]{{\tiny Note: #1}}
\newcommand{\vd}{\vdash}
\newcommand{\of}{\mathrel{:}}
\newcommand{\bind}[2]{#1\mathrel{.} #2}
\newcommand{\subst}[3]{[#1 \mathbin{/} #2]#3}

\newcommand{\oversetr}[2]{\overset{#2}{#1}}

\newcommand{\ace}{\Leftrightarrow} % Algorithmic Constructor Equivalence
\newcommand{\ape}{\leftrightarrow} % Algorithmic Path Equivalence

\newcommand{\Darrow}{\mathrel{\scalebox{1.2}[1]{$\Downarrow$}}}
\newcommand{\whn}{\Darrow}
\newcommand{\whr}{\mathrel{\leadsto}}
\newcommand{\Uarrow}{\mathrel{\scalebox{1.2}[1]{$\Uparrow$}}}
%\newcommand{\nk}{\mathrel{\Uarrow}}
\newcommand{\nk}{\mathrel{\uparrow}}
\newcommand{\lift}{\mathrel{\uparrow}}
\newcommand{\sk}{\mathrel{\le}} % TODO: triangle less equal

\newcommand{\synthesis}{\Rightarrow}
\newcommand{\checking}{\Leftarrow}

\newcommand{\comp}{\mathrel{\circ}}

\newcommand{\singleton}[1]{\mathop{S}(#1)}

%\newcommand{\test}[3][]{\ifthenelse{\isempty{#1}}{omitted}{given} #2 #3}
\newcommand{\inferr}[3][]{\inferrule*[Right=#1]{#3}{#2}}


\newcommand{\nats}{\mathcal{N}}
\newcommand{\reals}{\mathcal{R}}

\newcommand{\translatesto}{\mathrel{\leadsto}}

\DeclareMathOperator{\kind}{kind}
\DeclareMathOperator{\ok}{ok}
\DeclareMathOperator{\fv}{FV}
\DeclareMathOperator{\id}{id}
\DeclareMathOperator{\type}{Type}
\DeclareMathOperator{\refraw}{ref}
\DeclareMathOperator{\intt}{int}
\newcommand{\sref}[1]{\refraw(#1)}


\title{HOT Compilation Notes}
\author{Rahul Manne\\
{\tt rmanne}@andrew.cmu.edu}
\date{}

\begin{document}
\maketitle

\section*{Disclaimer/README}
These are only reference notes, and by no means fully capture what
is taught in class.\\

Notes for 170131 (on substitution) are extremely incoherent
so I did not include them by default.\\

There may be errors, feel free to report them to me.\\

\section{Compiler Structure}
SML \\
\tab\trans{elaborate} IL-Module \\
\tab\tab\trans{phase-splitting} IL-Direct \\
\tab\tab\tab\trans{cps conversion} IL-CPS \\
\tab\tab\tab\tab\trans{closure conversion} IL-Closure \\
\tab\tab\tab\tab\tab\trans{hoisting} IL-Hoist \\
\tab\tab\tab\tab\tab\tab\trans{allocation} IL-Alloc \\
\tab\tab\tab\tab\tab\tab\tab\trans{code-generation} C

\newpage
\section{Introduction to the $F\omega$ type system}

% This came first, fundementally useless for us
%\subsection{Grammar of the $F$ type system (``System F'')}
%\begin{align*}
%\tau & \bnfdef \alpha \bnfalt \tau \arrow \tau \bnfalt \forall \alpha. \tau \\
%e &\bnfdef x \bnfalt \lambda x \of \tau. e \bnfalt e\ e \bnfalt \Lambda \alpha. e \bnfalt e [ \tau ]
%\end{align*}
%Type $\tau$ and Term $e$.

\begin{grouped}{\subsection{Context for Judgements}}
\subsection{Grammar}
\begin{align*}
k & \bnfdef \type \bnfalt k \arrow k \\
c & \bnfdef \alpha \bnfalt c \arrow c \bnfalt \forall \bind{\alpha \of k}{c} \bnfalt c\ c \\
e & \bnfdef x \bnfalt \lambda \bind{x \of c}{e} \bnfalt e\ e
    \bnfalt \Lambda \bind{\alpha \of k}{e} \bnfalt e [ c ] \\
\end{align*}
Kind $k$, Type Constructor $c$, and Term $e$.\\
\note{$\type$ is often referred to with just ``T'' (by Crary), for simplicity.\\}
\end{grouped}

\begin{grouped}{\subsection{Context for Judgements}}
\begin{equation} \label{Context}
\Gamma \bnfdef \epsilon \bnfalt \Gamma, x \of \tau \bnfalt \Gamma, \alpha \of k
\end{equation}
\note{For simplicity, whenever a new $\alpha$ appears in the context, we
implicitly ensure that $\alpha$ is not already in $\Gamma$.\\}
\end{grouped}

\begin{grouped}{\subsection{$\Gamma \vd c \of k$}}
\begin{mathpar}
\inferr{\Gamma \vd \alpha \of k}
       {\Gamma(\alpha) = k}

\inferr{\Gamma \vd \tau_1 \arrow \tau_2 \of T}
       {\Gamma \vd \tau \of T \\ \Gamma \vd \tau_2 \of T}

\inferr{\Gamma \vd \forall\bind{\alpha \of k}{\tau}}
       {\Gamma, \alpha \of k \vd \tau \of T}

\inferr{\Gamma \vd \lambda\bind{\alpha \of k}{c} \of k \arrow k'}
       {\Gamma, \alpha \of k \vd c \of k'}

\inferr{\Gamma \vd c_1\ c_2 \of k'}
       {\Gamma \vd c_1 \of k \arrow k' \\ \Gamma \vd c_2 : k}
\end{mathpar}
\end{grouped}

\begin{grouped}{\subsection{$\Gamma \vd e \of \tau$}}
\begin{mathpar}
\inferr{\Gamma \vd x \of \tau}{\Gamma(x) = \tau}

\inferr{\Gamma \vd \lambda\bind{x \of \tau}{e} \of \tau \arrow \tau'}
       {\Gamma, x \of \tau \vd e \of \tau'}

\inferr{\Gamma \vd e_1\ e_2 \of \tau'}
       {\Gamma \vd e_1 \of \tau \arrow \tau' \\ \Gamma \vd e_2 : \tau}

\inferr{\Gamma \vd \Lambda\bind{\alpha \of k}{e} \of \forall\bind{\alpha \of k}{\tau}}
       {\Gamma, \alpha \of k \vd e \of \tau}

\inferr{\Gamma \vd e [c] \of \subst{c}{\alpha}{e}}
       {\Gamma \vd e \of \forall\bind{\alpha \of k}{e} \\ \Gamma \vd c \of k}

\inferr{\Gamma \vd e \of \tau'}
       {\Gamma \vd e \of \tau \\ \Gamma \vd \tau \equiv \tau' : T}
\end{mathpar}
\end{grouped}

\begin{grouped}{\subsection{$\Gamma \vd c \equiv c \of k$}}
Definitional Equivalence.\\

\begin{mathpar}
\inferr{\Gamma \vd c \equiv c \of k}{\Gamma \vd c \of k}

\inferr{\Gamma \vd c' \equiv c \of k}{\Gamma \vd c \equiv c' \of k}

\inferr{\Gamma \vd c_1 \equiv c_3 \of k}
       {\Gamma \vd c_1 \equiv c_2 \of k \\ \Gamma \vd c_2 \equiv c_2 \of k}
\end{mathpar}
The above are identity, reflexivity, and transitivity respectively.\\

The following are ``compatibility'' rules.\\
\begin{mathpar}
\inferr{\Gamma \vd c_1c_2 \equiv c_1'c_2' \of k}
       {\Gamma \vd c_1 \equiv c_1' \of k \\ \Gamma \vd c_2 \equiv c_2' \of k}

\inferr{\Gamma \vd \lambda\bind{\alpha \of k_1}{c} \equiv
                   \lambda\bind{\alpha \of k_1}{c'} \of k_1 \arrow k_2}
       {\Gamma, \alpha \of k_1 \vd c \equiv c' \of k_2}

\inferr{\Gamma \vd \tau_1 \arrow \tau_2 \equiv \tau_1' \arrow \tau_2' \of T}
       {\Gamma \vd \tau_1 \equiv \tau_1' \of k \\
        \Gamma \vd \tau_2 \equiv \tau_2' \of k}

\inferr{\Gamma \vd \forall\bind{\alpha \of k}{\tau} \equiv
                   \forall\bind{\alpha \of k}{\tau'} \of T}
       {\Gamma, \alpha \of k \vd \tau \equiv \tau' : T}
\end{mathpar}

congruence = compatible equivalence relation \\ %% TODO: WTF IS THIS
The following are the rules for beta equivalence and extensionality:\\
\begin{mathpar}
\inferr{\Gamma \vd (\lambda\bind{\alpha \of k}{c_1})\ c_2 \equiv
                  \subst{c_2}{\alpha}{c_1} \of k'}
       {\Gamma \vd c_2 \of k \\ \Gamma, \alpha \of k \vd c_1 \of k'}

\inferr{\Gamma \vd c \equiv c' \of k_1 \arrow k_2}
       {\Gamma, \alpha \of k_1 \vd c\ \alpha \equiv c'\ \alpha \of k_2 \\
        \Gamma \vd c \of k_1 \arrow k_2 \\ \Gamma \vd c' : k_1 \arrow k_2}
\end{mathpar}
\end{grouped}

\begin{grouped}{\subsection{Extending $F\omega$}}
\note{This helps in the understanding of sml's module system\\}

Grammar:
\begin{align*}
k & \bnfdef \ldots \bnfalt k \times k \\
c & \bnfdef \ldots \bnfalt \langle c, c \rangle \bnfalt \pi_1 c \bnfalt \pi_2 c
\end{align*}

New Judgements:
\begin{mathpar}
\inferr{\Gamma \vd \langle c_1, c_2 \rangle \of k_1 \times k_2}
       {\Gamma \vd c_1 \of k_2 \\ \Gamma \vd c_2 \of k_2}

\inferr{\Gamma \vd \pi_i\ c \of k_1}{\Gamma \vd c \of k_1 \times k_2}

\inferr{\Gamma \vd \langle c_1, c_2 \rangle \equiv
                   \langle c_1', c_2' \rangle \of k_1 \times k_2}
       {\Gamma \vd c_1 \equiv c_1' \of k_1 \\ \Gamma \vd c_2 \equiv c_2' \of k_2}

\inferr{\Gamma \vd \pi_i\ c \equiv \pi_i\ c' \of k_i}
       {\Gamma \vd c \equiv c' \of k_1 \times k_2}

\inferr{\Gamma \vd \pi_i\langle c_1, c_2 \rangle \equiv c_i \of k_i}
       {\Gamma \vd c_1 \of k_1 \\ \Gamma \vd c_2 \of k_2}

\inferr{\Gamma \vd c \equiv c' \of k_1 \times k_2}
       {\Gamma \vd \pi_1\ c \equiv \pi_1\ c' \of k_1 \\
        \Gamma \vd \pi_2\ c \equiv \pi_2\ c' \of k_2}
\end{mathpar}
\end{grouped}


\newpage
\section{Algorithmic Equivalence in the $F\omega$ Type System}
\begin{grouped}{\subsection{Normalize-and-Compare}}
\note{We don't use this.\\}

$\lambda\bind{\alpha \of k}{c_1}\ c_2 \xrightarrow{\beta} \subst{c_2}{\alpha}{c_1}$ \\
$\pi_i \langle c_1, c_2 \rangle \xrightarrow{\beta} c_i$ \\
+ some $\eta$ reduction rules \\

According to some equivalence theorem, they will have normal forms
and those normal forms will be equal if they are equivalent.
\end{grouped}

\begin{grouped}{\subsection{Grammar and Properties}}
Paths:\\
$p \bnfdef \alpha \bnfalt p\ c \bnfalt \pi_1\ p \bnfalt \pi_2\ p$\\

Weak-Head Normal Form:\\
$n \bnfdef p \bnfalt c_1 \arrow c_2 \bnfalt \forall\bind{\alpha \of k}{c}$.\\

Regularity:\\
If $\vd \Gamma \ok$ and $\Gamma \vd c_1 \equiv c_2 \of k$,
then $\Gamma \vd c_1 \of k$ and $\Gamma \vd c_2 \of k$.\\
If $\vd \Gamma \ok$ and $\Gamma \vd c \of k$, then $\Gamma \vd k \of \kind$.\\

Soundness:\\
If $\vd \Gamma \ok$ and $\Gamma \vd c_1, c_2 \of k$ and $\Gamma \vd c_1 \ace c_2 \of k$,
then $\vd c_1 \equiv c_2 \of k$.\\

Completeness:\\
If $\vd \Gamma \ok$ and $\Gamma \vd c_1 \equiv c_2 \of k$,
then $\Gamma \vd c_1 \ace c_2 \of k$.\\

\begin{mathpar}
\inferr{\vd \epsilon \ok}{\strut}

\inferr{\vd \Gamma, \alpha \of k \ok}{\vd \Gamma \ok \\ \Gamma \vd k \of \kind}

\inferr{\vd \Gamma, x \of \tau \ok}{\vd \Gamma \ok \\ \Gamma \vd \tau \of \type}
\end{mathpar}
\end{grouped}

\begin{grouped}{\subsection{Algorithmic Constructor Equivalence}}
Form: $\oversetr{\Gamma}{+} \vd \oversetr{c_1}{+} \ace \oversetr{c_2}{+} \of \oversetr{k}{+}$\\
\note{$\oversetr{x}{+}$ indicates that $x$ is an input.}

\begin{mathpar}
\inferr{\Gamma \vd c \ace c' \of k_1 \arrow k_2}
       {\Gamma, \alpha \of k_1 \vd c\ \alpha \ace c'\ \alpha \of k_2}

\inferr{\Gamma \vd c \ace c' \of k_1 \times k_2}
       {\Gamma \vd \pi_1\ c \ape \pi_1\ c' \of k_1 \\
        \Gamma \vd \pi_2\ c \ape \pi_2\ c' \of k_2}

\inferr{\Gamma \vd c_1 \ace c_2 \of \type}
       {c_1 \whn c_1' \\ c_2 \whn c_2' \\ \Gamma \vd c_1' \ape c_2' \of \type}
\end{mathpar}

\subsection{Algorithmic Path Equivalence}
Form: $\oversetr{\Gamma}{+} \vd \oversetr{c_1}{+} \ace \oversetr{c_2}{+} \of \oversetr{k}{-}$\\
\note{$\oversetr{x}{-}$ indicates that $x$ is an output.}

\begin{mathpar}
\inferr{\Gamma \vd \alpha \ape \alpha \of k}{\Gamma(\alpha) = k}

\inferr{\Gamma \vd p\ c \ape p'\ c' \of k_1}
       {\Gamma \vd p \ape p' \of k_1 \arrow k_2 \\ \Gamma \vd c \ace c' \of k_1}

\inferr{\Gamma \vd \pi_i\ p \ape \pi_i\ p' \of k_i}
       {\Gamma \vd p \ape p' \of k_1 \times k_2}

\inferr{\Gamma \vd c_1 \arrow c_2 \ape c_1' \arrow c_2' \of T}
       {\Gamma \vd c_1 \ace c_1' \of T \\ \Gamma \vd c_1 \ace c_2' \of T}

\inferr{\Gamma \vd \forall\bind{\alpha \of k}{c} \ape
                   \forall\bind{\alpha \of k}{c'} \of T}
       {\Gamma, \alpha \of k \vd c \ace c' \of T}
\end{mathpar}

\subsection{Weak-Head Normalization}
Form: $\oversetr{c}{+} \whn \oversetr{n}{-}$

\begin{mathpar}
\inferr{c \whn c''}{c \whr c' \\ c' \whn c''}

\inferr{c \whn c}{c \not\whr}
\end{mathpar}

\subsection{Weak-Head Reduction}
Form: $\oversetr{c}{+} \whr \oversetr{c'}{-}$

\begin{mathpar}
\inferr{(\lambda\bind{\alpha \of k}{c_1})\ c_2 \whr \subst{c_2}{\alpha}{c_1}}{\strut}

\inferr{\pi_i \langle c_1, c_2 \rangle \whr c_i}{\strut}

\inferr{c_1\ c_2 \whr c_1'\ c_2}{c_1 \whr c_1'}

\inferr{\pi_i c \whr \pi_i c'}{c \whr c'}
\end{mathpar}
\end{grouped}

\begin{grouped}{\subsection{Kind Synthesis and Checking}}
Form: $\oversetr{\Gamma}{+} \vd \oversetr{c}{+} \synthesis \oversetr{k}{-}$ and
$\oversetr{\Gamma}{+} \vd \oversetr{c}{+} \checking \oversetr{k}{+}$

\begin{mathpar}
\inferr{\Gamma \vd \alpha \synthesis k}{\Gamma(\alpha) = k}

\inferr{\Gamma \vd \lambda\bind{\alpha \of k}{c} \synthesis k \arrow k'}
       {\Gamma, \alpha \of k \vd c \synthesis k'}

\inferr{\Gamma \vd c_1\ c_2 \synthesis k'}
       {\Gamma \vd c_1 \synthesis k \arrow k' \\ \Gamma \vd c_2 \checking k}

\inferr{\Gamma \vd \langle c_1, c_2 \rangle \synthesis k_1 \times k_2}
       {\Gamma \vd c_1 \synthesis k_1 \\ \Gamma \vd c_2 \synthesis k_2}

\inferr{\Gamma \vd \pi_i\ c \synthesis k_1}{\Gamma \vd c \synthesis k_1 \times k_2}

\inferr{\Gamma \vd c_1 \arrow c_2 \synthesis T}
       {\Gamma \vd c_1 \checking T \\ \Gamma \vd c_2 \checking T}

\inferr{\Gamma \vd \forall\bind{\alpha \of k}{c} \synthesis T}
       {\Gamma, \alpha \of k \vd c \checking T}

\inferr{\Gamma \vd c \checking k}{\Gamma \vd c \synthesis k}
\end{mathpar}

\subsection{Type Checking and Synthesis}
Form: $\oversetr{\Gamma}{+} \vd \oversetr{e}{+} \synthesis \oversetr{c}{-}$ and
$\oversetr{\Gamma}{+} \vd \oversetr{e}{+} \checking \oversetr{c}{+}$

\begin{mathpar}
\inferr{\Gamma \vd x \synthesis \tau}{\Gamma(x) = \tau}

\inferr{\Gamma \vd \lambda\bind{x \of \tau}{e} \synthesis \tau \arrow \tau'}
       {\Gamma \vd \tau \checking T \\ \Gamma, x \of \tau \vd e \synthesis \tau'}

\inferr{\Gamma \vd e_1\ e_2 \synthesis \tau'}
       {\Gamma \vd e_1 \synthesis \tau_1 \\
        \tau_1 \whn \tau \arrow \tau' \\
        \Gamma \vd e_2 \checking \tau}

\inferr{\Gamma \vd \Lambda\bind{\alpha \of}{e} \synthesis
                   \forall\bind{\alpha \of k}{\tau}}
       {\Gamma, \alpha \of k \vd e \synthesis \tau}

\inferr{\Gamma \vd e [c] \synthesis \subst{c}{\alpha}{\tau'}}
       {\Gamma \vd e \synthesis \tau \\
        \tau \whn \forall\bind{\alpha \of k}{\tau'} \\
        \Gamma \vd c \checking k}

\inferr{\Gamma \vd e \checking \tau}
       {\Gamma \vd e \synthesis \tau' \\ \Gamma \vd \tau \ace \tau' : T}
\end{mathpar}
\end{grouped}


%\newpage
% TODO: this section is super incoherent
\section{Substitution}

\begin{grouped}{\subsection{Explicit Variables}}
\begin{lstlisting}
datatype con = Var of variable | Lam of variable * kind * con
datatype term = Var of variable | Lam of variable * con * term
\end{lstlisting}

$c \bnfdef \alpha \bnfalt \lambda\bind{\alpha \of \tau}{c}$\\
We always alpha variate because we can (saves the cost of a search):\\
$\subst{M}{x}{(\lambda\bind{y}{N})} = \lambda\bind{y'}{\subst{M}{x}{\subst{y'}{y}{N}}}$ (find $y' \not\in \fv(M)$)\\
\end{grouped}

\begin{grouped}{\subsection{de Bruijn}}
\begin{lstlisting}
datatype con = Var of int | Lam of kind * con
\end{lstlisting}

Rules for substitution:
\begin{align*}
\subst{M}{i}{j} &=
  \begin{cases} 
    M & i = j \\
    j - 1 & i < j \\
    j & i > j
  \end{cases}\\
\subst{M}{i}{N\ P} &= \subst{M}{i}{N}\ \subst{M}{i}{P} \\
\subst{M}{i}{\lambda\bind{}{N}} &= \lambda\bind{}{\subst{\lift_{\ge0}M}{i+1}{N}}
\end{align*}

Rules for $\lift$:
\begin{align*}
\lift_{\ge i} j = \begin{cases} j + 1 & j \ge i \\ j & j < i \end{cases} \\
\lift_{\ge i} M\ N = \uparrow_{\ge i} M\ \uparrow_{\ge i} N \\
\lift_{\ge i} \lambda\bind{}{N} = \lambda\bind{}{\lift_{\ge i + 1} N}
\end{align*}
\end{grouped}

\begin{grouped}{\subsection{Explicit Substitution}}
Grammar:\\
$\sigma \bnfdef \bind{M}{\sigma} \bnfalt \lift^i$\\

\note{$\id \overset{\mathop{def}}{=} \lift^0$\\}

Syntax Definition:\\
\begin{flalign*}
0 [\bind{M}{\sigma}] &= M &\\
i + 1 [\bind{M}{\sigma}] &= i [\sigma] &\\
n [\lift^i] &= n + i &\\
(MN) [\sigma] &= M[\sigma] N[\sigma] &\\
(\lambda\bind{A}{M}) [\sigma] &= \lambda A[\sigma]. M[0.(\sigma \comp \lift^1)] &\\
M[\sigma \comp \sigma'] &= M[\sigma][\sigma'] &\\
(M.\sigma) \comp \sigma' &= M[\sigma'].(\sigma \comp \sigma') &\\
\lift^0 \comp \sigma &= \sigma &\\
\lift^{i + 1} \comp \sigma &= \lift^i \comp \sigma &\\ % TODO
\lift^i \comp \lift^j &= \lift^{i + j}
\end{flalign*}

Syntax Exampes:\\
$[M/0, N/1]$ % TODO: wtf
\begin{flalign*}
0 [\bind{M}{\bind{N}{\id}}] &= M &\\
1 [\bind{M}{\bind{N}{\id}}] &= N &\\
2 [\bind{M}{\bind{N}{\id}}] &= 0 &\\
M &= M[0.\uparrow^1] &\\
  &= M[0.1.2....n.\uparrow^{n + 1}] &\\
\lift_{\ge 0}M &= M[\uparrow^1] &\\
\lift_{\ge 0}\lift_{\ge 0}M &= M[\uparrow^2] &\\
\lift_{\ge 1}M &= M[0.\uparrow^2]
\end{flalign*}
The last one here is a bit tricky. If you see $0$, we want to leave it as $0$,
and if we see $1$ or more, it is lowered to $0$, so to make up for that
change, shift up by $2$.\\
\end{grouped}

\begin{grouped}{Substitution Code (provided)\\}
\begin{lstlisting}
subst X Gen : int -> con list -> int -> X -> X
subst X Gen i [$c_1 \ldots c_k$] l M = M[$0. ... i - 1. c_1[\lift^i]. ... c_k[\lift^i]. \lift^{i+l}$]
  (* leave first i vars alone ($0 \ldots i-1$)
   * subst $c_i$ but need to shift up by i when passing into $c_i$ *)

substX c M = substXGen 0 [c] 0

liftX l M = substXGen 0 [] l (* whenever you move under a binder *)
\end{lstlisting}
\end{grouped}

% TODO: huge disconnect, FORMAT-------------------------------------------------

$\Gamma \bnfdef \epsilon \bnfalt \Gamma, A$ \\

% TODO: huge disconnect

Typing Judgement for a substitution:\\
if $\Gamma \vd \sigma \of \Gamma'$ and $\Gamma' \vd M \of B$
then $\Gamma \vd M[\sigma] \of B[\sigma]$\\
``Any term in $\Gamma'$, after we apply $\sigma$, we should get $\Gamma$''\\

\begin{mathpar}
\inferr{\Gamma, A_1 \ldots A_i \vd \lift^i \of \Gamma}{\strut}

\inferr{\Gamma \vd \bind{M}{\sigma} \of \Gamma', A}
       {\Gamma' \vd \sigma \of \Gamma' \\ \Gamma \vd M \of A[\sigma]}
\end{mathpar}

Example:\\
\newcommand{\str}{\texttt{string}}
\[
\infer{\Gamma \vd [\bind{\texttt{"hello world"}}
                        {\bind{\str}{\id}}] \of \Gamma, \type, 0}
      {\infer{\Gamma \vd \bind{\str}{\id} \of \Gamma, \type} % TODO
             {\Gamma \vd \id \of \Gamma & \Gamma \vd \str \of \type[\id]}
      &\infer{\Gamma \vd \texttt{"hello world"} \of 0[\bind{\str}{\id}]}{\strut}}
\]
\note{$0[\bind{\str}{\id}]$ is just $\str$}
% TODO: FORMAT ^^^^^^^^^^^^^^^^^^^^^^^^^^^^^^^^^^^^^^^^^^^^^^^^^^^^^^^^^^^^^^^^^


\newpage
\section{Singleton Kinds}

\begin{lstlisting}
sig
  type t
  type 'a u
  type ('a,'b) v
  type w = int
  type w' = w
  .
  .
  .
end
\end{lstlisting}

To represent this in type our type system,
$t \of \type$\\
$u \of \type \arrow \type$\\
$v \of \type \arrow \type \arrow \type$\\
(or $v \of \type \times \type \arrow \type$)\\
$w \of \singleton{\int}$\\
$w' \of \singleton{w}$\\

\subsection{Grammar and Judgements (Attempt 1)}
Grammar:
\begin{flalign*}
k &\bnfdef \type \bnfalt k \arrow k \bnfalt k \times k \bnfalt \singleton{c} &\\
c &\bnfdef \ldots
\end{flalign*}

Judgements:
\begin{mathpar}
\inferr{\Gamma \vd c \of \singleton{c}}{\strut}

\inferr{\Gamma \vd c \equiv c' \of \type}{\Gamma \vd c \of \singleton{c}}

\inferr{\Gamma \vd \singleton{c} \of \kind}{\Gamma \vd c \of \type}
\end{mathpar}

Signature for {\tt list}.
\begin{lstlisting}
sig
  .
  .
  .
  type 'a s = 'a list
  type 'a t
end
\end{lstlisting}
So we have $t \of \type \arrow \type$.\\
How do we represent 'a s? Is $s \of \type \arrow \singleton{\alpha}$? But
then what's $\alpha$.

\begin{grouped}{\subsection{Dependent Kinds (Grammar)}}
\begin{flalign*}
k &\bnfdef \type \bnfalt \Pi\bind{\alpha \of k}{k}
   \bnfalt \Sigma\bind{\alpha \of k}{k} \bnfalt \singleton{c} &\\
c &\bnfdef \ldots
\end{flalign*}

% TODO
\note{$\Pi$ is also known as ``dependent product''\\
$\Sigma$ is also known as ``dependent sum'' (but also sometimes as
``dependent product'').\\
To avoid confusion, we name $\Pi$ ``dependent function (spaces)''.\\}

Now, we have $s \of \Pi\bind{\alpha \of \type}{\singleton{\mathop{list} \alpha}}$.\\

New judgements we need to be able to make: \\
$\Gamma \vd k \of \kind$ \\
$\Gamma \vd k \equiv k' \of kind$ \\
$\Gamma \vd k \le k'$ \\
$\Gamma \vd c \of k$ \\
$\Gamma \vd c \equiv c' \of k$ \\
$\Gamma \vd e \of \tau$ \\

\note{$\singleton{\int} \le \type$}
\end{grouped}

\begin{grouped}{\subsection{$\Gamma \vd k \of \kind$}}
\begin{mathpar}
\inferr{\Gamma \vd \type \of \kind}{\strut}

\inferr{\Gamma \vd \singleton{c} \of \kind}{\Gamma \vd c \of \type}

\inferr{\Gamma \vd \Pi\bind{\alpha \of k_1}{k_2} \of \kind}
       {\Gamma \vd k_1 \of \kind \\ \Gamma, \alpha \of k_1 \vd k_2 \of \kind}

\inferr{\Gamma \vd \Sigma\bind{\alpha \of k_1}{k_2} \of \kind}
       {\Gamma \vd k_1 \of \kind \\ \Gamma, \alpha \of k_1 \vd k_2 \of \kind}
\end{mathpar}
\end{grouped}

\begin{grouped}{\subsection{$\Gamma \vd k \equiv k' \of \kind$}}
\begin{mathpar}
\inferr{\Gamma \vd k \equiv k \of \kind}{\Gamma \vd k \of \kind}

\inferr{\Gamma \vd k_2 \equiv k_1 \of \kind}{\Gamma \vd k_1 \equiv k_2 \of \kind}

\inferr{\Gamma \vd k_1 \equiv k_2 \of \kind}
       {\Gamma \vd k_1 \equiv k_2 \of \kind \\ \Gamma \vd k_2 \equiv k_2 \of \kind}

% compatibility rules
\inferr{\Gamma \vd \singleton{c} \equiv \singleton{c'} \of \kind}
       {\Gamma \vd c \equiv c' \of \type}

\inferr{\Gamma \vd \Pi\bind{\alpha \of k_1}{k_2} \equiv
                   \Pi\bind{\alpha \of k_1'}{k_2'} \of \kind}
       {\Gamma \vd k_1 \equiv k_1' \of \kind \\
        \Gamma, \alpha \of k_1 \vd k_2 \equiv k_2' \of \kind}

\inferr{\Gamma \vd \Sigma\bind{\alpha \of k_1}{k_2} \equiv
                   \Sigma\bind{\alpha \of k_1'}{k_2'} \of \kind}
       {\Gamma \vd k_1 \equiv k_1' \of \kind \\
        \Gamma, \alpha \of k_1 \vd k_2 \equiv k_2' \of \kind}
\end{mathpar}
\note{for the latter two, keep
$\Pi\bind{\alpha \of k_1}{k_2} \oversetr{\equiv}{?} \Pi\bind{\alpha' \of k_1'}{k_2'}$
in mind}
\end{grouped}

\begin{grouped}{\subsection{$\Gamma \vd \alpha \of k$}}
\begin{mathpar}
\inferr{\Gamma \vd \alpha \of k}{\Gamma(\alpha) = k}

\inferr{\Gamma \vd c_1 \arrow c_2 \of \type}
       {\Gamma \vd c_1 \of \type \\ \Gamma \vd c_2 \of \type}

\inferr{\Gamma \vd \forall\bind{\alpha : k}{c} \of \type}
       {\Gamma \vd k \of kid \\ \Gamma, \alpha \of k \vd c \of \type}

\inferr{\Gamma \vd \lambda\bind{\alpha \of k_1}{c} \of \Pi\bind{\alpha \of k_1}{k_2}}
       {\Gamma \vd k_1 \of \kind \\ \Gamma, \alpha \of k_1 \vd c \of k_2}

\inferr{\Gamma \vd c_1\ c_2 \of \subst{c_1}{\alpha}{k'}}
       {\Gamma \vd c_1 \of \Pi\bind{\alpha \of k}{k'} \\ \Gamma \vd c_2 \of k}

\inferr{\Gamma \vd \langle c_1, c_2 \rangle \of \Sigma\bind{\alpha \of k_2}{k_2}}
       {\Gamma \vd c_1 \of k_2 \\
        \Gamma \vd c_2 \of \subst{c_1}{\alpha}{k_2} \\
        \Gamma, \alpha \of k_1 \vd k_2 \of \kind}

\inferr{\Gamma \vd \pi_1 c \of k_1}
       {\Gamma \vd c \of \Sigma\bind{\alpha \of k_1}{k_2}}

\inferr{\Gamma \vd \pi_2 c \of \subst{\pi_1 c}{\alpha}{k_2}}
       {\Gamma \vd c \of \Sigma\bind{\alpha \of k_1}{k_2}}

% TODO
\inferr{\Gamma \vd c \of k'}{\Gamma \vd c \of k \\ \Gamma \vd k \le k'}
\end{mathpar}
\end{grouped}

\begin{grouped}{Additional Judgements}
If $\vd \Gamma \ok$ and $\Gamma \vd c \of k$, then $\Gamma \vd k \of \kind$.\\
If $\vd \Gamma \ok$ and $\Gamma \vd k_1 \equiv k_2$, then $\Gamma \vd k_1, k_2 \kind$.\\
If $\vd \Gamma \ok$ and $\Gamma \vd k_1 \le k_2$, then $\Gamma \vd k_1, k_2 \kind$.\\

\begin{mathpar}
\inferr{\Gamma \vd c \of \singleton{c}}{\Gamma \vd c \of \type}
\end{mathpar}

Sub-kinding:
\begin{mathpar}
\inferr{\Gamma \vd k \le k'}{\Gamma \vd k \equiv k' \of \kind}

\inferr{\Gamma \vd k_1 \le k_3}{\Gamma \vd k_1 \le k_2 \\ \Gamma \vd k_2 \le k_3}

% substantiative
\inferr{\Gamma \vd \singleton{c} \le \type}{\Gamma \vd c \of \type}

% compatibility rules
\inferr{\Gamma \vd \singleton{c} \le \singleton{c'}}
       {\Gamma \vd c \equiv c' \of \type}

\inferr{\Gamma \vd \Pi\bind{\alpha \of k_1}{k_2} \le
                   \Pi\bind{\alpha \of k_1'}{k_2'}}
       {\Gamma \vd k_1' \le k_1 \\
        \Gamma, \alpha \of k_1' \vd k_2 \le k_2' \\
        \Gamma, \alpha \of k_1 \vd k_2 \of \kind}

\inferr{\Gamma \vd \Sigma\bind{\alpha \of k_1}{k_2} \le
                   \Sigma\bind{\alpha \of k_1'}{k_2'}}
       {\Gamma \vd k_1 \le k_1' \\
        \Gamma, \alpha \of k_1 \vd k_2 \le k_2' \\
        \Gamma, \alpha \of k_1' \vd k_2' \of \kind}
\end{mathpar}
\note{Something about contravariance for 1st condition.
$\Pi$ contravariant the same way arrow is contravariant.
Covariance for 2nd condition. This is for $\Pi$.}

\end{grouped}

\begin{grouped}{\subsection{$\Gamma \vd c \equiv c \of k$}}
\begin{mathpar}
\inferr{\Gamma \vd c \equiv c \of k}{\Gamma \vd c \of k}

\inferr{\Gamma \vd c_2 \equiv c_1 \of k}{\Gamma \vd c_1 \equiv c_2 \of k}

\inferr{\Gamma \vd c_1 \equiv c_3 \of k}
       {\Gamma \vd c_1 \equiv c_2 \of k \\ \Gamma \vd c_2 \equiv c_3 \of k}

% SUBSTANTIATIVE
\inferr{\Gamma \vd (\lambda\bind{\alpha \of k}{c_1})\ c_2 \equiv \subst{c_2}{\alpha}{c_1} \of \subst{c_2}{\alpha}{k'}}{\Gamma \vd c_2 \of k \\ \Gamma, \alpha \of k \vd c_1 \of k'}

\inferr{\Gamma \vd \pi_i \langle c_1, c_2 \rangle \equiv c_i \of k_i}{\Gamma \vd c_1 \of k_1 \\ \Gamma \vd c_2 \of k_2}

\inferr{\Gamma \vd c \equiv c' \of \singleton{c'}}
       {\Gamma \vd c \of \singleton{c'}}

\inferr[$\star$]
       {\Gamma \vd c_1 \equiv c_2 \of k'}
       {\Gamma \vd c_1 \equiv c_2 \of k \\ \Gamma \vd k \le k'}

\inferr[$\star$]
       {\Gamma \vd c \equiv c' \of \singleton{c}}
       {\Gamma \vd c \equiv c' \of \type}

\inferr{\Gamma \vd \lambda\bind{\alpha \of k_1}{c} \equiv
                   \lambda\bind{\alpha \of k_1'}{c'} \of \Pi\bind{\alpha \of k_1}{k_2}}
       {\Gamma \vd k_1 \equiv k_1' \of \kind \\
        \Gamma, \alpha \of k_1 \vd c \equiv c' \of k_2}

\inferr{\Gamma \vd c_1\ c_2 \equiv c_1'\ c_2' \of \subst{c_2}{\alpha}{k'}}
       {\Gamma \vd c_1 \equiv c_1' \of \Pi\bind{\alpha \of k}{k'} \\
        \Gamma \vd c_2 \equiv c_2' \of k}
\end{mathpar}
\end{grouped}

%$\Gamma \vd k \le k'$ \\
%$\Gamma \vd c \equiv c' \of k$ \\
%$\Gamma \vd e \of \tau$ \\



\newpage
\section{Sub-Typing}

$\tau \le \tau'$ means you can use a $\tau$ whrever a $\tau'$ is expected.\\

\begin{mathpar}
\inferr%[subsumption]
       {\Gamma \vd e \of \tau'}{\Gamma \vd e \of \tau \\ \tau \le \tau'}

\inferr%[$*^1$]
       {\Gamma \vd \tau \times \tau_2 \le \tau_1' \times \tau_2'}
       {\tau_1 \le \tau_1' \\ \tau_2 \le \tau_2'}

\inferr%[$*^2$]
       {\tau_1 \arrow \tau_2 \le \tau_1' \arrow \tau_2'}
       {\tau_1' \le \tau_1 \\ \tau_2 \le \tau_2'}
\end{mathpar}
Note 1: This is a case of covariance on both sides.\\
Note 2: This is contravariant on the left and covariant on the right.\\

$\nats \le \reals$.\\
Assume we have $f \of \nats \arrow \nats$.\\
$f \not\of \reals \arrow \reals$.\\

Assume we have $f \of \reals \arrow \reals$.\\
Contravariance: $f \of \nats \arrow \reals$.\\

\begin{mathpar}
\inferr{\sref{\tau} \le \sref{\tau'}}{\tau \equiv \tau' \of \type}
\end{mathpar}
$\sref{\tau}$ is neither covariant nor contravariant. Called ``invariant''.
(Poorly named, but it's what's used in literature.)\\

\begin{mathpar}
\inferr{\Gamma \vd \Pi\bind{\alpha \of k_1}{k_2} \le
                   \Pi\bind{\alpha \of k_1'}{k_2'}}
       {\Gamma \vd k_1' \le k_1 \\ \Gamma, \alpha \of k_1' \vd k_2 \le k_2' \\
        \Gamma, \alpha \of k_1 \vd k_2 \of kind}

\inferr{\Gamma \vd \Sigma\bind{\alpha \of k_1}{k_2} \le
                   \Sigma\bind{\alpha \of k_1'}{k_2'}}
       {\Gamma \vd k_1' \le k_1 \\ \Gamma, \alpha \of k_1' \vd k_2 \le k_2' \\
        \Gamma, \alpha \of k_1 \vd k_2 \of \kind}

\inferr%[$\times^1$]
       {\Gamma \vd c \equiv c' \of \singleton{c'}}
       {\Gamma \vd c \of \singleton{c'}}

\inferr{\Gamma \vd c \equiv c' \of \type}{\Gamma \vd c \of \singleton{c'}}

\inferr{\Gamma \vd c \of c' \of \singleton{c}}{\Gamma \vd c \equiv c' \of \type}
\end{mathpar}
Note 1: Sound, but not what we want.\\

More compatiblity rules.\\
\begin{mathpar}
\inferr{\Gamma \vd \langle c_1, c_2 \rangle \equiv \langle c_1', c_2' \rangle \of \Sigma\bind{\alpha \of k_1}{k_2}}
       {\Gamma \vd c_1 \equiv c_1' \of k_1 \\
        \Gamma \vd c_2 \equiv c_2' \of \subst{c_1}{\alpha}{k_2} \\
        \Gamma, \alpha \of k_1 \vd k_2 \of \kind}

\inferr{\Gamma \vd \pi_1 c \equiv \pi_1 c' \of k_1}
       {\Gamma \vd c \equiv c' \of \Sigma\bind{\alpha \of k_1}{k_2}}

\inferr{\Gamma \vd \pi_2 c \equiv \pi_2 c' \of \subst{\pi_1 c}{\alpha}{k_2}}
       {\Gamma \vd c \equiv c' \of \Sigma\bind{\alpha \of k_1}{k_2}}

\inferr{\Gamma \vd c_1 \arrow c_2 \equiv c_1' \arrow c_2' \of \type}
       {\Gamma \vd c_1 \equiv c_1' \of \type \\
        \Gamma \vd c_2 \equiv c_2' \of \type}

\inferr{\Gamma \vd \forall\bind{\alpha \of k}{c} \equiv
                   \forall\bind{\alpha \of k'}{c'} \of \type}
       {\Gamma \vd k \equiv k' \of \kind \\ \Gamma, \alpha \of k \vd c \equiv c' \of \type}
\end{mathpar}

Rules for extentionality.
\begin{mathpar}
\inferr{\Gamma \vd c \equiv c' \of \Pi\bind{alpha \of k_1}{k_2}}
       {\Gamma, \alpha \of k_1 \vd c\ \alpha \equiv c'\ \alpha \of k_2 \\
        \Gamma \vd c \of \Pi\bind{alpha \of k_1}{k_2'} \\
        \Gamma \vd c' \of \Pi\bind{alpha \of k_1}{k_2''}}

\inferr%[$*^1$]
       {\Gamma \vd c \equiv c' \of \Pi\bind{\alpha \of k_1}{k_2}}
       {\Gamma, \alpha \of k_1 \vd c\ \alpha \equiv c'\ \alpha \of k_2 \\
        \Gamma \vd c \equiv c' \of \Pi\bind{alpha \of k_1}{k_2'}}

\inferr{\Gamma \vd c \equiv c' \of \Sigma\bind{alpha \of k_1}{k_2}}
       {\Gamma \vd \pi_1 c \equiv \pi_1 c' \of k_1 \\
        \Gamma \vd \pi_2 c \equiv \pi_2 c' \of \subst{\pi_1 c}{\alpha}{k_2} \\
        \Gamma, \alpha \of k_1 \vd k_2 \of \kind}
\end{mathpar}
Note 1: We only need this for proofs (regularity). We can safely ignore this.\\

% TODO: motivation?
We have no way of dealing with $\singleton{c \of k}$. So instead of redefining
everything, treat it as a macro following the following rules:\\
$\singleton{c \of \type} = \singleton{c}$ \\
$\singleton{c \of \Pi\bind{\alpha \of k_1}{k_2}}
  = \Pi\bind{\alpha \of k_1}{\singleton{c\ \alpha \of k_2}}$ \\
$\singleton{c \of \singleton{c'}} = \singleton{c}$
(note here, $c \equiv c'$, so we can use either, but it's easier for us to use $c$) \\
$\singleton{c \of \Pi\bind{\alpha \of k_1}{k_2}}
  = \Sigma\bind{\alpha \of \singleton{\pi_1 c \of k_1}}{\singleton{\pi_2 c \of k_2}}$ \\
OR $\singleton{\pi_1 c \of k_1} \times \singleton{\pi_2 c \of \subst{\pi_1 c}{\alpha}{k_2}}$ \\
We use the 2nd because it's nicer when not working without theory. The first is
more theoretic, the second is syntactic.\\


% TODO
\begin{enumerate}
\item If $\Gamma \vd c \of k$, then $\Gamma \vd c \of \singleton{c \of k}$
\item If $\Gamma \vd c \of \singleton{c' \of k}$, then $\Gamma \vd c \equiv c' \of k$
\end{enumerate}

But the first doesn't hold. So let's make it hold. Add ``declarative'' rules:
\begin{mathpar}
\inferr{\Gamma \vd c \of \Pi\bind{\alpha \of k_1}{k_2}}
       {\Gamma \vd k_1 \of \kind \\
        \Gamma, \alpha \of k_1 \vd c\ \alpha \of k_2}

\inferr{\Gamma \vd c \of \Sigma\bind{\alpha \of k_1}{k_2}}
       {\Gamma \vd \pi_1 c \of k_2 \\
        \Gamma \vd \pi_1 c \of \subst{\pi_1 c}{\alpha}{k_2} \\
        \Gamma, \alpha \of k_1 \vd k_2 \of \kind}
\end{mathpar}

Notes on definitional equivalence:\\
$\alpha \of \type \vd \alpha \not\equiv \intt \of \type$ \\
$\alpha \of \singleton\intt \vd \alpha \equiv \intt \of \type$ \\
$\vd \lambda\bind{\alpha \of \type}{\alpha} \not\equiv \lambda\bind{\alpha \of \type}{\intt} \of \type \arrow \type$ \\
$\vd \lambda\bind{\alpha \of \type}{\alpha} \not\equiv \lambda\bind{\alpha \of \type}{\intt} \of \singleton\intt \arrow \type$ \\
$\beta \of (\type \arrow \type) \arrow \type \vd \beta(\lambda\bind{\alpha \of \type}{\alpha} \not\equiv \beta(\lambda\bind{\alpha \of \type}{\intt} \of \type$ \\
$\beta \of (\singleton\intt \arrow \type) \arrow \type \vd \beta(\lambda\bind{\alpha \of \type}{\alpha} \equiv \beta(\lambda\bind{\alpha \of \type}{\intt} \of \type$ \\
$\type \arrow \type \le \singleton\intt \arrow \type$ \\


\subsection{Algorithm for Equivalence Checking}
\begin{mathpar}
\inferr{\Gamma \vd c \ace c' \of \Pi\bind{\alpha \of k_1}{k_2}}
       {\Gamma, \alpha \of k_1 \vd c\ \alpha \ace c'\ \alpha \of k_2}

\inferr{\Gamma \vd c \ace c' \of \Sigma\bind{\alpha \of k_1}{k_2}}
       {\Gamma \vd \pi_1 c \ace \pi_2 c' \of k_1
        \Gamma \vd \pi_2 c \ace \pi_2 c' \of \subst{\pi_1 c}{\alpha}{k_2}}

\inferr{\Gamma \vd c_1 \ace c_2 \of \type}
       {\Gamma \vd c_1 \whn c_1' \\ \Gamma \vd c_2 \whn c_2' \\
        \Gamma \vd c_1' \ape c_2' \of \type}

\inferr{\Gamma \vd c \whn c''}{\Gamma \vd c \whr c' \\ \Gamma \vd c' \whn c''}

\inferr{\Gamma \vd c \whn c}{\Gamma \vd c \not\whr}

\inferr{\Gamma \vd (\lambda\bind{\alpha \of k}{c_1})\ c_2 \whr \subst{c_2}{\alpha}{c_1}}{\strut}

\inferr{\Gamma \vd c_1\ c_2 \whr c_1'\ c_2}{\Gamma \vd c_1 \whr c_1'}

\inferr{\Gamma \vd \pi_i \langle c_1, c_2 \rangle \whr c_i}{\strut}

\inferr{\Gamma \vd pi_i c \whr pi_i c'}{\Gamma \vd c \whr c'}

\inferr{\Gamma \vd p}{\Gamma \vd p \nk \singleton{c}}

\inferr{\Gamma \vd \alpha \nk k}{\Gamma(\alpha) = k}

\inferr{\Gamma \vd p\ c \nk \subst{c}{\alpha}{k_2}}
       {\Gamma \vd p \nk \Pi\bind{\alpha \of k_1}{k_2}}

\inferr{\Gamma \vd \pi_1 p \nk k_1}{\Gamma \vd p \nk \Sigma\bind{\alpha \of k_1}{k_2}}

\inferr{\Gamma \vd \pi_2 p \nk \subst{\pi_1 p}{\alpha}{k_2}}
       {\Gamma \vd p \nk \Sigma\bind{\alpha \of k_1}{k_2}}
\end{mathpar}

% TODO nk "natural kind" is uparrow
% \Gamma+ \vd p+ \nk k-




\newpage
% TODO: continuation from previous time

\begin{mathpar}
\inferr{\Gamma \vd p \whr c}{\Gamma \vd p \nk \singleton{c}}
\end{mathpar}

Example:\\
\begin{mathpar}
\inferr
  {\vd \lambda\bind{\alpha \of \type}{\alpha} \ace \lambda\bind{\alpha \of \type}{\intt} \of \singleton{\intt} \arrow \type}
  {\inferr
    {\alpha \of \singleton{\intt} \vd
      (\lambda\bind{\alpha \of \type}{\alpha}) \alpha \ace
      (\lambda\bind{\alpha \of \type}{\intt}) \alpha \ace}
    {\inferr
      {\alpha \of \singleton{\intt} \vd
        (\lambda\bind{\alpha \of \type}{\alpha}) \alpha \whn
        }
      {\ldots \vd (\lambda{\alpha \of \type}{\alpha}{\alpha} \whr \alpha \\
       \inferr
        {\ldots \vd \alpha \whn \intt}
        {\inferr
          {\alpha \of \singleton{\intt} \vd \alpha \whr \intt}
          {\alpha \of \singleton{\intt} \vd \alpha \nk \singleton{\intt}} \\
         \inferr{\ldots \vd \intt \whn \intt}{\ldots \vd \intt \not\whr}
        }
      }
     \inferr{\alpha \of \singleton{\intt} \vd (\lambda{\alpha \of \type}{\intt}) \alpha \whn \intt}{\strut} \\
     \inferr{\alpha \of \singleton{\intt} \vd \intt \ape \intt \of \type}{\strut}
    }
  }
\end{mathpar}

One final rule: \\
\begin{mathpar}
\inferr{\Gamma \vd c_1 \ace c_2 \of \singleton{c}}{\strut}
\end{mathpar}
The precondition is that both $c_1$ and $c_2$ belong to $\singleton{c}$, meaning
they are equivalent to $c$ and by transitivity, equivalent to each other.\\

Some rules that we will never use:\\
\begin{mathpar}
\inferr{\Gamma \vd c_1 \arrow c_2 \nk \type}{\strut}

\inferr{\Gamma \vd \forall\bind{\alpha \of k}{c} \nk \type}{\strut}
\end{mathpar}

Structural rules:
\begin{mathpar}
\inferr{\Gamma \vd \alpha \ape \alpha \of k}{\Gamma(\alpha) = k}

\inferr{\Gamma \vd p\ c \ape p'\ c' \of \subst{c}{\alpha}{k_2}}
       {\Gamma \vd p \ape p' \of \Pi\bind{\alpha \of k_1}{k_2} \\
        \Gamma \vd c \ace c' \of k_1}

\inferr{\Gamma \vd \pi_1 p \ape \pi_1 p' \of k_1}
       {\Gamma \vd p \ape p' \of \Sigma\bind{\alpha \of k_1}{k_2}}

\inferr{\Gamma \vd \pi_1 p \ape \pi_1 p' \of \subst{\pi_1 p}{\alpha}{k_2}}
       {\Gamma \vd p \ape p' \of \Sigma\bind{\alpha \of k_1}{k_2}}

\inferr{\Gamma \vd c_1 \arrow c_2 \ape c_1' \arrow c_2' \of \type}
       {\Gamma \vd c_1 \ace c_1' \of \type \\ \Gamma \vd c_2 \ace c_2' \of \type}

\inferr{\Gamma \vd \forall\bind{\alpha \of k}{c} \ape \forall\bind{\alpha \of k'}{c'} \of \type}
       {\Gamma \vd k \ace k' \of \kind \\ \Gamma, \alpha \of k \vd c \ace c' \of \type}
\end{mathpar}

If $\Gamma \vd c \ape c' \of k$ then $\Gamma \vd c \nk k$ also
$\exists\bind{k'}{\Gamma \vd c' \nk k'}$ and $\Gamma \vd k \equiv k' \of \kind$\\

Structural comparison:
\begin{mathpar}
\inferr{\Gamma \vd \type \ace \type \of \kind}{\strut}
\inferr{\Gamma \vd \singleton{c} \ace \singleton{c'} \of \kind}
       {\Gamma \vd c \ace c' \of \type}

\inferr{\Gamma \vd \Pi\bind{\alpha \of k_1}{k_2} \ace
                   \Pi\bind{\alpha \of k_1'}{k_2'}}
       {\Gamma \vd k_1 \ace k_1' \of \kind \\
        \Gamma, \alpha \of k_1 \vd k_2 \ace k_2' \of \kind}

\inferr{\Gamma \vd \Sigma\bind{\alpha \of k_1}{k_2} \ace
                   \Sigma\bind{\alpha \of k_1'}{k_2'}}
       {\Gamma \vd k_1 \ace k_1' \of \kind \\
        \Gamma, \alpha \of k_1 \vd k_2 \ace k_2' \of \kind}
\end{mathpar}

% TODO different from \le triangle less/equal subkind
$\Gamma \vd k \sk k'$
\begin{mathpar}
\inferr{\Gamma \vd \type \sk \type}{\strut}

\inferr{\Gamma \vd \singleton{c} \sk \type}{\strut}

\inferr{\Gamma \vd \singleton{c} \sk \singleton{c'}}{\Gamma \vd c \ace c' \of \type}

\inferr{\Gamma \vd \Pi\bind{\alpha \of k_1}{k_2} \sk
                   \Pi\bind{\alpha \of k_1'}{k_2'}}
       {\Gamma \vd k_1' \sk k_1 \\ \Gamma, \alpha \of k_1' \vd k_2 \sk k_2'}

\inferr{\Gamma \vd \Sigma\bind{\alpha \of k_1}{k_2} \sk
                   \Sigma\bind{\alpha \of k_1'}{k_2'}}
       {\Gamma \vd k_1 \sk k_1' \\ \Gamma, \alpha \of k_1 \vd k_2 \sk k_2'}
\end{mathpar}

% TODO checks against
$\Gamma \vd k \checking \kind$
\begin{mathpar}
\inferr{\Gamma \vd \type \checking \kind}{\strut}

\inferr{\Gamma \vd \singleton{c} \checking \kind}{\Gamma \vd c \checking \type}

\inferr{\Gamma \vd \Pi\bind{\alpha \of k_1}{k_2} \checking \kind}
       {\Gamma \vd k_1 \checking \kind \\ \Gamma, \alpha \of k_1 \vd k_2 \checking \kind}
\end{mathpar}

% TODO
Suppose $\vd \Gamma \ok$. Then:\\

{\underline Soundness}\\
\begin{itemize}
\item If $\Gamma \vd c_1, c_2 \of k$ and $\Gamma \vd c_1 \ace c_2 \of k$ then
$\Gamma \vd c_1 \equiv c_2 \of k$
\item If $\Gamma \vd k_1, k_2 \of \kind$ and $\Gamma \vd k_1 \ace k_2 \of \kind$
then $\Gamma \vd k_1 \equiv k_2 \of \kind$
\item If $\Gamma \vd k_1, k_2 \of \kind$ and $\Gamma \vd k_1 \sk k_2$
then $\Gamma \vd k_1 \le k_2$
\item If $\Gamma \vd k \checking \kind$ then $\Gamma \vd k \of \kind$
\item If $\Gamma \vd c \synthesis k$ then $\Gamma \vd c \of k$
\end{itemize}

{\underline Completeness}\\
\begin{itemize}
\item If $\Gamma \vd c_1 \equiv c_2 \of k$ then $\Gamma \vd c_1 \ace c_2 \of k$
\item If $\Gamma \vd k_1 \equiv k_2 \of \kind$
then $\Gamma \vd k_1 \ace k_2 \of \kind$
\item If $\Gamma \vd k_1 \le k_2$ then $\Gamma \vd k_1 \sk k_2$
\item If $\Gamma \vd k \of \kind$ then $\Gamma \vd k \checking \kind$
\item If $\Gamma \vd c \of k$
then $\Gamma \vd c \synthesis k'$ and $\Gamma \vd k' \le \singleton{c \of k}$
\end{itemize}

\vspace{1cm}
TODO: principle type \\
TODO: principle kind is a subkind of every other kind \\

Checking principle...\\
$\Gamma \vd c \synthesis k$
\begin{mathpar}
\inferr{\Gamma \vd c \checking k}{\Gamma \vd c \synthesis k' \\ \Gamma \vd k' \sk k}
\end{mathpar}


% TODO
\begin{mathpar}
\inferr%[selfification]
       {\Gamma \vd \alpha \synthesis \singleton{\alpha \of k}}
       {\Gamma(\alpha) = k}

\inferr{\Gamma \vd \lambda\bind{\alpha \of k}{c} \synthesis \Pi\bind{\alpha \of k}{k'}}
       {\Gamma \vd k \checking \kind \\
        \Gamma, \alpha \of k \vd c \synthesis k'}

\inferr{\Gamma \vd c_1\ c_2 \synthesis \subst{c_2}{\alpha}{k'}}
       {\Gamma \vd c_1 \synthesis \Pi\bind{\alpha \of k}{k'} \\
        \Gamma \vd c_2 \checking k}

\inferr{\Gamma \vd \langle c_1, c_2 \rangle \synthesis k_1 \times k_2}
       {\Gamma \vd c_1 \synthesis k_1 \\ \Gamma \vd c_2 \synthesis k_2}

\inferr{\Gamma \vd \pi_1 c \synthesis k_1}
       {\Gamma \vd c \synthesis \Sigma\bind{\alpha \of k_1}{k_2}}

\inferr{\Gamma \vd \pi_2 c \synthesis \subst{\pi_1 c}{\alpha}{k_2}}
       {\Gamma \vd c \synthesis \Sigma\bind{\alpha \of k_1}{k_2}}

\inferr{\Gamma \vd c_1 \arrow c_2 \synthesis \singleton{c_1 \arrow c_2}}
       {\Gamma \vd c_1 \checking \type \\ \Gamma \vd c_2 \checking \type}

\inferr{\Gamma \vd \forall\bind{\alpha \of k}{c} \synthesis
                        \singleton{\forall\bind{\alpha \of k}{c}}}
       {\Gamma \vd k \checking \kind \\ \Gamma, \alpha \of k \vd c \checking \type}
\end{mathpar}

% TODO: algorithmic kind formation call what again?




\newpage
\section{Checking Expressions}

$\Gamma \vd e \synthesis \tau$
\begin{mathpar}
\inferr{\Gamma \vd x \synthesis \tau}{\Gamma(x) = \tau}

\inferr{\Gamma \vd e_1\ e_2 \synthesis \tau'}
       {\Gamma \vd e_1 \synthesis \tau_1 \\
        \Gamma \vd \tau_1 \whn \tau \arrow \tau' \\
        \Gamma \vd e_2 \checking \tau}
\end{mathpar}

\section{Type-Directed Translation / Syntax-Directed Translation}
A more accurate name: ``Typing-derivation-directed translation''.
We proceed by the analysis of the typing derivation of the rules.

\newcommand{\target}[1]{\textcolor{green}{#1}}
\newcommand{\tto}{\translatesto}
Let's represent the source and target languages in different colors, to
indicate that they are different.\\

Property:\\
$\Gamma \vd e \of \tau$ if and only if $\exists\bind{\target{e}}{\Gamma \vd e \of \tau \translatesto \target{e}}$.\\

We also want:\\
If $\Gamma \vd e \of \tau \tto \target{e}$, something like
$\target{\Gamma \vd e \of \tau}$.\\
But we have no concept of $\target{\Gamma}$ or $\target{\tau}$ or its derivations.\\
Instead:\\
Property:\\
If $\Gamma \vd e \of \tau \tto \target{e}$
and $\tau \tto \target{\tau}$
and $\Gamma \tto \target{\Gamma}$,
then $\target{\Gamma \vd e \of \tau}$. \\
Why not $\Gamma \vd \tau \of \type \tto \tau$.\\

Simply, we'll use `` If $\Gamma \vd e \of \tau \tto \target{e}$
then $\target{\Gamma} \vd \target{e} \of \target{\tau}$

\subsection{Coherence}
For Terms:\\
Suppose $\Gamma \vd e \of \tau \tto \target{e}$ and
$\Gamma \vd e \of \tau \tto \target{e'}$.\\
$\target{\Gamma \vd e \cong e' \of \tau}$.\\
This is too hard to even define, this is left to graduate courses.
We aspire to it but it's too much of a pain to actually do.\\

For Types:\\
Suppose $\Gamma \vd c \of k \tto \target{c}$ and
$\Gamma \vd c \of k \tto \target{c'}$.\\
Then,\\
$\Gamma \vd c \equiv c' \of k$. \\
This is not an aspiration, we cannot live without this. \\
The 2nd property above can't even be made without this, but it doesn't have
to be kind directed. And instead, we'll just make it syntax directed, which
will trivially prove that the two are equivalent.


\subsection{Definition of $\target{e}$}
\begin{align*}
\target{\alpha} &= \alpha \\
\target{\tau_1 \times \tau_2} &= \target{\tau_1} \times \target{\tau_2} \\
\target{\tau_1 \arrow \tau_2} &= \unit \arrow \target{\tau_1} \arrow \target{\tau_2} \\
&\ldots \\
\target{\epsilon} &= \epsilon \\
\target{\Gamma, x \of \tau} &= \target{\Gamma}, x \of \target{\tau} \\
\target{\Gamma, \alpha \of k} &= \target{\Gamma}, \alpha \of \target{k} \\
\end{align*}

Convoluted example:\\
\begin{mathpar}
\inferr{\Gamma \vd \lambda\bind{x \of \tau}{e} \of \tau \arrow \tau'
                    \tto \lambda\bind{z \of \unit}{\lambda\bind{x \of \target{\tau}}{\target{e}}}}
       {\Gamma \vd \tau \of \type \\
        \Gamma, x \of \tau \vd e \of \tau \tto \target{e}}
\end{mathpar}
NOTE: the right part (after $\tto$) indicates the need to shift in terms of
debruijn indices. \\

More: \\
\begin{mathpar}
\inferr{\Gamma \vd e_1\ e_2 \of \tau' \tto \target{e_1} <> \target{e_2}}
       {\Gamma \vd e_1 \of \tau \arrow \tau' \tto
          \target{e_1}^{\of \target{\tau_1 \arrow \tau_2}
                          = \unit \arrow \target{\tau} \arrow \target{\tau'}} \\
        \Gamma \vd e_2 \of \tau \tto \target{e_2}^{\of \target{\tau}}}
\end{mathpar}

More (only well typed things translate): \\
\begin{mathpar}
\inferr{\Gamma \vd e_1\ e_2 \synthesis \tau' \tto \target{e_1} <> \target{e_2}} % TODO what is <>
       {\Gamma \vd e_1 \synthesis \tau_1 \tto \target{e_1} \\
        \Gamma \vd e_1 \whn \tau \arrow \tau' \\
        \Gamma \vd e_2 \synthesis \tau_2 \tto \target{e_2} \\
        \Gamma \vd \tau_2 \ace \tau \of \type}
        % NOTE: we don't need this last one, we can optimize by skipping this rule
\end{mathpar}


\subsection{$\Gamma \vd e \of \tau \translatesto \target{e}$}
\begin{mathpar}
\inferr{\Gamma \vd x \of \tau \translatesto \target{x}}{\Gamma(x) = \tau}

\inferr{\Gamma \vd \langle e_1, e_2 \rangle \of \tau_1 \times \tau_2
                        \tto \target{\langle e_1, e_2 \rangle}}
       {\Gamma \vd e_1 \of \tau_1 \tto \target{e_1} \\
        \Gamma \vd e_2 \of \tau_2 \tto \target{e_2}}

\inferr{\Gamma \vd \pi_1 e \of \tau_1 \tto \target{\pi_1 e}}
       {\Gamma \vd e \of \tau_1 \times \tau_2 \tto \target{e}}
\end{mathpar}


\end{document}
